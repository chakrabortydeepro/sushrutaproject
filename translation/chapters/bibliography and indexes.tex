% !TeX encoding = UTF-8
% !TeX root = incremental_SS_Translation.tex
\frenchspacing % don't want to fuss about end-of-sentence spacing in glossaries

%\nocite{*} % include everything from the bib file in the bibliography

\printbiblist[%
heading=biblistintoc,
title=Editions and Abbreviations,
notkeyword=botanical
%
]
{shorthand} % the Abbreviations (shorthands)

\indexprologue{\emph{\footnotesize Numbers after the final 
        colon refer to pages in this book.}} 

\printindex[manuscripts]

\printbibliography[notkeyword=edition,
notkeyword=shorthand,
notkeyword=botanical,
heading=bibintoc]

\newpage

\begin{footnotesize}
    
% Skt-Eng and Eng-Skt of \se{}{} words
%\printindex[lexical]  
%\chapter{Materia Medica}    
%    
%    \defbibheading{bibliography}[\bibname]{%
%        \subsection*{#1}%
%        \addxcontentsline{toc}{subsection}{#1}}
       
       
    \chapter{Materia Medica}
        
   
% Take the heading of the Abbreviations down from chapter to section:
\defbibheading{biblistintoc}[\bibname]{%
       \addcontentsline{toc}{section}{#1}%
       \section*{#1}%
       \markboth{#1}{#1}}
       
% \bibfont{\footnotesize }    
\printbiblist[
    heading=biblistintoc,
    title=Abbreviations,
    keyword=botanical]{shorthand}
    
%\renewcommand{\glossaryname}{Flora and Fauna}
%\newpage
% \renewcommand{\glossaryname}{Materia Medica}
   
% If you want to print all the glossary entries, 
% use the selection=all option in \GlsXtrLoadResources (see 
% xelatex-glossaries.sty.
%
%\printunsrtglossaries % plant names.  iterate over all defined entries 
%% the setup for the glossaries package is in xelatex-indexing etc.
%%

    \printunsrtglossary[type=plants]
\printunsrtglossary[type=animals]
    
\clearpage % prevent the page heading “Glossary” printing on previous page
    \chaptermark{Glossary}\sectionmark{Glossary}
    \printindex[lexical]  
%
    
    \clearpage
% Todo list
\end{footnotesize}

\thispagestyle{empty}