% !TeX root = ../incremental_SS_Translation.tex

\chapter{Introduction}

What follows is a translation of selected chapters of the
\emph{Compendium of Suśruta} (\emph{Suśrutasaṃhitā}).  This differs
from former translations, being based on the Nepalese
version of the text.\footnote{See \cite{wuja-2023} for an
    introduction to the Nepalese text and \cite{wuja-2021b} for
    background on the Suśruta Project, 2021--2024.}  The Nepalese version of the 
    work has been reconstructed on the basis of three manuscripts from 
    Kathmandu, 
    \begin{enumerate}
        \item \MS{Kathmandu KL 699} (siglum K),
        \item \MS{Kathmandu NAK 1-1079} (N), and
        \item \MS{Kathmandu NAK 5-333} (H).
    \end{enumerate}
The first of these MSS is the oldest, dated to
\CE~878.\footcite[15]{kleb-2021b}  It covers most of the \SS, but
lacks the \emph{Nidānasthāna} and the \emph{Śārīrasthāna} (see 
Fig.\,\ref{fig:mss-1-visual-paradigm}).  The
second is undated but is datable on palaeographical grounds to the
twelfth or thirteenth centuries.\footcite[17--18]{kleb-2021b} It
contains the \emph{Sūtrasthāna} and \emph{Nidānasthāna} but breaks
off shortly afterwards.  The third manuscript is the most complete,
covering the whole of the \SS. It is dated \CE~1513.\footnote{I
    follow the arguments of \citet[21--26]{kleb-2021b} on the
    interpretation of the colophon although, as he pointed out, some
    interpret the date as \CE\ 1573.}  
    
The text of this manuscript follows K very closely but is probably
not a direct apograph.\footcite{chak-2022} I conjecture that it was
either copied from an intermediary that followed K very closely or
from a ancestor of K.\footnote{“\ldots as neither my own research
    \ldots\ nor the study undertaken in Harimoto \ldots\ could determine
    any linear connection between any of the Nepalese manuscripts of the
    SS, one may assume that [there exists] an older common ancestor of
    both of the manuscripts K and H.” \citep[21]{kleb-2021a}.}
        
        \begin{figure}[t]
            \centering
            \includegraphics[width=\textwidth]{"media/MSS 1 visual paradigm.art"}
            \caption{Coverage of the text by MSS K, N and H.}
            \label{fig:mss-1-visual-paradigm}
        \end{figure}
        
    
    
The translation follows the methods of rigorous philological care and 
modern principles of translation theory.\footnote{See 
\cite[intro.]{wuja-2003} and \cite[81--83]{wuja-2021} for an overview.}  Major 
differences in sense from the vulgate text are marked  \diff{in this 
manner}.

The text-historical state of the \SS\ bears many resemblances to other early 
textual transmissions in South Asia.  The situation was articulated particularly 
clearly for the case of Pāli by \citet{hinu-1978}, in the opening of his chapter, 
\begin{quote}
    \ldots we cannot go back beyond the council of Aluvihāra (Ālokavihāra) under 
    Vaṭṭagāmaṇī Abhaya (29--17 B.C.) where the Pāli canon ws written down for 
    the first time in Ceylon.  This is the very starting point of our tradition 
    handed down to us by the monks of the Mahāvihāra.  About recensions of 
    the Pāli canon different from the Mahāvihāra tradition and deviating from its 
    wording\ldots\ we scarcely have any knowledge at all.
\end{quote}
Similarly, the manuscript evidence for the \SS\ that is available
today allows us to reconstruct a version of the work after it was
consolidated into a text of five parts with a sixth or “later”
(\emph{uttara}) and somewhat different part already appended to the
first five.  The prehistory of the work before this form is
tantalizingly unknown to us.  That the work was assembled from
diverse sources and that many hands were involved is without doubt.
The oldest surviving manuscript, \MS{Kathmandu KL 699}, gives us
physical evidence for the state of the text in the ninth century.  We
little insight into the formational processes affecting the text
before that time.  But what we can see plainly is that the text was
edited pervasively after that time, being influenced especially by
the commentators Jejjaṭa, Candraṭa, Gayadāsa and Cakrapāṇidatta and
the editor Candraṭa. However, a
clear picture of how these later editorial processes took place will only be 
possible as a result of further research into a wider manuscript base.