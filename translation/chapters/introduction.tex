% !TeX root = ../incremental_SS_Translation.tex

\chapter{Introduction}

What follows is a draft translation of selected chapters of the
\emph{Compendium of Suśruta} (\emph{Suśrutasaṃhitā}).  This differs
from former translations, being based on the text that survives in the
oldest known manuscripts of the work.\footnote{See \cite{wuja-2023}
    for an introduction to the Nepalese text and \cite{wuja-2021b} for
    background on the Suśruta Project, 2021--2024.}  These old manuscripts
    are located in Nepal, so we refer to this as “the Nepalese version” of
    the work, although future research may show that this old version was
    more widely known.\footnote{For more discussion of this issue, see 
    \cite[Introduction and ch.\,2]{wuja-2023}.}  
   
\section{The Nepalese Version}    

The Nepalese version has been reconstructed on the basis of three
manuscripts from Kathmandu,
    \begin{enumerate}
        \item \MS{Kathmandu KL 699} (siglum K),
        \item \MS{Kathmandu NAK 1-1079} (N), and
        \item \MS{Kathmandu NAK 5-333} (H).
    \end{enumerate}
The first of these MSS is the oldest, dated to
\CE~878.\footcite[15]{kleb-2021b}  It covers most of the \SS, but
lacks the \emph{Nidānasthāna} and the \emph{Śārīrasthāna} (see
Fig.\,\ref{fig:mss-1-visual-paradigm}).  The second is undated but is
datable on palaeographical grounds to the twelfth or thirteenth
centuries.\footcite[17--18]{kleb-2021b} It contains the
\emph{Sūtrasthāna} and \emph{Nidānasthāna} but breaks off shortly
afterwards.  The third manuscript, H, is the most complete, supporting
the text of the whole of the \SS. It is dated \CE~1513.\footnote{I
    follow the arguments of \citet[21--26]{kleb-2021b} on the
    interpretation of the colophon although, as he pointed out, some
    interpret the date as \CE\ 1573.} %
    The text of manuscript H follows K very closely but is probably
    not a direct apograph.\footcite{chak-2022} I conjecture that it
    was either copied from an intermediary that followed K very
    closely or from a ancestor of K.\footnote{“\ldots as neither my
        own research \ldots\ nor the study undertaken in Harimoto \ldots\
        could determine any linear connection between any of the Nepalese
        manuscripts of the SS, one may assume that [there exists] an older
        common ancestor of both of the manuscripts K and H.”
        \citep[21]{kleb-2021a}.}
        
        \begin{figure}[t]
            \centering
            \includegraphics[width=\textwidth]{"media/MSS 1 visual paradigm.art"}
            \caption{Coverage of the text by MSS K, N and H.}
            \label{fig:mss-1-visual-paradigm}
        \end{figure}
        
\section{The vulgate}

The version of the \SS\ that we refer to as “the vulgate” is the
version of the text that circulates in print today in multiple
editions.  The most careful and authoritative edition is that of
\citet{vulgate}.\footnote{This and the following issues have been
    discussed by \citet[2 and ch.\,3]{wuja-2023}.}  It is telling that
    this edition includes the commentary of Ḍalhaṇa (b.\ ca.\ 1175) and,
    for the \emph{Nidānasthāna}, also that of Gayadāsa (fl.\ ca.\ 1000).
    These important authors commented on a text that is, broadly speaking,
    what we call “the vulgate.”  But they both mentioned quite often that
    the manuscripts they were consulting contained other versions of the
    text and in a high number of cases, these variations match the
    Nepalese version.\footnote{E.g., see the discussion in footnote
        \ref{ancient-variants} below.}  It is possible that Gayadāsa  and Ḍalhaṇa,
        through their commentarial work on the text, participated in shaping
        “the vulgate."

The scholar Rudolph Hoernle was also aware of this cleavage in the
transmission-history of the \SS.  But with the more limited materials
available to him at the turn of the twentieth century he drew the line
a little differently.  He referred to the text of the
\emph{Śārīrasthāna} of the \SS, transmitted in the printed editions of his day, as
“the Traditional Recension."
\begin{quote}
    The recension which is found in Jīvānanda's and all other
prints,\footnote{Hoernle listed four,
    \cites{bhat-1889,gupt-1835,govi-1901,vira-nd}.} and which, in the
    sequel, will be referred to as the Traditional Recension, has in
    its favour not only all available manuscripts, but also all
    ancient commentaries on the Compendium of Suśruta, \ldots.  Or,
    shortly, the Traditional Recension is supported by the whole body
    of existing witnesses.\footcite[68]{hoer-1907}
\end{quote}
However, Hoernle was unfortunately not aware of the Nepalese
manuscripts of the \SS, which at the time he was writing were in
Nepalese libraries that had not yet been explored by scholars of the
time.  The contrast that Hoernle was drawing was between the
Traditional Recension and the \emph{Śārīrasthāna} of the \CS\ as
printed by the influential Bengali scholar, Kavirāja Gaṅgādhara Ray
(1798--1885).\footnote{\cite{gang-1868}.  Hoernle's evaluation of this
    edition was not entirely kind: “I have not been able to discover for
    it any authority whatsoever. \ldots\ it is probably that the recension
    of Gangādhar is a reconstruction of his own to meet those of the
    difficulties which he had noticed” \citep[70]{hoer-1907}.  For a full
    account of the genesis of this edition, see \cite{pecc-2022}.}

    
\section{The Translation}    
The translation follows the methods of rigorous philological care and
modern principles of translation theory.\footnote{See
    \cite[intro.]{wuja-2003} and \cite[81--83]{wuja-2021} for an
    overview.}  Major differences in sense from the vulgate text are
    marked  \diff{in this manner}, but the differences are so pervasive
    and fine-grained that most have not been explicitly marked.

The text-historical state of the \SS\ bears many resemblances to other early 
textual transmissions in South Asia.  The situation was articulated particularly 
clearly for the case of Pāli by \citet{hinu-1978}, in the opening of his chapter, 
\begin{quote}
    \ldots we cannot go back beyond the council of Aluvihāra (Ālokavihāra) under 
    Vaṭṭagāmaṇī Abhaya (29--17 B.C.) where the Pāli canon ws written down for 
    the first time in Ceylon.  This is the very starting point of our tradition 
    handed down to us by the monks of the Mahāvihāra.  About recensions of 
    the Pāli canon different from the Mahāvihāra tradition and deviating from its 
    wording\ldots\ we scarcely have any knowledge at all.
\end{quote}
Similarly, the manuscript evidence for the \SS\ that is available
today allows us to reconstruct a version of the work after it was
consolidated into a text of five parts with a sixth or “later”
(\emph{uttara}) and somewhat different part already appended to the
first five.  The prehistory of the work before this form is
tantalizingly unknown to us.  That the work was assembled from diverse
sources and that many hands were involved is without doubt. The oldest
surviving manuscript, \MS{Kathmandu KL 699}, gives us physical
evidence for the state of the text in the ninth century.  We little
insight into the formational processes affecting the text before that
time.  But what we can see plainly is that the text was edited
pervasively after that time, being influenced especially by the
commentators Jejjaṭa, Candraṭa, Gayadāsa and Cakrapāṇidatta and the
editor Candraṭa. However, a clear picture of how these later editorial
processes took place will only be possible as a result of further
research into a wider manuscript base.