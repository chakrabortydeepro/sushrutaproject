% !TeX root = incremental_SS_Translation.tex

%  Hello, Lisa!

\chapter{Sūtrasthāna 13:  On Leeches}


\section{Literature} 


Meulenbeld offered an annotated overview of this chapter and a bibliography
of studies on Indian leeches and their application.\footcite[IA, 209; IB,
324, n.\,131]{meul-hist}

A Persian version of this chapter of the \SS\ was included in \emph{Sikandar
    Shāh's Mine of Medicine} (\emph{Ma`din al-\underbar{sh}ifā' i
    Sikandar-\underbar{Sh}āhī}) composed in 1512 by Miyān Bhūwah b.
\underline{Kh}awāṣṣ \underline{Kh}ān.\footcites[96--109]{sidd-1959}
{azee-1971} [231--232]{stor-1971} [IB, 324,
n.\,128]{meul-hist}[8--9]{spez-2019}

More recently Brooks has examined this chapter and leech therapy more 
broadly terms of leeches and classification, multispecies agencies, and the 
tactile and intersensory dynamics of leech therapy.\footcite{%
    broo-2020,
    broo-2020b,
    %broo-2018,
    broo-2020c,
    broo-2021}

\section{Translation}

\begin{translation}    
\item [1] 
    And now we shall explain \diff{the chapter} about leeches.
    
\item [2]The leech is for the benefit of kings, rich people, 
delicate people,
children, the elderly, fearful people and women.  It is said to be the most
gentle means for letting blood. 

\item [3]

In relation to that, one should let blood that is corrupted by wind, bile or
phlegm with a horn, a leech, or a \gls{alābu}, respectively.   Or, each kind
can be made to flow by any of them in their particular way.\footnote{This
    sentence is hard to construe grammatically, although its meaning seems
    clear. In place of \dev{viśeṣastu}, Cakrapāṇidatta and Ḍalhaṇa both read
    \dev{viśeṣatas}, which helps interpretation (\cite[95]{acar-1939},
    \cite[55]{vulgate}). It is noteworthy that the critical syllable \dev{stu} is
    smudged or corrected in both \MScite{Kathmandu NAK 1-1079} and in 1-1146, a
    much later Devanāgarī manuscript.\MSsilent{Kathmandu NAK 1-1146}
      
There is an insertion in the text, printed in parentheses in the
vulgate at \Su{1.13.4}{55} as  \dev{viśeṣatastu visrāvyaṃ
śṛṅgajalaukālābubhirgṛhṇīyāt}.  This insertion is not included in the
earlier edition of the vulgate, but is replaced by
\dev{snigdhaśītarūkṣatvāt} \citep[54]{susr-trikamji2}. Ḍalhaṇa noted that,
“this reading is discussed to some extent by some compilers
(\dev{nibandhakāra}), but it is definitely rejected by most of them,
including Jejjhaṭa.” }



\item [4]
And there are the following about this:

\begin{sloka}
    A cow's horn is praised for being unctuous, \diff{smooth}, and very
sweet.  Therefore, when wind is troubled, that is good for
bloodletting.\footnote{The vulgate replaced “smooth” with “hot.”}
\end{sloka}

\item [5]

\begin{sloka}
    A horn shaped like a half-moon,
    with a large body the length of seven fingers , should first be placed on the 
    incision.  A strong person should 
    suck with
the mouth.\footnote{This passage is not found in the vulgate, but it is
    similar to the passage cited by Ḍalhaṇa at \Su{1.13.8}{56} and attributed to
    Bhāluki.  Bhāluki was the author of a \emph{Bhālukitantra} that may have
    predated Jejjaṭa and might even have been one of the sources for the \SS\
    \pvolcite{IA}[689--690 \emph{et passim}]{meul-hist}. The editor Ācārya was
    aware of this reading in the Nepalese manuscripts; see his note 4 on
    \Su{1.13.5}{55, note 4}.}
\end{sloka}

\item[6]

\begin{sloka}
    A leech lives in the cold, is sweet and is born in the water. So when
someone is afflicted by bile, they are suitable for
bloodletting.\footnote{Note that the particular qualities (\emph{guṇa}s) of
    the leech in this and the following verses counteract the quality of the
    affliction.  See \cite[113, table 1]{broo-2018}.}
\end{sloka}

\item[7]

\begin{sloka}
    A \gls{alābu} is well known for being pungent, dry and sharp.  So
when someone is afflicted by phlegm it is suitable for bloodletting.

\end{sloka}
\item[8]

In that context, at the scarified location one should let blood using a horn
wrapped in a covering of a thin bladder, or with a \gls{alābu} with a flame
inside it because of the suction.\footnote{There are questions about the
    wrapping or covering of the horn.  Other versions of the text, and the
    commentator, propose that there may be two coverings, or that cloth may be a
    constituent. 
%    Comparison with contemporary horn-bloodletting practice by
%    traditional Sudanese healers suggests that a covering over the top hole in
%    the horn is desirable when sucking, to prevent the patient's blood entering
%    the mouth \citep{pbs-2020}.  
    Our understanding of this verse is that the
    bladder material is used to cover the mouthpiece and then to block it, in
    order to preserve suction in the horn for a few minutes while the blood is
    let. }

\item[9]

Leeches are called “\emph{jala-āyu-ka}” because their \se{āyu-}{life}
is in \se{jala}{water}.\footnote{The lexeme \emph{-āyu-} is known almost 
exclusively from the \emph{Ṛgveda}.} “Home” (\emph{okas}) means 
“{dwelling};” their home is water, so they are called 
“\se{jalaukas}{water-dwellers}.”

\item[10]

There are twelve of them: six are venomous and just the same number are 
non-venomous. 

\item[11]
Here is an explanation of the venomous ones, together with the therapy:
\begin{itemize}
    \item \se{kṛṣṇā}{Black}
    \item \se{karburā}{Mottled}
    
    \item \se{alagardā}{Sting-gush}\footnote{Treating \dev{gardā} as
    \dev{galdā} and translating as in RV 8.1.20, with \citet[1023, verse 20
    and cf.\ commentary]{jami-2014}. But if \dev{garda} is to be taken from
    $\surd$\dev{gard} then we might have “crying from the sting.”}
    
    \item \se{indrāyudhā}{Rainbow} 
    
    \item \se{sāmudrikā}{Oceanic} \item
\diff{\se{govandanā}{Cow-praising}}\footnote{The manuscripts all read
    \dev{govandanā} against the vulgate's \dev{gocandanā}.}
\end{itemize}

Among these, 
\begin{itemize}
    \item The one called a Black is the colour of kohl and has a broad head;

    \item The one called Mottled is like the \gls{varmimatsya}, long with a
\se{chinna}{segmented}, humped belly.
    
    \item  The one called Sting-gush is hairy, has large sides and a black mouth.
    
    \item  The one called Rainbow is coloured like a rainbow, with vertical stripes.
    
    \item  The one called Oceanic is slightly blackish-yellow, and is covered with
    variegated flower patterns. 
    
    \item The one called Govandana is like a cow's testicles, having a bifurcated 
    appearance on the lower side, and a tiny mouth. 
       
\end{itemize}
When someone is bitten by them, the symptoms are: a swelling at the site of 
the
bite, excessive itching and fainting, fever, a temperature, and vomiting. In
that context the \se{mahāgada}{Great Antidote} should be applied in drinks 
and
\se{ālepana}{liniments}, etc.\footnote{Ḍalhaṇa and the vulgate included 
errhines
    in the list of therapies, and Ḍalhaṇa added that “etc.” indicated sprinkling and
    immersion too.   The “Great Antidote” is described in the Kalpasthāna, at
    \Su{5.5.61--63ab}{578}.} A bite by the Rainbow leech is not treatable.  
    These
    venomous ones have been explained together with their remedies.

\item[12]

Now the ones without venom.\footnote{The translations of the names of these 
leeches are slightly whimsical, but give a sense of the original; \dev{sāvarikā} 
remains etymologically puzzling.} 
\begin{itemize}
    \item \se{kapilā}{Tawny}
    \item \se{piṅgalā }{Ruddy}
    \item \se{śaṅkumukhī }{Dart-mouth}
    \item \se{mūṣikā }{Mouse}
    \item \se{puṇḍarīkamukhī}{Lotus-mouth}
    \item \se{sāvarikā }{Sāvarikā}
\end{itemize}
Among these,
\begin{itemize}
    \item The one called Tawny has sides that look as if they are dyed with
realgar and is the colour of glossy mung beans on the back.\footnote{The
    compound \dev{snigdhamudgavarṇṇā} is supported by all the manuscript
    witnesses and is translated here.  Nevertheless, the reading of the
    vulgate, that separates \dev{snigdhā}, f., “slimy” as an adjective for the
    leech, seems more plausible: “it is slimy and the colour of a mung
    bean.”}
    
    \item The one called Ruddy is a bit red, has a round body, is yellowish, and 
    moves fast.
    
    \item The one called Dart-mouth is the colour of liver, drinks fast and has a long 
    mouth.
    
    \item The one called Mouse is the colour and shape of a mouse and has an 
    undesirable smell.
    
    \item The one called Lotus is the colour of mung beans and has a mouth that looks 
    like a lotus.
    
    \item The one called Sāvarikā has the colour of a lotus leaf and is eighteen 
    centimetres long.  But that one is used when the purpose is an animal. 
\end{itemize}
The non-venomous ones have been explained.

\item [13]

Their lands are Yavana, Pāṇḍya, Sahya, Potana and so on.\footnote{This
    passage is discussed by \citet[109--110, 388--389]{kart-2015}.  At the time
    of the composition of the \SS, Yavana would most likely have referred the
    Hellenistic Greek diaspora communities in Bactria and India
    \parencites[136--137]{law-1984}{mair-2013}{mair-2014}. Unproblematically,
    the Pāṇḍya country is the extreme south-eastern tip of the Indian
    subcontinent \citep[E8, p.\,20 \emph{et passim}]{schw-1978}, and Sahya
    refers to the Western Ghats \citep[D5--7, p.\,20 \emph{et
    passim}]{schw-1978}.  The vulgate reading “Pautana” is not a known toponymn.
    Potana was the ancient capital of the Aśmaka Mahājanapada mentioned in Pali
    sources and in inscriptions at Ajāntā and elsewhere, and identified by
    \textcites[142, 179]{law-1984}[18]{gupt-1989} with Pratiṣṭhāna, modern
    Paithan on the Godavari river.  The recurring ancient epithet describing the
    Aśmaka kingdom is that it was on the Godāvarī, and Paithan is flanked to the
    south west and south east by this river.
    
    Some scholars have identified the name with modern Bodhan in Telangana
\parencites[189]{sirc-1971}[E6, p.\,14, 140 \emph{et
passim}]{schw-1978}[102]{sen-1988}, but this implausible identification
is traceable to a speculative suggestion by \citet[89, n.\,5,
143]{rayc-1953} based on a variant form “Podana” found in some early
manuscripts of the \emph{Mahābhārata}: “This name reminds one of Bodhan
in the Nizam's dominions,” “possibly to be identified with Bodhan.”
    
         Ḍalhaṇa on \Su{1.13.13}{57} anachronistically identified “Yavana”
as the land of the Turks (\dev{turuṣka}) and “Pautana” as the
Mathurā region.  He also noted, as did Cakrapāṇidatta
\citep[97]{acar-1939}, that this passage was not included by some
authorities on the grounds that the habitats of poisonous and
non-poisonous creatures are defined by other criteria.}  Those in
particular have large bodies and are strong, they drink rapidly,
consume a lot, and are without venom.

\item [14]

In reference to that, venomous leeches are those originating in decomposing
venomous insects, frogs, urine, feces and in polluted water.\footnote{The
    vulgate on \Su{4.13.14}{57} includes fish in this list.}  Non-venomous
    ones originate in decomposing \gls{padma}, \gls{utpala}, \gls{kumuda},
    \gls{saugandhika}, \gls{śevāla} and in pure waters.

\item[15]
There is a verse on this:

\begin{sloka}
    These ones move about in sweet-smelling habitats that abound in water.
Tradition teaches that they do not behave in a confused manner or lie in the
mud.\footnote{Ḍalhaṇa on \Su{1.13.14}{57} discussed why non-venomous 
leeches
    would not “behave in a confused manner” (\dev{saṅkīrṇacārin}), saying that
    they do not “eat a diet that is contra-indicated because of poison etc.”
    (\dev{viṣādiviruddhāhārabhujaḥ}).  On the use of the term \dev{viruddha} in
    the sense of “incompatible,” see \Su{4.23.4}{485}. Ḍalhaṇa there noted that
    such foods are explained in the chapter on wholesome and unwholesome 
    foods
    (\dev{hitāhitādhyāya}, \Su{1.20}{94--99}).}
\end{sloka}

\item[16]

They can be caught with a fresh hide or one may catch them by other 
means.\footnote{“Fresh hide” (\dev{ārdracarman}) may suggest that the 
animal skin still 
includes
meat or blood that is attractive to a leech.

   Ḍalhaṇa on \Su{1.13.15}{57} quoted “another treatise”
(\dev{tantrāntaravacanāt}) that said that autumn is the time to collect
leeches.  He also explained that “other methods” of collecting leeches
included smearing a leg or other limb with cream, butter or milk, etc.,
or using a piece of flesh from a freshly killed animal.

The Nepalese witnesses all read \dev{gṛhītvā} “having (been)
caught” for the vulgate's \dev{gṛhṇīyāt} “one may grasp (by other means).” 
The Nepalese reading is hard to construe and we have emended to the 
vulgate's reading.}


\item [17] 

Then these should be put into a large new pot furnished with mud
and the water from lakes or wells. One should provide what they need to eat.
One should grind up \gls{śevāla}, \gls{vallūra}, and aquatic tubers, and one
should give them grass and aquatic leaves to lie on, and every three days
water and food.
%\footnote{In NIA, \dev{jalabhakta} refers to rice that is
%    fermented overnight.  See also Rājaśekhara ch. 18 p. 265.
%    
%\url{https://archive.org/details/BivY_kavya-mimamsa-of-raj-shekhar-commentary-by-ganga-sagar-rai-chowkhamba-vidya-bhawan-varanasi/page/n343/mode/2up?view=theater}
%     But this seems implausible in the context of leeches, that need fresh water.  The 
%    vulgate dissolved the compound, removing the ambiguity.}  
After seven nights one should transfer them to a different pot.

\item[18]

And on this:

\begin{sloka}
    One should not \diff{nurture} those that are thick in the middle, that are
injured,\footnote{\emph{Pace} Ḍalhaṇa on \Su{1.13.18}{57} who glossed
    \dev{parikliṣṭa} “injured” as \dev{amanojñadarśana} “disagreeable
    looking.”} or \diff{small}, those that are not born in the proper habitat,
    those that will not attach, that drink little or those that are venomous.
\end{sloka}

\item[19] 

First of all, if the patient  has an ailment that is treatable by bloodletting
with leeches, get them to sit or lie down.  Then, dry \diff{any 
    \se{avakāśa}{place} that is diseased} with powders of earth and 
    cow-dung.\footnote{\Dalhana{1.13.19}{57} read
    \dev{arujam} (n.), against the vulgate's \dev{arujaḥ}; \Cakra{this 
    verse}{98}
    read \dev{arujaḥ}. Both commentators specified that the \SS\ said this 
    procedure
    should only be applied when there is no wound or opening, for fear of
    exacerbating the condition.  The Nepalese text is saying, differently, that the
    desiccating powders should be applied to a diseased wound. }  
    
    
Then the leeches, free from impurities, with their bodies smeared with
\gls{sarṣapa} and \gls{rajanī}, moving about in the middle of a cup of water,
should be made to attach to the site of the ailment.  Now, for one that is not
attaching, one should provide a drop of milk or a drop of blood. Alternatively,
one should make some \se{śastrapada}{marks with a
    knife}.\footnote{\label{pada-leeches}On \dev{pada} as a “mark,” “imprint,” 
    or
    “place of application,” cf.\ \Su{4.1.29}{399}, \Su{5.4.15}{571}, etc.  See
    footnote \ref{pada-snakes}.} And if it still will not attach, make a different 
    one
    attach.

\item [20]
 
One can know that it is attached when it fixes on, making its mouth like a 
horse's hoof and hunching its neck. Then, one should cover it with a wet
cloth and keep it there.
 
 \item[21]
 
Now, if one knows, from the arising of pricking and itching at the
bite, that clean blood is being taken, one should take it off.  Then,
if it does not release because of the scent of blood one should sprinkle its
mouth with powdered \gls{saindhava}.
    
\item[22]

Then one should coat it with \gls{śālitaṇḍulakāṇḍana}, rub its mouth with
sesame oil and salt and cause it to vomit by holding its tail in the left
hand and very slowly rubbing it with the thumb and finger of the right hand
in the proper direction, as far as the mouth, until it is properly purged.\footnote{The 
expression \dev{śālitaṇḍulakāṇḍana}, “rice-grain chaff” could be read as “\gls{śāli}, 
\gls{taṇḍula} and \gls{kāṇḍana}” but this seems unlikely in the context.} 
A properly purged leech placed in a goblet of water moves about, wanting to
eat.  If it sinks down, not moving, it is badly purged; one should make it vomit
once again.  

A badly purged leech develops an incurable disease called 
Indrapada.\footnote{At
    this point, the Nepalese witnesses read \dev{indrapada}/\dev{indrāpada}, 
    but the
    vulgate reads \dev{indramada}, a term that is found in other texts such as 
    the
    \cite{Manasollasa} 6.641 (vol.\,1, 87), where it is a fever affecting fish, and
    the \cite{Garudapurana} 1.147.3 (tr.\ \volcite{2}[425]{garuda}) where it is
    fever affecting clouds; see further \cite{broo-inpress}.} 
    
\item[23]
    
    \begin{sloka}
        \diff{One that protects its deflated head with its body, suddenly curls up
    and makes the water warm is traditionally said to have
    Indrapada.}\footnote{At this point, witness H, the latest MS, reads
    \dev{indrapada} as before, but the older witnesses K and N have muddled
    readings, \dev{idamadaḥ} and \dev{idramadaḥ}.  The scribes may have 
    been
    responding to a \dev{-pada/-mada} confusion about the name of this
    condition.}
    \end{sloka}
\diff{Thus, one should keep such a one as before.}\footnote{The vulgate
    includes “well purged” as the object in this sentence, which makes better
    sense.}

\item[24]

\diff{After observing the proper or improper flow of the blood, one
    should rub  the opening made by the leech with honey.\footnote{In the
        Nepalese witnesses, the object of this passage is \dev{jalaukāmukham}
        “the mouth of the leech,” that we have interpreted, perhaps freely, as
        “opening made by the leech.” Logically and as transmitted in the vulgate,
        this passage should be about managing the wound on the patient that has
        been made by the leech.}  Alternatively, one may bind it up and smear it
        with ointments that are astringent, sweet, oily and cold.}
            
%    The
%    vulgate expands this passage with several other techniques.}

\item[25]
And about this there is the following:
\begin{sloka}
     \diff{When the leeches have just drunk, one should pour ghee on it.
    And one should pour on to the blood things that are capable of
    stopping the blood.}
\end{sloka}

\item[26]

\begin{sloka}
    Someone who knows habitats, the capture, feeding and bloodletting of
leeches is worthy to treat a king.
\end{sloka}

\end{translation}



% % % % % % % % % % % % % % % % % % % % % % % SS 1.28
