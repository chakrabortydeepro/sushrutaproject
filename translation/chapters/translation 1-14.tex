% !TeX root = ../incremental_SS_Translation.tex

\chapter{Sūtrasthāna 14:  On the Properties of Blood}


\subsection{Previous scholarship}

Meulenbeld offered both an annotated summary of this chapter as well as a
study specifically on the place of blood in Ayurvedic
theory.\footnote{\volcite{IA}[209--201]{meul-hist}  and \cite{meul-1991}.
    Meulenbeld's footnotes on this chapter in
    \volcite{IB}[325\,ff.]{meul-hist} refer often to ``Hoernle's note.'' 
    This appears to be a reference to Hoernle's copious notes to his
    translation of this chapter \citep[87--98]{hoer-1897}. \citet{meul-1990}
    also discussed Sanskrit veterinary texts in the light of their standard
    theory of four humours, including blood.}

\subsection{Translation}

\begin{translation}    
\item [1] Now we shall declare the chapter about blood.

\item [2]

    %\cite[33]{adri-1984} -> Author 1999: 33
    
\item [3] Food is of four types.\footnote{Ḍalhaṇa on \Su{1.14.3}{59} said
    that the four types of food are those that can be drunk, licked, eaten
    and chewed(\dev{peyalehyabhojyabhakṣya}). The main text of the \CS\ is
    explicit about these categories at \Ca{4.3.4(1)}{308}:
    \dev{pānāśanabhakṣyalehya/} “things drunk, eaten, chewed or licked.”
    \citet{yagi-1994} discussed the distinction between \dev{bhakṣya} and
    \dev{bhojya}; for further Indological background on foods, see the
    studies by \citet{oliv-1995,oliv-2001} and the classic reference works by
    \citet{acha-1994,acha-1998}. The long, final adhyāya of the
    \SS's sūtrasthāna (ch.\,46) is a treatise on food in āyurveda.} It is endowed 
    with six tastes and is made
    of the five elements.\footnote{\emph{Idem}, Earth, water, fire, air,
        space} It has either two or eight potencies, and is endowed with many
        qualities. \footnote{Ḍalhaṇa related these qualities to the twenty
            standard \dev{guṇa} of āyurveda; see, e.g., their listing by Vāgbhaṭa,
            translated by \citet[207]{wuja-2003}. }
            \se{rasa}{Chyle} is the most intangible essence of this food that is
            properly transformed. It is of the nature of fire.
                
Chyle is situated in the heart. From the heart, it enters into the
twenty-four arteries—ten upward arteries, ten downward, and four
sideways—and doing so day after day owing to the reaction of past
activities that are caused by the invisible,\footnote{\dev{adṛṣṭa}
    (unseen): Doing any righteous or unrighteous action produces good merit
    and demerit respectively. This good merit and demerit are called
    \dev{adṛṣṭa} (invisible) because it cannot be directly known but can only
    be assumed through logical deduction.} it satisfies the entire body,
    enlivens it, prolongs it,\footnote{In the sense of prolonging its
        lifespan} and makes it grow. The motion of the entity that flows
        throughout the body should be understood by inference. That motion causes
        deterioration and growth.
    
With regards to the chyle that flows through all the limbs, humours, body tissues, 
and impurities of the body, the question arises, “Is it moist or is it fiery?” It is 
understood to be moist because of its fluidity while flowing\footnote{The vulgate 
emends \dev{anusaraṇe} to \dev{anusaraṇa-} against the Nepalese MSS. This is 
logical because mobility would seem to be one of the attributes.  Although it is 
awkward, we read \dev{anusaraṇe} as a locative absolute ``while flowing.''} and 
due to attributes such as mobility, lubrication, enlivening, satisfaction, and 
supporting.\footnote{The duality being discussed here is that of the essential 
qualities of Fire and of Soma (\emph{agni} and \emph{soma}). See further 
discussion by \citet{wuja-2004} and \citet{ange-2021}.}
    
\item [4]  
That watery chyle is then reddened after reaching the liver and spleen.

  
\item [5]
There are verses about this.

\begin{sloka}
Experts know that blood is the untransformed fluid that is reddened by the pure fire element within the bodies of living beings. %[by them??].
% Transcription of vulage has the verse divided into two IDs. They have to be combined into one.
\end{sloka}

\item [6]

\begin{sloka}
It is only due to chyle that women's blood called menses exists. It increases from the twelfth year and decreases after the fiftieth year. 
\end{sloka}

\item [7]

The menstrual blood, however, is called fiery.\footnote{Ḍalhaṇa commented that this is to distinguish the menstrual blood from regular blood that is gentle.} That is due to the embryo being fiery and moist.\footnote{Ḍalhaṇa commented here that the embryo is called such 
because the menstrual blood is fiery and the semen is gentle (\dev{saumya}). On  the fiery/moist distinction (\dev{āgneya/saumya}), see \cite{wuja-2004,ange-2021}.}

%\footnote{\dev{agnīṣomīya} is a 
%particular Vedic sacrifice which is related to the deities of fire (\dev{agni}) and 
%moon (\dev{soma}). Ḍalhaṇa commented that the embryo is called such 
%because 
%the menstrual blood is fiery and the semen is gentle (\dev{saumya}). The word 
%\dev{saumya} is derived from the word \dev{soma}, where it means that 
%which 
%has the qualities of the moon, i.e. that which is gentle. }

\item [8]

Others state that the embryo as constituted of the five elements and the preceptors call it the living blood. 

\item [9]
There are verses about this.
\begin{sloka}
That is because blood exhibits the qualities of earth, etc. such as a fleshy smell, 
fluidity, redness, pulsation and thinness.
\end{sloka}

\item [10]

\begin{sloka}
Blood is formed from chyle, flesh from blood, lymph from flesh, bone from lymph, marrow from bone, semen from marrow, and progeny from semen. 
\end{sloka}

\item [11]

There, the essence (chyle) of food and drink is the nourisher of these body tissues.


\item[12]

There is a verse about this.

\begin{sloka}
A living being should be known as born from chyle. One should diligently preserve\footnote{All three manuscripts have \dev{rakṣeta} which is an incorrect form. \dev{rakṣet} is the correct form.} chyle by administering food and drink, being nicely disciplined with food\footnote{\dev{āhāreṇa} - The third case is used. The semantic property of the third case used here is unclear. Unclear regarding if there is any rule in the \emph{Aṣṭādhyāyī} justifying this usage.}.
\end{sloka}

\item[13]

The verbal root \emph{rasa} means movement.\footnote{\cite[109]{bhis-1907}} Because it keeps moving day after day, it is called \emph{rasa} (chyle).\footnote{In the list of verbal roots of Pāṇini, the verbal root \dev{rasa}(\emph{rasa}) means taste and moistening. It does not mean movement.}    

\item[14]

Chyle stays in every body tissue for 2548 ((25*100)+48) \emph{kalās} and nine \emph{kāṣṭhas}. As such, it becomes semen after a month. For women, it becomes menses.  

\item[15] 
Here are verses about this.

\begin{sloka}
According to similar and dissimilar treatises, the quantity of \emph{kalās} in 
this group\footnote{The duration of chyle in all the body tissues as a whole.} 
is 18,090.

%\item[15ef-gh]

This is the particular transformation period regarding chyle that lasts for a person with mild fire\footnote{Perhaps this refers to the digestive fire.}. For a person with developed fire, one should know it to last for the exact same time\footnote{Although the vulgate does not have this verse, there is an argument presented in Ḍalhaṇa's commentary on \Su{1.14.16}{63} that for a person with intense fire, chyle becomes semen after eight days, and for a person with mild fire, chyle becomes semen after a month. Ḍalhaṇa said that this opinion is refuted by Gayadāsa Ācārya in many different ways. Ḍalhaṇa continued that the proper understanding is that for a person with a strong fire, chyle becomes blood in a little less than a month, and for a person 
with a mild fire, chyle becomes blood in a little more than a month.}
%This particular transformation in chyle comes to an end due to the mild fire. Also, in the same duration of time, that\footnote{The end of transformation of chyle} is to be known due to the grown/big/manifest fire.

%OR

%This particular transformation in chyle forms the end due of the mild fire. Also, in the same duration of time, that of the intense fire is to be known.
\end{sloka}

\item[16]

Resembling the expanse of sound, flame, and water, that entity moves along in a minute manner throughout the  entire body\footnote{Ḍalhaṇa comments \citep[63]{vulgate} that the expanse of sound indicates the sideways movement of chyle, the expanse of flame indicates the upward movement of chyle, and the expanse of water indicates the downward movement of chyle.}.

\item[17]

The aphrodisiac medicines, however, being used like a purgative due to their excessively strong characteristics, evacuate the semen.      

\item[18]

Just as it cannot be said that the fragrance in a flower bud is present in it or not, but 
accepting that there is the manifestation of existing entities\footnote{This is the 
doctrine of pre-existence of the effect (\dev{satkāryavāda}, \textit{satkāryavāda}) 
first propounded by Sāṅkhya philosophers.}, it,\footnote{fragrance} however, is not 
experienced only due to its intangibility. That same entity is experienced at another 
time in the blossomed flower. In the same way regarding children also, the 
manifestation of semen happens because of the advancement of 
age\footnote{Since chyle becomes semen in a month's time, a question arises "Why 
then is semen absent in young children?". The reply is given in this passage.}. For 
women, the manifestation is different as rows of hair, menses, etc. 

\item[19]

That very essence of food does not nourish very old people due to their decaying bodies.

\item[20]

These entities are called body tissues (\emph{dhātu-s}) because they bear the body\footnote{The etymological meaning of the Sanskrit word \dev{dhātu} (\emph{dhātu}) is "that which bears [the body]". Thus, the body tissues are called \emph{dhātu-s} because they bear the body. This means that the body tissues are the elements that make up the body and sustain it.}. 

\item[21]

Their decay and growth are due to blood. Therefore, I will speak about blood. 
In that regard: The blood that is foamy, tawny, black, rough, thin, quick-moving, and non-coagulating is vitiated by air. The blood that is dark green, yellow, green, brown, sour-smelling, and unpleasant to ants and flies is vitiated by bile. The blood that is orange, unctuous, cool, dense, slimy, flowing, and resembling the colour of flesh-muscles is vitiated by phlegm. The blood having all these characteristics is vitiated by the combination of all three of them. The blood that is extremely black is vitiated by blood\footnote{\citet[64]{vulgate} quote Cakrapāṇidatta in a footnote: "This is the symptom when the blood vitiated in one part of the body vitiates the blood in another part."} just as bile. The blood that has the combined characteristics of vitiations of two humours is vitiated by two humours.

\item[22]

The blood that is of the colour of insect cochineal, not thick, and not discoloured should be understood to be in its natural state. 

\item[23]

I will speak of the types of blood that should be let out in another section. 

\item[24]

Now, I speak of those that should not be let out.
The swelling appearing in all the limbs of the body of a weak person that happens due to consuming sour food. The swellings of people with jaundice, piles, large abdomen, emaciation, and those of pregnant women.

\item[26]

In that regard, one should quickly insert the surgical instrument that is simple, not very close, fine, uniform, not deep, and not shallow. 

\item[26a] 
One should not insert the instrument into the heart, lower belly, anus, navel, waist, groins, eyes, forehead, palms, and soles.

\item[26b]

In the case of swellings filled with pus, one should treat them in the same way as stated earlier.

\item[27-27a]

There, when the swelling is not pierced properly, when phlegm and air have not been sweated out, after having a meal, and due to thickness, the blood does not ooze out or oozes out less.
Here is a verse regarding it.

\item[28ab-cd]

\begin{sloka}
Blood does not ooze out of humans when in contact with air, passing stool or urine, and when intoxicated, unconscious, fatigued, sleeping, or in cold surroundings.
\end{sloka}

\item[29] 

That vitiated blood when not taken out increases the disease.

\item[30]

The blood that is let by an ignorant physician in cases of very hot surroundings, profuse perspiration, and excessive piercing, flows excessively. That profuse bleeding causes the appearance of acute headache, blindness, and partial blindness, or it quickly causes subsequent wasting, convulsions, tremors, hemiplegia, paralysis in a limb, hiccups, coughing, panting, jaundice, or death.  

\item[31ab-cd]

The physician should let out the blood when the weather is not very hot or cold, when the patient is not perspiring or heated up, and after the patient has had a sufficient intake of gruel. 

\item[32ab-cd]

After coming out properly, when the blood stops automatically, one should know that blood to be pure and drained properly.

\item[33ab-cd]

The symptoms of the proper drainage of blood are the experience of lightness, alleviation of pain, a complete end of the intensity of the disease, and satisfaction of the mind.

\item[34ab-cd] 

Defects of the skin, tumours, swellings, and all diseases caused by blood never arise for those who regularly drain their blood.

\item[35]

	When the blood does not flow out, the physician should rub cardamom and camphor on the opening of the boil with three or four or all among crêpe ginger (Cheilocostus speciosus), butterfly gardenia (Ervatamia coronaria Stapf), \gls{pāṭhā}, \gls{bhadradāru}, \gls{viḍaṅga}, \gls{citraka}, the three spices (black pepper, long pepper, and dry ginger), \se{āgāradhūma}{soot from the chimney}, turmeric, sprouts of \gls{arka}, and fruit of the \gls{naktamāla}, according to availability, with excessive salt. By doing so, the blood flows out properly.

\item[36]

When there is an excessive flow of blood, the physician should sprinkle the 
opening of the boil with dry powders of \gls{lodhra}, liquorice, \gls{priyaṅgu}, 
\gls{pattāṅga}, red chalk, \gls{rasāñjana}, seashell, barley, \gls{māṣa}, 
wheat, and resin of the Sāla tree, and then press it with the tip of a finger. 
One should tightly bind it with powdered barks of Sāla, \gls{sarja}, \gls{arjuna}, \gls{arimeda}, \emph{granthi}, \gls{dhava}, and \emph{dhanvana} (Camelthorn), or a linen cloth\footnote{\cite[66]{vulgate} has \dev{kṣaumeṇa vā dhmāpitena} - "with linen reduced to ashes". Presumably, it is this ash that is also referred to in item 40.}, or \emph{vadhyāsita}, or bone of cuttlefish, or powdered lac, along with the binding materials mentioned. 
After the piercing, the physician should pierce it again. 
The physician should serve cool clothing, food, a dwelling place, a bath, cooling 
ointments, and plastering. Or, one can cauterize
%\footnote{Cauterization: The use of 
%heat to destroy tissues or close minute bleeding vessels.(Reference: 
%https://medical-dictionary.thefreedictionary.com/cauterization)} 
it with heat. Or, as 
mentioned, one should give a decoction of \emph{kākolī}, etc. sweetened by sugar and honey to drink. 
Or, one should consume the blood of black buck, deer, ram, buffalo, rabbit, or pig, accompanied by milk, green gram soup and meat soup\footnote{Based on Ḍalhaṇa's comment as found in \cite[66]{vulgate}}. 
The physician should treat the pains as mentioned. 

\item[36a]

Here are verses about this.

\item[37ab-cd]

\begin{sloka}
When blood flows out due to the decay of body tissue, fire becomes weak\footnote{This refers to the digestive fire.} and the wind becomes highly agitated because of that endeavour.
\end{sloka}

\item[38ab-cd]

\begin{sloka}
The physician should serve the patient food that is not very cold, light in digestion, unctuous, increases blood, slightly sour or not sour at all. 
\end{sloka}

\item[39ab-cd]

\begin{sloka}
This is the four-fold method of hindering blood: joining, coagulation, 
haemostasis.
%\footnote{Deliberate arrest of bleeding by local compression or 
%clamping of bleeding vessels...(Reference: 
%https://medical-dictionary.thefreedictionary.com/haemostasis)}, 
and cauterization.  
\end{sloka}

\item[40ab-cd]

\begin{sloka}
The astringent substance joins the opening, the cold substance coagulates the blood, the ash stops the blood, and cauterization contracts the blood vessel.
\end{sloka}

\item[41ab-cd]

\begin{sloka}
If the blood does not coagulate, the physician should employ joining. If the blood does not stop by joining the opening then he should employ haemostasis.
\end{sloka}

\item[42ab-cd]

\begin{sloka}
The physician should endeavour by employing these three methods according to the procedure. If these methods are unsuccessful then cauterization is highly desirable.
\end{sloka}

\item[43ab-cd]

\begin{sloka}
If the blood remains impure, the disease does not aggravate. The physician should 
then make the blood pure\footnote{Ḍalhaṇa comments \citep[66]{vulgate} that 
one should purify the blood again by sedation, etc.}\q{Can't be “sedation”} and not 
drain blood in excess.
\end{sloka}

\item[44ab-cd]

\begin{sloka}
Blood is the basis of the body. It is sustained by blood only.
\end{sloka}

\item[44ef]

\begin{sloka}
Blood is called life. One should therefore save blood.
\end{sloka}

\item[45ab-cd]

\begin{sloka}
If the air in the person who underwent blood-letting is aggravated due to a cold shower, etc., the swelling with pricking pain should be sprinkled with lukewarm clarified butter.   
\end{sloka}

\end{translation}

