% !TeX root = incremental_SS_Translation.tex
\chapter{Sūtrasthāna 28: Unfavourable Prognosis in Patients with Sores}

\section{Literature}

Meulenbeld offered an annotated overview of this chapter and a bibliography
of earlier scholarship to 2002.\fvolcite{IA}[219]{meul-hist} 

\citeauthor{gosw-2011} studied the commentaries of Ḍalhaṇa and 
Cakrapāṇidatta on this and the following adhyāyas up to 32, focussing on the 
topic of \se{ariṣṭa}{omens}.  He concluded that both 
authors were influenced by the Indriyasthāna of the \CS\ in their commentaries 
on this topic.\footcite{gosw-2011}

\section{Translation}
    
\begin{translation}    
    \item [1] Thus, living creatures and their strength,
\se{varṇa}{complexion} and \se{ojas}{energy} are rooted in food.  That
(food) depends on the six \se{rasa}{flavours}. Thus, the flavours depend
on \se{dravya}{substance}, and substances depend on medicinal herbs. 
There are two kinds of them (herbs):  stationary and 
mobile.\footnote{\Su{1.1.28}{7}, tr.\ \volcite{1}[21]{shar-1999}.}

\end{translation}