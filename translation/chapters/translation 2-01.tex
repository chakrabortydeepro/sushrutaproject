%!TeX root = incremental_SS_Translation.tex

\chapter{Nidānasthāna 1: The Diagnosis of Diseases Caused by Wind}

% Harshal Bhatt

\section{Literature}

Meulenbeld offered an annotated overview of this chapter and a bibliography of
earlier scholarship to 2002.\footnote{\volcite{IA}[234]{meul-hist}.
\citep{rube-1954b} studied the wind doctrines in the \CS.}

%Existing research on this chapter to 2002: \volcite{IB}[354--369]{meul-hist}.

\section{Subject matter}

It is notable that this nosological part of the \SS\ opens with a chapter on
diseases of \se{vāta}{wind}.  In all other major Āyurvedic works, including the
\CS, the first chapter in the section on nosology deals with the symptoms of
\se{jvara}{fever}.  This is almost a defining feature of works on nosology.  But
in the \SS, fever is not addressed at all in the first five sections of the
work, but only in the thirty-ninth chapter of the Uttaratantra, which is
exceptionally long at about three hundred verses.

The present chapter describes the diseases caused by vitiated wind and wind's
mixing with other humours. Contemporary Ayurvedic physicians consider these
diseases to include rheumatism. 

We have not translated the terms prāṇa \ldots because the text defines
them.\q{complete this thought}

\section{Translation}

\begin{translation}

\item [1] 

	And now we shall explain the chapter about the aetiology of wind
	diseases.



\item[3]

	After holding the feet of Dhanvantari, the foremost of the upholders of
	righteousness who emerged out of nectar, Suśruta makes this
	enquiry.\footnote{Explain the nectar myth.}\q{add footnote here}

\item[4]
 

	O King! O best of orators! Explain the location and types of diseases of
	the wind, whether in its natural state or disordered.\footnote{MSS H and
	N both read \dev{bhūpate} instead of \dev{kopanaiḥ} in the vulgate:
	instead of addressing the king, the vulgate is saying “by irritations of
	the wind\ldots.”  The vulgate also has Suśruta asking about \dev{karma},
	whereas in the Nepalese version he asks only about the types of
	diseases. Note that Dhanvantari is here addressed as king, a title
	associated elsewhere with Divosdāsa.}\q{add refs to Divodāsa as king. }
	%\footnote{https://doi.org/10.20935/AL2992}
	.

\item[5--9]

	On hearing his words, the venerable sage spoke.  This lordly wind is
	declared to be self-born because it is independent, constant and
	omnipresent. It is worshipped by the whole world. Amongst all beings, it
	is the self of all. During creation, continued existence and
	destruction, it is the cause of beings. 

	It is unmanifest though its actions are manifest; it is cold, dry,
	light, and mobile.  It moves horizontally, has two attributes and is
	full of \se{rajas}{dust}.\footnote{According to \Dalhana{2.1.8}{257},
	the two qualities are sound and tangibility.  The word \dev{rajas} could
	also refer to the quality of activity in the three-quality (\emph{guṇa})
	theory, which is how Ḍalhaṇa interpreted it.  On the semantic field of
	\dev{rajas}, see \cite[14 note 26 and ff.]{das-2003}.} %
	%\footnote{ it has power to divide humours, fluids, feces etc. moving
	%inside the body and it is the cause to the disease in the limbs. It
	%carries humours, chyle, semen/7 fluids?  and feces further in the body.
	%The wind which is moving outside is holding
	%the earth and body. (\dev{sā cāsya śaktiḥ
	%śarīradoṣamūtrapurīṣādivibhāgo'vayavasaṃsthānakā(ka)raṇaṃ
	%doṣadhātumalasaṃvahanādiśca, śarīrādbahistu saṃcarato dharaṇīdhāraṇādiḥ
	%}  Su 1938:257)}
    %
	It has inconceivable power. It is the leader of the
	humours\footnote{\Dalhana{2.1.8}{257} interpreted \dev{netā} “leader” as
	\dev{preraka} “impeller.”} and the ruler of the multitude of diseases.
    
	It moves fast, it moves constantly, it is located in the stomach and in
	the rectum.\footnote{MS H read \dev{āśucārī}, which we have translated
	(“moves fast”), but MS N and the commentators of the vulgate read
	\dev{āśukārī}, “quick-acting.”}

\item[9cd]

	Now, learn from me the characteristics of wind as it moves inside the
	body.\footnote{Ḍalhaṇa and Cakrapāṇidatta both interpreted \dev{me} as
	an ablative (\Su{2.1.8}{258}).}

\item[10] 

	Wind connects the senses and the sense objects.  Unvitiated, it
	maintains a state of equality between the \se{doṣa}{humours}, the
	\se{dhātu}{bodily tissues} and \se{agni}{heat} and the
	\se{ānulomya}{rightness} of actions.\footnote{According to
	\Dalhana{1.6.3}{23}, \dev{sampattiḥ=sampannatā}.  According to Ḍalhaṇa,
	Gayadāsa read \dev{indriyārthopasaṃprāptiṃ} but Ḍalhaṇa did not accept
	this on the grounds that it was too verbose: \dev{gayadāsācāryastu imaṃ
	ślokaṃ `indriyārthopasaṃprāpti' ityādi kṛtvā paṭhati, sa ca
	vistarabhayānna likhitaḥ /} But witnesses H and N suggest the reading
	\dev{indriyārthopasampattiḥ}.  
    
	The expression “qualities” is used advisedly. It is almost universal
	practice to refer to “balance” or “equilibrium” in such contexts, but
	this misrepresents the metaphor that the Sanskrit sources are using. As
	the commentators on \AH\ \Ah{1.1.20}{14} make abundantly clear, the
	expression \emph{doṣasāmya} means “equality of humours,” as in
	\emph{quantitative} equality, not balance.}

\item[11] 

	Just as the fire is divided into five types by name, place and their
	actions, similarly, one type of air is divided into five types based on
	name, place, action and diseases.

\item[12] 

Five types of wind:\footnote{See \cite{zysk-1993}.  \citet[S110]{zysk-2007}
translated the following descriptions of the winds.} \begin{enumerate} \item
		prāṇa\sse{prāṇa}{prāṇa}, \item udāna\sse{udāna}{udāna}, \item
			samāna\sse{samāna}{samāna}, \item
vyāna\sse{vyāna}{vyāna}, \item apāna\sse{apāna}{apāna}.\footnote{We use the
Sanskrit terms which are generally recognizable to English readers.}
\end{enumerate} The above five types of wind remain in their state of equality
and support the body.\footnote{According to \Dalhana{2.1.12}{259},
\dev{sthāna=sāmya, yāpayanti=dhārayanti}. All the manuscripts read
\dev{prāṇodānaḥ samānaśca vyānopānastathaiva ca/}  against the vulgate's
\dev{prāṇodānau samānaśca vyānaścāpāna eva ca/}.}



\item[13--14ab] 

	The wind that flows through the mouth is called the \se{prāṇa}{vital
	wind}, the sustainer of the body.  It causes food to enter within and
	supports the breaths.\sse{prāṇa}{breath}\footnote{According to
	\Dalhana{2.1.13--14ab}{259}, \dev{prāṇa} also resides in the throat and
	nose.} It mostly causes diseases like hiccups and \se{śvāsa}{wheezing}.



\item[14cd--15] 

	The wind which flows upwards, which is the best among winds, is called
	udāna.\footnote{According to \Dalhana{2.1.14cd--15}{260}, the places of
	udāna wind are not mentioned here, but it also flows in the navel,
	stomach and throat.  In yoga literature, it is more common for prāṇa to
	be called the principle breath.} Special acts like speech and singing
	are all initiated by it. It particularly causes diseases above the
	\se{jatru}{neck}.\footnote{Ḍalhaṇa noted that “above the \emph{jatru}”
	would include eyes, nose, ears, face, and head.  Meulenbeld cited
	discussions on the difficulties of interpreting the term \dev{jatru}
	\citep[465]{meul-1974}.  \citet[\S\S62, 98]{hoer-1907} translated
	\emph{jatru} as “neck, windpipe”. See also Hoernle's notes on the
	expression “above the \emph{jatru}” (idem, 237--238).}

\item [16--17ab]

	The samāna wind flows in the receptacles of raw and of digested
	matter.\footnote{The “receptacle of raw matter” (\dev{āmāśaya}) is
	described at \Su{1.21.12}{102} as one of the locations of phlegm, and
	 the place where food arrives, just above the location of bile, and
	 where the food is moistened and broken down for easy digestion.  The
	“receptacle of digested matter” (\dev{pakvāśaya}) is described at
	\Su{1.21.6}{100} as being located below the navel and above the pelvis
	and rectum.} Assisting the \se{agni}{digestive fire}, it cooks food and
	 separates out the substances produced from it.\footnote{Gayadāsa had
	 the same reading \dev{sahāyavān} as the Nepalese version \citep[260,
	 note~1 and the text of the \emph{Nyācacandrikā}]{vulgate}.  This
	 suggests that it is the samāna that cooks food, while the vulgate
	 reading involves the equal participation of digestive fire.}

	It mainly causes \se{gulma}{abdominal swelling},
	\se{agnisaṅga}{diminished digestive fire} and
	diarrhoea.\footnote{\Dalhana{1.11.8}{46} described \dev{agnisaṅga} as
	“the fire is stuck, dissolved.”}



\item[17cd--18] 

	The vyāna moves everywhere in the body, active in making
	\se{rasa}{chyle} flow.   It also makes sweat and blood flow as well as
	causing movement \diff{in every respect}.\footnote{The vulgate text
	reads \dev{pañcadhā} “in five ways,” and Ḍalhaṇa listed five kinds of
	 movement (\Dalhana{2.1.18}{260}).} Angered, it  causes diseases that
	generally exist throughout the whole body.
    

    
    \item [19--20ab]
    
	The apāna resides in the place of digested food and, at the right
	moment, it draws  wind, urine, and feces, as well as semen, fetus and
	 menstrual blood downwards.  Angered, it causes terrible diseases
	 located in the bladder and rectum.

\item[20cd--21ab]

	Irritated  vyāna and apāna winds cause defects of semen and
	\se{prameha}{urinary diseases}.  Simultaneously aggravated, they surely
	destroy the body.\footnote{\Dalhana{2.1.21ab}{261} clarified that this
	refers to all five winds being aggravated at once.}

	% The diseases caused by contaminated wind staying in different places
	% of the 
	%body are being described.

	\bigskip

\item[21cd--22ab] 

	From here, I shall describe all the diseases, located in the various
	places of the body, that are caused by wind that is irritated in various
	ways.

\item[22cd--24] 


	Aggravated wind in the stomach causes diseases like vomiting, as well as
	\se{moha}{disorientation}, fainting, thirst, heart-seizure, pain in the
	flanks.\footnote{On “disorientation,” \Dalhana{2.1.23ab}{261} noted that
	the condition was \dev{naivātyantaṃ cittanāśaḥ} “not the complete loss
	of awareness."} It also causes rumbling of the bowels,
	\se{śūla}{gripes}, swollen belly, painful urine and feces, constipation,
	and pain in the \se{trika}{sacrum}.\footnote{\citet[140]{hoer-1907}
	attributed the quite different interpretation of \dev{trika} by
	\Dalhana{1.21.14}{102} to “the  decay of anatomical knowledge subsequent
	to the time of Suśruta."} Aggravated wind in the ears etc., destroys the 	senses.




\item[25abc--29] 

Located in the skin, it causes pallor, throbbing, dryness,
\se{supti}{numbness}, \se{cumucumāyana}{itching}, and pricking
pain.\footnote{Ḍalhaṇa and Gayadāsa both suggested that in this
    passage, \dev{tvak} “skin” should be understood to mean \dev{rasa}
    “chyle” (on \Su{2.1.25}{262}). Gayadāsa explained in more detail that
    chyle is located in the skin and therefore, the expression
    \dev{tvakstha} “located in the skin” should, by extension, be read as
    \dev{rasastha} “located in the chyle.”  He proposed the parallel with
    the well-known grammatical example of figurative meaning,
    \dev{gaṅgāyāṃ ghoṣaḥ} “the village on the Ganges,” which means,
    really, “the village on the bank of the Ganges” (on this example of
    figurative meaning, \emph{lakṣaṇā}, see
    \cites[698--699]{jhal-1978}[ch.\,6]{kunj-1963}).} Located in the
    flesh, painful lumps.\footnote{At this point, the vulgate has a
        passage that is not present in the Nepalese witnesses.  It gives more
        symptoms of wind in the skin and then addresses wind in the blood:
        “(wind in the skin) may cause prickling, splitting of the skin and
        peeling; and when it is in the blood, it causes wounds”
        \citep[261]{vulgate}. \label{ancient-variants}The commentators
        Gayadāsa and Ḍalhaṇa were aware that this passage was missing in some
        of their manuscripts.  Gayadāsa said that this was because some
        authors noticed that \dev{vātarakta} “wind-blood” would be discussed
        later in the chapter. But they both thought this absence was incorrect
        \citep[262]{vulgate}.} Located in the fat, it causes slightly painful
        lumps that are not wounds. 
        
        % DW got to here

	Residing in the artery causes acute pain, contraction and filling up
	of the artery.\footnote{According to \Dalhana{2.1.27ab}{262} 	\dev{sirākuñcanaṃ} is also
	known as \dev{kuṭilā sirā} which is called varicose veins in the modern 	medical term } It stuns, vibrates and
	destroys the muscle tissues by residing in the muscle.
	
	Residing in the joints it destroys joints and causes pain and swelling. 	
	Residing in the bone
	it causes fracture and dryness of bones which also cause to acute pain
	and, in the marrow, it dries up marrow which may never be cured.
	
	Residing in the semen it causes non-production or distorted production
	of semen.\footnote{\Dalhana{2.1.29cd}{262} and Gayadāsa both suggest that a 	distorted production \dev{vikṛtāṃ pravṛttim} means a production of semen 	which is too fast, too slow, knotty and discolored.} 


\item[30--31ab]

	Similarly it moves in the hands, feet, head, and fluids or omnipresent
	it may be pervade the entire body of men and causes stiffness,
	convulsion, numbness and acute pain.

\item[31cd--32ab]

	Wind (5 types) mixed with other doṣas (bile etc.) in the places
	mentioned above produces mixed types of pains.

	% Symptoms of diseases combined with other humours (bile etc.) of prāṇa 
	%wind.
\item[34cd--35ab]

	Prāṇa wind surrounded by bile causes vomiting and burning sensation, by
	phlegm it causes weakness, exhaustion, laziness and bad taste. 

	% Symptoms of diseases combined with other humours (bile etc.) of udāna 
	%wind.
\item[35cd--36ab]

	Udāna wind surrounded by bile causes loss of consciousness, stupor,
	dizziness and fatigue, by phlegm it causes absence of perspiration,
	slowness of digestion, sensation of coldness.

	% Symptoms of diseases combined with other humours (bile etc.) of samāna 
	%wind.
\item[36cd--37ab]

	Samāna wind surrounded by bile causes perspiration, a burning sensation,
	heat and stupor, association with phlegm it causes erection in urine,
	feces and limbs.  

	% Symptoms of diseases combined with other humours (bile etc.) of apāna 
	%wind.
\item[37cd--38ab]

	Apāna wind associated with bile causes a burning sensation, heat and the
	voiding of blood with urine, with phlegm it causes a feeling of
	heaviness in the lower part of the body and coldness.

	% Symptoms of diseases combined with other humours (bile etc.) of vyāna 
	%wind.
\item[38cd--39ab]

	Vyāna wind surrounded by bile causes a burning sensation, tossing of the
	limbs and fatigue, by phlegm it causes stiffening limbs, uddaṇḍaka? and
	pain in the swelling.

\item[40--41]

	Persons who are of delicate nature, follow faulty diet and lifestyle, ?
	also afflicted with intoxicating drinks, 	sexual enjoyment,
	exercise causes vitiation of wind and blood.??

\item[42]

	Riding elephant, horse and camel, lifting great weights, consuming
	vegetables which are pungent, hot, sour, alkali and being frequently
	distressed situation causes contamination of wind. 

\item[43--44]

	Blood flowing in the body blocks the passage of contaminated wind which
	moves quickly in the body. Excessively irritated wind--being
	contaminated by wind and dominance of wind, it is called \dev{वातरक्त}
	Gout\footnote{In the medical term \dev{वातरक्त} is known as Gout.
	Cakrapāṇi called it \dev{आढ्यरोगः} Carakasaṃhitā sū.14.18 and ci.28.66}.

\item[45-46]

	Vātarakta causes -- pricking pain, dryness, loos of sensation in the
	feet. Contaminated Bile mixed with blood causes sharp burning sensation,
	excessive heat and soft swelling with red color in the feet.
	Contaminated Phlegm mixed with the blood causes itching in the feet. It
	makes feet white, cold, dry, thick and hard. All defects
	\footnote{Gayadāsa suggests \dev{सर्वे दुष्टाः शोणितं चापि} nominative plural
	instead of locative singular.} in the blood contaminated by humours
	(wind, bile, phlegm) manifest their symptoms in the feet.

\item[48]

	This disease spreads all over the body like rat poison by staying in
	feet or sometimes hands.

\item[49]

	Gout spreads in the knee and the skin bursts and starts bleeding makes
	it incurable. It is mitigatable if it is of a year’s old.

\item[50--51]

	When vitiated wind enters in the all arteries it causes quickly
	convulsions again and again and because of frequent
	\se{ākṣepa}{contractions} it is called \se{ākṣepaka}{convulsions}.

	%types of \dev{अपतानक} are being described.
\item[52--56]

	Because in this situation a person often sees darkness and fall, it
	calls \se{apatānaka}{spasmodic contraction} \footnote{Gayadāsa accepted
	the Nepalese reading \dev{ताम्यते} which vulgate does not read.  Gayadāsa
	gives definition of \dev{अपतानक} as \dev{येनापताम्यते} means a situation in
	that a person sees the dark.} . If wind mixed with phlegm stays
	excessively in the arteries, it stiffs body like a staff and it is
	called \dev{दण्डापतानकः} epilepsy with convulsions. Vitiated wind entered
	in the arteries and bends the body like a bow, it is called
	\dev{धनुःस्तम्भ} Tetanus. When vitiated wind accumulated in the regions of
	finger, ancle, abdomen, heart, chest, and throat swiftly attack on the
	group of vain and ligaments, it gets a person’s eyes stuck, chin stuns,
	side breaks and vomiting phlegm he moves inwards like a bow and this
	situation is known as \se{antarāyāma}{emprosthotonos}. When vitiated
	wind attacks on outside ligaments, body of a person will stretch forward
	like a bow. In this situation, if the chest, hip or thigh break, wise
	men call it incurable.

\item[58]

	Aggravated phlegm and bile mixed with wind or only vitiated wind causes
	fourth convulsive disease due to trauma.

\item[59]

	Convulsions due to miscarriage, excessive bleeding, and injury are
	incurable \footnote{According to Ḍalhaṇa \se{ākṣepaka}{convulsion} is
	also known as \dev{अपतानक} (Su 1938:266). He further mentions that even
	if fortunately, it is cured, it cripples the limb.}.

\item[60--62]

	When excessively agitated and strong wind flows in the arteries which
	spread downward, upward, and sideways, it loses the joints and kills the
	other side of body. The best of physicians calls it
	\se{pakṣāghāta}{paralysis}.  \footnote{In the ca.6.28.55 \dev{पक्षाघात}
	is described as \se{ekāṅgaroga}{monoplegia}. In that case it damages one
	of the limbs.  In the medical terms \se{apakṣāghāta}{paralysis} is known
	as hemiplegia.} Then half of his entire body becomes inefficient and
	unconscious. Afflicted by wind he suddenly falls or dies.

\item[62.1]

	Bile integrates with wind causes burning sensation, affliction, and
	infatuation. When it integrates with phlegm causes coldness, morbid
	swelling, and heaviness. \footnote{This verse is not available in
	vulgate. It deals with the symptoms when bile and phlegm mix with the
	wind. It is already discussed in su.2.1.38.}. 

\item[63]

	A \se{pakṣāghāta}{paralysis} caused by wind \footnote{Here the term
	\dev{शुद्धवात} suggests the meaning of the wind that is devoid of bile and
	phlegm.} is curable with most difficulty. It becomes curable when caused
	by bile and phlegm mix with the wind. It becomes incurable when caused
	by the loss of bodily constituents.

\item[64--66]

	Verses from 64--66 are not found in the Nepalese manuscripts.  These
	verses discuss the term \se{āpatantraka}{spasmodic contradiction} which
	is the same as \dev{अपतानक}. Ḍalhaṇa commented on ni.1.64-66 (Su
	1938:267) that because of having the similar condition in both
	situations, some scholars do not read the \dev{अपतन्त्रक}. In the verse
	ni.1.59 Ḍalhaṇa commented that the \dev{आक्षेपक} and \dev{अपतानक} is same
	(Su 1938:266) and again he suggested that the \dev{अपतानक} and
	\dev{अपतन्त्रक} both are similar condition. Therefore, \dev{आक्षेपक},
	\dev{अपतानक} and \dev{अपतन्त्रक} should be the same. Gayadāsa further
	commented that the Caraka has not read \dev{आक्षेपक} as \dev{अपतानक} and
	therefore described the \dev{अपतन्त्रक} separately (Su 1938:267).

\item[67]

	This verse also not found in the Nepalese Manuscripts. The verse
	describes \se{manyāsthambha}{rigidity of neck}. According to Ḍalhaṇa,
	rigidity of neck is a prior symptom of spasmodic contradiction. 

	% \se{अर्दित}{spasm of the jaw-bones}
\item[68--72]

	By speaking very loudly, eating hard foods, excessively laughing and
	yawning, lifting heavy loads and sleeping in an awkward position,
	vitiated wind lodges into face painfully and produces
	\se{ardita}{paralysis of the jaw-bones} disease. In that case, half of
	the face and neck become curved, head trembles, speech hindrances,
	deformity occurs in the eys, eyebrows and cheeks.\footnote{Ḍalhaṇa
	suggests \dev{नेत्रादीनाम् इत्यादि शब्दात् भूगण्डादि उपसङ्ग्रहः}} Experts in
	diseases call this disease \se{ardita}{spasm of the jaw-bones}. 

\item[73]

	Spasm of the jawbones cannot be cured when it stays in a person for
	three years, who is very weak, stays without blinking, trembles, and
	constantly speaks gibberish.

\item[74]

	Arteries of Heel and toes stricken by vitiated wind prevents stretching
	of thighs. This disease is known as \se{gṛdhrasī}{sciatica}.

\item[75]

	Arteries which run to the tips of fingers from behind the roots of the
	upper arm affected by vitiated wind terminates all activities of arms
	and back.  This disease is called \se{viśvañci}{paralysis of arms and
	back}. \footnote{Both the MSS N and H read \dev{विश्वञ्चि} instead of the
	vulgate reading \dev{विश्वाची}. There is no such word found in other
	Āyurveda texts.}

\item[76]

	Vitiated wind and blood in the joint of knee causes
	\se{kroṣṭukaśīrṣa}{synovitis of knee join}. In this extremely painful
	situation, the shape of swelling in knee joints seems like a head of
	Jackal. 

\item[77]

	Vitiated wind resides in the waist attacks on the arteries of thigh
	causes \se{khañja}{limpness} and when it attacks on both the thighs a
	person becomes \se{paṅgu}{lame}.

\item[78]

	A person who trembles at the beginning of walking or walks limping and
	whose foot joint has become loose is called
	\se{kalāyakhañja}{lathyrism}.

\item[79]

	Vitiated wind residing in the ankle-joint causes pain when one steps on
	uneven ground. This disease occurs is called \dev{वातकण्टक}.

\item[80]

	Vitiated wind mixed with bile and blood cause burning sensation in feet.
	It should be declared as \se{pādadāha}{burning sensation in feet}.

\item[81]

	A person whose feet tingle and become insensible due to vitiation of
	phlegm and wind is called \dev{पादहर्ष}.

\item[82]

	Vitiated wind lying in the shoulder dries the shoulder joints and it is
	called \dev{अंसशोष}. It also bends the arteries of shoulder, and this
	disease is called \dev{अवबाहुक}. \footnote{Ḍalhaṇa and Gayadāsa both have
	defined two diseases i.e., \dev{अंसशोष} and \dev{अवबाहुक} respectively.}

\item[83]

	Vitiated wind singly or mixed with phlegm cover the channel of ears
	causes deafness.

\item[84]

	Vitiated wind saturated with phlegm covering the arteries which conduct
	the sound of speech makes a person \se{akriya}{inactive},
	\se{mūka}{dumb}. He \se{mimmira}{mumbles} through the nose and
	\se{gadgad}{stammers}.\footnote{Nepalese Manuscripts read \dev{मिर्म्मिर}
	instead of the Vulgate’s reading \dev{मिन्मिण}. Dictionary of MW suggests
	the meaning of \dev{मिर्म्मिर} = having fixed unwinking eyes which is not
	relevant to the disease of tongue.}

\item[85]

	Vitiated wind penetrating into the cheekbones, temporal bones, head and
	neck causes piercing pain in the ears. It is called
	\se{karṇaśūla}{ear-ache}.\footnote{In the medical terms, this disease is
	known as Otitis.}

\item[86--87]

	The pain that arises from the bladder or feces goes down as if it were
	breaking the rectum and…… ? is called \dev{तूनी}, whereas the pain,
	rising upward from the rectum extending up to the region of the
	intestines, is called \dev{प्रतितूनी}.

\item[88--89]

	Retention of vitiated wind inside abdomen causes distension of the
	stomach and flatulence and intense pain and rumbling inside, is called
	\se{ādhmāna}{tympanites}. Vitiated wind mixed with phlegm causes
	\dev{प्रत्याध्मान}. It rises in the stomach anda causes pain in the heart
	and sides. \footnote{There’s an addition in MS N. \dev{नाभेरधस्तात् संजातः
	संचारी यदि वाऽचलः}}

\item[90--91]

	A knotty stone-like tumour caused by wind appearing in the stomach
	having an elevated shape and stretched upward direction which
	obstructing the passage of faeces and urine should be known as
	\dev{वाताष्ठीला}. A tumour of similar shape rose obliquely in the abdomen
	obstructing the passage of wind, faeces and urine should be known as
	\dev{प्रत्यष्ठीला}. 


	Names of diseases discussed in the chapter 2.1

	\se{vātarakta}{Gout} \se{ākṣepaka}{convulsion} \se{pakṣāghāta}{paralysis
	of one side} \se{ardita}{paralysis of the jaw-bones}
	\se{gṛdhrasī}{sciatica} \se{viśvañci}{paralysis of arms and back}
	\se{kroṣṭukaśīrṣa}{synovitis of knee join} \se{kalāyakhañja}{lathyrism}
	\se{vātakaṇṭaka}{} \se{avabāhuka}{} \se{tūnī}{} \se{pratitūnī}{}
	\se{ādhmāna}{tympanites} \se{pratyādhmāna}{} \se{vātāṣṭhīlā}{}
	\se{pratyaṣṭhīla}{}




\end{translation}
