%!TeX root = incremental_SS_Translation.tex

\chapter{Nidānasthāna 1: The Diagnosis of Diseases Caused by Wind}

% Harshal Bhatt

\section{Literature}

Meulenbeld offered an annotated overview of this chapter and a bibliography of
earlier scholarship to 2002.\footnote{\volcite{IA}[234]{meul-hist}.
\citep{rube-1954b} studied the wind doctrines in the \CS.}

%Existing research on this chapter to 2002: \volcite{IB}[354--369]{meul-hist}.

\section{Subject matter}

It is notable that this nosological part of the \SS\ opens with a chapter on
diseases of \se{vāta}{wind}.  In all other major Āyurvedic works, including the
\CS, the first chapter in the section on nosology deals with the symptoms of
\se{jvara}{fever}.  This is almost a defining feature of works on nosology.  But
in the \SS, fever is not addressed at all in the first five sections of the
work, but only in the thirty-ninth chapter of the Uttaratantra, which is
exceptionally long at about three hundred verses.

The present chapter describes the diseases caused by vitiated wind and wind's
mixing with other humours. Contemporary Ayurvedic physicians consider these
diseases to include rheumatism. 

We have not translated the terms prāṇa \ldots because the text defines
them.\q{complete this thought}

\section{Translation}

\begin{translation}

\item [1] 

	And now we shall explain the chapter about the aetiology of wind
	diseases.



\item[3]

	After holding the feet of Dhanvantari, the foremost of the upholders of
	righteousness who emerged out of nectar, Suśruta makes this
	enquiry.\footnote{Explain the nectar myth.}\q{add footnote here}

\item[4]
 

	O King! O best of orators! Explain the location and types of diseases of
	the wind, whether in its natural state or disordered.\footnote{MSS H and
	N both read \dev{bhūpate} instead of \dev{kopanaiḥ} in the vulgate:
	instead of addressing the king, the vulgate is saying “by irritations of
	the wind\ldots.”  The vulgate also has Suśruta asking about \dev{karma},
	whereas in the Nepalese version he asks only about the types of
	diseases. Note that Dhanvantari is here addressed as king, a title
	associated elsewhere with Divosdāsa.}\q{add refs to Divodāsa as king. }
	%\footnote{https://doi.org/10.20935/AL2992}
	.

\item[5--9]

	On hearing his words, the venerable sage spoke.  This lordly wind is
	declared to be self-born because it is independent, constant and
	omnipresent. It is worshipped by the whole world. Amongst all beings, it
	is the self of all. During creation, continued existence and
	destruction, it is the cause of beings. 

	It is unmanifest though its actions are manifest; it is cold, dry,
	light, and mobile.  It moves horizontally, has two attributes and is
	full of \se{rajas}{dust}.\footnote{According to \Dalhana{2.1.8}{257},
	the two qualities are sound and tangibility.  The word \dev{rajas} could
	also refer to the quality of activity in the three-quality (\emph{guṇa})
	theory, which is how Ḍalhaṇa interpreted it.  On the semantic field of
	\dev{rajas}, see \cite[14 note 26 and ff.]{das-2003}.} %
	%\footnote{ it has power to divide humours, fluids, feces etc. moving
	%inside the body and it is the cause to the disease in the limbs. It
	%carries humours, chyle, semen/7 fluids?  and feces further in the body.
	%The wind which is moving outside is holding
	%the earth and body. (\dev{sā cāsya śaktiḥ
	%śarīradoṣamūtrapurīṣādivibhāgo'vayavasaṃsthānakā(ka)raṇaṃ
	%doṣadhātumalasaṃvahanādiśca, śarīrādbahistu saṃcarato dharaṇīdhāraṇādiḥ
	%}  Su 1938:257)}
    %
	It has inconceivable power. It is the leader of the
	humours\footnote{\Dalhana{2.1.8}{257} interpreted \dev{netā} “leader” as
	\dev{preraka} “impeller.”} and the ruler of the multitude of diseases.
    
	It moves fast, it moves constantly, it is located in the stomach and in
	the rectum.\footnote{MS H read \dev{āśucārī}, which we have translated
	(“moves fast”), but MS N and the commentators of the vulgate read
	\dev{āśukārī}, “quick-acting.”}

\item[9cd]

	Now, learn from me the characteristics of wind as it moves inside the
	body.\footnote{Ḍalhaṇa and Cakrapāṇidatta both interpreted \dev{me} as
	an ablative (\Su{2.1.8}{258}).}

\item[10] 

	Wind connects the senses and the sense objects.  Unvitiated, it
	maintains a state of equality between the \se{doṣa}{humours}, the
	\se{dhātu}{bodily tissues} and \se{agni}{heat} and the
	\se{ānulomya}{rightness} of actions.\footnote{According to
	\Dalhana{1.6.3}{23}, \dev{sampattiḥ=sampannatā}.  According to Ḍalhaṇa,
	Gayadāsa read \dev{indriyārthopasaṃprāptiṃ} but Ḍalhaṇa did not accept
	this on the grounds that it was too verbose: \dev{gayadāsācāryastu imaṃ
	ślokaṃ `indriyārthopasaṃprāpti' ityādi kṛtvā paṭhati, sa ca
	vistarabhayānna likhitaḥ /} But witnesses H and N suggest the reading
	\dev{indriyārthopasampattiḥ}.  
    
	The expression “qualities” is used advisedly. It is almost universal
	practice to refer to “balance” or “equilibrium” in such contexts, but
	this misrepresents the metaphor that the Sanskrit sources are using. As
	the commentators on \AH\ \Ah{1.1.20}{14} make abundantly clear, the
	expression \emph{doṣasāmya} means “equality of humours,” as in
	\emph{quantitative} equality, not balance.}

\item[11] 

	Just as the fire is divided into five types by name, place and their
	actions, similarly, one type of air is divided into five types based on
	name, place, action and diseases.

\item[12] 

Five types of wind:\footnote{See \cite{zysk-1993}.  \citet[S110]{zysk-2007}
translated the following descriptions of the winds.} \begin{enumerate} \item
		prāṇa\sse{prāṇa}{prāṇa}, \item udāna\sse{udāna}{udāna}, \item
			samāna\sse{samāna}{samāna}, \item
vyāna\sse{vyāna}{vyāna}, \item apāna\sse{apāna}{apāna}.\footnote{We use the
Sanskrit terms which are generally recognizable to English readers.}
\end{enumerate} The above five types of wind remain in their state of equality
and support the body.\footnote{According to \Dalhana{2.1.12}{259},
\dev{sthāna=sāmya, yāpayanti=dhārayanti}. All the manuscripts read
\dev{prāṇodānaḥ samānaśca vyānopānastathaiva ca/}  against the vulgate's
\dev{prāṇodānau samānaśca vyānaścāpāna eva ca/}.}



\item[13--14ab] 

	The wind that flows through the mouth is called the \se{prāṇa}{vital
	wind}, the sustainer of the body.  It causes food to enter within and
	supports the breaths.\sse{prāṇa}{breath}\footnote{According to
	\Dalhana{2.1.13--14ab}{259}, \dev{prāṇa} also resides in the throat and
	nose.} It mostly causes diseases like hiccups and \se{śvāsa}{wheezing}.



\item[14cd--15] 

Since it is the one that flows upwards, that highest of winds is
called udāna.\footnote{The sentence plays on the sound
    \dev{ut-}\dev{/ūrdh-} in the qualifiers (\dev{udāna, ūrdhvam,
    uttama}).  According to \Dalhana{2.1.14cd--15}{260}, the places of
    udāna wind are not mentioned here, but it also flows in the navel,
    stomach and throat. In yoga literature, it is more common for prāṇa to
    be called the principle breath.} Special acts like speech and singing
    are all initiated by it. It particularly causes diseases above the
    \se{jatru}{neck}.\footnote{Ḍalhaṇa noted that “above the \emph{jatru}”
        would include eyes, nose, ears, face, and head. Meulenbeld cited
        discussions on the difficulties of interpreting the term \dev{jatru}
        \citep[465]{meul-1974}.  \citet[\S\S62, 98]{hoer-1907} translated
        \emph{jatru} as “neck, windpipe”. See also Hoernle's notes on the
        expression “above the \emph{jatru}” (idem, 237--238).}

\item [16--17ab]

	The samāna wind flows in the receptacles of raw and of digested
	matter.\footnote{The “receptacle of raw matter” (\dev{āmāśaya}) is
	described at \Su{1.21.12}{102} as one of the locations of phlegm, and
	 the place where food arrives, just above the location of bile, and
	 where the food is moistened and broken down for easy digestion.  The
	“receptacle of digested matter” (\dev{pakvāśaya}) is described at
	\Su{1.21.6}{100} as being located below the navel and above the pelvis
	and rectum.} Assisting the \se{agni}{digestive fire}, it cooks food and
	 separates out the substances produced from it.\footnote{Gayadāsa had
	 the same reading \dev{sahāyavān} as the Nepalese version \citep[260,
	 note~1 and the text of the \emph{Nyācacandrikā}]{vulgate}.  This
	 suggests that it is the samāna that cooks food, while the vulgate
	 reading involves the equal participation of digestive fire.}

	It mainly causes \se{gulma}{abdominal swelling},
	\se{agnisaṅga}{diminished digestive fire} and
	diarrhoea.\footnote{\Dalhana{1.11.8}{46} described \dev{agnisaṅga} as
	“the fire is stuck, dissolved.”}



\item[17cd--18] 

	The vyāna moves everywhere in the body, active in making
	\se{rasa}{chyle} flow.   It also makes sweat and blood flow as well as
	causing movement \diff{in every respect}.\footnote{The vulgate text
	reads \dev{pañcadhā} “in five ways,” and Ḍalhaṇa listed five kinds of
	 movement (\Dalhana{2.1.18}{260}).} Angered, it  causes diseases that
	generally exist throughout the whole body.
    

    
    \item [19--20ab]
    
	The apāna resides in the place of digested food and, at the right
	moment, it draws  wind, urine, and feces, as well as semen, fetus and
	 menstrual blood downwards.  Angered, it causes terrible diseases
	 located in the bladder and rectum.

\item[20cd--21ab]

	Irritated  vyāna and apāna winds cause defects of semen and
	\se{prameha}{urinary diseases}.  Simultaneously aggravated, they surely
	destroy the body.\footnote{\Dalhana{2.1.21ab}{261} clarified that this
	refers to all five winds being aggravated at once.}

	% The diseases caused by contaminated wind staying in different places
	% of the 
	%body are being described.

	\bigskip

\item[21cd--22ab] 

	From here, I shall describe all the diseases, located in the various
	places of the body, that are caused by wind that is irritated in various
	ways.

\item[22cd--24] 

	Aggravated wind in the stomach causes diseases like vomiting, as well
as \se{moha}{disorientation}, fainting, thirst,
\se{hṛdgraha}{heart-seizure}, and pain in the flanks.\footnote{On
    “disorientation,” \Dalhana{2.1.23ab}{261} noted that the condition
    was \dev{naivātyantaṃ cittanāśaḥ} “not the complete loss of
    awareness."} It also causes rumbling of the bowels,
    \se{śūla}{gripes}, swollen belly, painful urine and feces,
    constipation, and pain in the
    \se{trika}{sacrum}.\footnote{\citet[140]{hoer-1907} attributed the
        quite different interpretation of \dev{trika} by
        \Dalhana{1.21.14}{102} to “the  decay of anatomical knowledge
        subsequent to the time of Suśruta."}
	Aggravated wind in the ears etc., destroys the senses.



\item[25abc--29] 

Located in the skin, it causes \se{vaivarṇya}{discolouration},
throbbing, dryness, \se{supti}{numbness}, \se{cumucumāyana}{itching},
and pricking pain.\footnote{\citet{maas-2008} definitively clarified
    the contrasting \dev{tvak}-first and (usually) \dev{rasa}-first models
    of the bodily elements (\emph{dhātu})\sse{dhātu}{bodily element} as
    distinct historical formulations in the earliest medical literature.
    \cite[267--282]{das-2003} also explored this issue, including the
    obeservation that the \emph{Bheḍasaṃhitā} seems to have taught that
    \dev{rasa} “chyle”  was the sources of menstrual blood, in contrast to
    the \emph{Kāśyapasaṃhitā} that assigned this role to \dev{tvak}
    “skin." In their comments on this passage, Gayadāsa and Ḍalhaṇa both
    tried to square the circle of these contrasting models by suggesting
    that \dev{tvak} “skin” should be understood to mean \dev{rasa} “chyle”
    (on \Su{2.1.25}{262}). Gayadāsa explained in more detail that chyle is
    located in the skin and therefore, the expression \dev{tvakstha}
    “located in the skin” should, by extension, be read as \dev{rasastha}
    “located in the chyle.”  He proposed the parallel with the well-known
    grammatical example of figurative meaning, \dev{gaṅgāyāṃ ghoṣaḥ} “the
    village on the Ganges,” which means, really, “the village on the bank
    of the Ganges” (on this example of figurative meaning, \emph{lakṣaṇā},
    see \cites[698--699]{jhal-1978}[ch.\,6]{kunj-1963}).  } Located in the
    flesh, painful lumps.\footnote{At this point, the vulgate has a
        passage that is not present in the Nepalese witnesses.  It gives more
        symptoms of wind in the skin and then addresses wind in the blood:
        “(wind in the skin) may cause prickling, splitting of the skin and
        peeling; and when it is in the blood, it causes wounds”
        \citep[261]{vulgate}. \label{ancient-variants}The commentators
        Gayadāsa and Ḍalhaṇa were aware that this passage was missing in some
        of their manuscripts. Gayadāsa said that this was because some authors
        noticed that \dev{vātarakta} “wind-blood” would be discussed later in
        the chapter. But they both thought this absence was incorrect
        \citep[262]{vulgate}.} Located in the fat, it causes slightly painful
        lumps that are not wounds.
        
Located in the ducts, it causes acute pain, contraction and filling up
of the duct.\footnote{According to Ḍalhaṇa \dev{sirākuñcanaṃ} is also
    known as \dev{kuṭilā sirā} \citep[262]{vulgate}, which may refer to
    varicose veins.} %
    When it reaches the sinews, it paralyses the network of sinews,
    and causes them to tremble.\label{ākṣepaka} %
    Located in the joints, it destroys the joints and it causes sharp
    pain and swelling. %
    It causes a splitting of the bones, when it acts there, and
    dryness as well as sharp pain; %
    and when it is in the marrow, it causes an sickness that never
    abates. %
    Wind located  in the semen, it  causes the non-production or faulty
    production of semen.\footnote{Ḍalhaṇa and Gayadāsa both suggest that
        a faulty production \dev{vikṛtāṃ pravṛttim} is too fast, too
        slow, knotty and discoloured \cite[262]{vulgate}.}


\item[30--31ab]

Wind moves incrementally from the hand to the foot, the head, and the
bodily tissues.  Or it may pervade people's entire bodies, causing
stiffness, convulsion, \se{svāpa}{numbness}, swelling, and acute pain
everywhere.



\subsection{Symptoms of diseases that arise because of a 
combination of the five breaths with bile and phlegm}

\item[31cd--32ab]

In the stated locations, wind that is compounded causes compounded
afflictions.\footnote{\Dalhana{2.1.31cd}{262} explained “wind that is
    compounded” as wind being mixed with bile and phlegm.}  And located in
    the limbs, it can cause a multitude of diseases.\footnote{The Nepalese
        version omits passages 2.1.32cd--33ab which are about the diseases that
        arise when 	contaminated wind mixes with cough, phlegm and blood
        \citep[263]{vulgate}.}

	% Symptoms of diseases that arises when prāṇa mixes with other humours (bile etc.)

\subsubsection{Prāṇa}

\item[34cd--35ab]

Prāṇa covered by bile causes vomiting and a burning sensation and when
covered by phlegm it causes weakness, exhaustion, lassitude and
\diff{loss of the sense of taste}.\footnote{\dev{vairasya} “loss of
    the sense of taste” may refer to ageusia.  The vulgate reads
    \dev{vaivarṇya} “loss of colour” \citep[263]{vulgate}.  The vulgate's
    footnote 1 says that the palm-leaf manuscript reads  \dev{vaiśvarya}
    but this is not correct.  The palm-leaf manuscript whose readings were
    sent to Trivikrama Ācārya was witness N, which reads \dev{vairasya}.}

	% Symptoms of diseases that arises when udāna mixes with other humours (bile etc.)

\subsubsection{Udāna}

\item[35cd--36ab]

	When udāna is joined with bile there is \se{moha}{bewilderment},
\se{mūrchā}{fainting}, \se{bhrama}{dizziness} and exhaustion. And
when covered by phlegm there is exhilaration and an absence of
perspiration, slow digestion, and coldness.\footnote{The expression
    “exhilaration and an absence of perspiration” translates the Nepalese
    version's \dev{asvedaharṣaḥ} as if it were a dvandva.  The vulgate
    has the easier dvandva, \dev{asvedaharṣau} “lack of sweating and also
    exhilaration” \Su{2.1.36ab}{263}.  Perhaps the Nepalese reading is an
    Epic form of m.\ sing.\ dvandva as described by \citet[361--362,
    n.\,3]{ober-2003}.\label{masc-dvandva}}

	% Symptoms of diseases that arises when samāna mixes with other humours 
	%(bile etc.)
    
\subsubsection{Samāna}
    
\item[36cd--37ab]

	When is samāna is combined with bile there is perspiration, a burning sensation,
	a temperature and \se{mūrchā}{fainting}. When in contact with phlegm 
    there is horripilation of the limbs during feces and urine.

	% Symptoms of diseases that arises when apāna mixes with other humours (bile etc.)

\subsubsection{Apāna}

\item[37cd--38ab]

	When apāna is associated with bile  there is a burning sensation, a temperature 
	and blood in the urine.\footnote{The This probably describes hematuria.  Again 
	we have an Epic m.\ sing.\ dvandva.}
     When  covered with phlegm there is a feeling of
	heaviness in the lower body and coldness.

	% Symptoms of diseases that arises when vyāna mixes with other humours (bile etc.)



\subsubsection{Vyāna}

\item[38cd--39.1]

When vyāna is covered by bile there is a \se{dāha}{burning sensation},
shaking of the limbs and fatigue.\footnote{The next vulgate verse is
    absent in the Nepalese version.  It describes diseases caused by
    contaminated vyāna mixed with cough and phlegm \citep[264]{vulgate}.
    Instead of this verse, Nepalese version has the following sentence
    about phlegm.} When covered by phlegm there is paralysis,
    \se{uddaṇḍaka}{stiffening}, and swelling with pain.\footnote{The word
        \dev{uddaṇḍaka} “being like a vertical stick” is rare or unknown as a
        medical term (unrelatedly, it is the name of an ascetic group listed
        in works such as the \emph{Cāturāśramyadharma} of Kāṇvāyana 
        \pvolcite{3}[306]{ncc}). Some of
        these symptoms are in common with Stiff Person Syndrome.}
 

	% Symptoms of vātarakta that arises when wind mixes with blood.
\item[40--41]

In general, wind-blood\sse{vātarakta}{wind-blood} causes inflammation in
those who are delicate and enjoy inappropriate food, and because of
the torment of the \diff{humours},\footnote{“Wind-blood” is described in the
    \SS\  as the combination of corrupted blood obstructing the path of
    inflamed wind and causing simultaneous pain due to wind and blood at
    once (\Su{4.5.4}{423}).  The \CS\ describes it as increased wind being
    blocked in its passage by increased blood (\Ca{6.29}{627--634}).  See
    also references at \volcite{1}[740--741]{josi-maha}.  Interpreted as
    leprosy by \tvolcite{1}[256--260]{seng-1901}.  Several symptoms described 
    below are similar to those of diabetic neuropathy.} the
    roads,\sse{adhva}{roads} intoxication from wine, and lack of
    exercise,\footnote{Probably, the “torment of the roads” refers
        metonymically to excess travel. “Lack of exercise” could be read as
        just “exercise,” and while that may sound like torment, the former
        interpretation better fits the context.  Note that the sequence
        \dev{-pramadāmadya-} in the vulgate separates “confusion” and “wine”
        while the Nepalese version's “wine-confusion” is a more obvious
        reading.  Ḍalhaṇa read \dev{mithyāhāravihārin} as a dual
        “inappropriate food and recreation” (\Dalhana{2.1.40 \& 4.5.5}{263 \&
        423}).} from the inversion of the seasons and locales, from the
        consumption of \se{asātmya}{uncongenial} foods, and because of the
        \diff{lack of exercise} taken by an overweight
        person.\footnote{Instead of “lack of exercise” the vulgate reads “lack of 
        sexual intercourse,” which makes little sense.
            
    \Dalhana{2.1.40--41}{263} commented that some scholars did not read
    these two verses here because these are read later, at Cikitsāsthāna
    \Su{4.5.5}{424}.  In fact, at that location, only 2.1.40ab and
    2.1.41cd are read.
                
    The word \dev{doṣa} appears in the Nepalese version of 2.1.40cd,
but not in the vulgate (which reads \dev{roga}). Therefore, when
Gayadāsa said \dev{doṣagrahaṇaṃ tu viśeṣārthamiti} “the use of the
word \dev{doṣa} is for the purpose of specificity,” at the end of
his comment on \Su{2.1.32--39}{263}, it is likely that
he had the Nepalese version of at least part of the text before
him, \emph{pace} the comment, “Gayadāsa did not accept this
reading” by \citet[\dev{gayadāsāsaṃmato'yaṃ pāṭhaḥ}][263, note
2]{vulgate}.}


\subsubsection{Wind-blood (\emph{vātarakta})}

\item[42--44]

The wind may become aggravated by riding elephants, horses, camels and
for other reasons.\footnote{Ḍalhaṇa exemplified “other reasons,” as
    carrying loads, etc.}

By consuming vegetables that are pungent, hot, sour, or alkali and by
strong, habitual \se{santāpa}{anguish}, the blood rapidly becomes
liquid and that quickly blocks the pathway of the quick-moving
wind;\footnote{The word \dev{santāpa}, “anguish” can mean physical as
    well as emotional pain.}  irritated by the obstruction of the pathway, it goes 
    wrong. 
%
That blood, mixed with corrupted wind is called 
“wind-blood”\sse{vātarakta}{wind-blood} because of  the wind's force. 

Similarly, bile may be tarnished by corrupted blood.\footnote{The Nepalese version 
omits the vulgate's similar statement about phlegm being affected by blood.} 



%	Gout\footnote{In the medical term \dev{वातरक्त} is known as Gout.
%	Cakrapāṇi called it \dev{आढ्यरोगः} Carakasaṃhitā sū.14.18 and ci.28.66}.
	
	% Caraka gave other names of \sse{vātarakta}{wind-blood} such as \dev{vātaśoṇita} \dev{khuḍa} 	\dev{vātabalāsa} \Ca{6.29.11} and has described the treatments of wind-blood in the separate 	chapter which is \Ca{6.29}.
%	and \dev{āḍhyavāta}


\item[45--46]

	Because of wind-blood, the feet have an
    aversion to touch, as well as 
    pricking, splitting, dryness, and a loss of sensation. 
    %
     Contaminated bile mixed with blood causes a sharp burning sensation,
	excessive heat, a red swelling and a softening of the feet.
    
        
When blood is contaminated by phlegm, the feet get itchy, cold and white, 
	swollen, thick and stiff.   
    	Furthermore, when blood is contaminated by all of them, the humours 
        display their respective signs in the feet.

\item[48]

	Residing in the soles of the feet, and sometimes in the hands, this disease 
	creeps through that body like angry rat poison.\footnote{The 
	commentators 
	Gayadāsa and Ḍalhaṇa both read “the whole body” (\dev{saddeham} for 
	\dev{taddeham}, interpreting \dev{sad} as \dev{sakalam}
	(\cite[264]{vulgate})).  The subject, “this disease,” is not expressed in the 
	Sanskrit sentence.}


\item[49ab, 50ab]

\se{vātarakta}{Wind-blood} that \se{sphuṭita}{bursts out} as far as
the knees, and that is split and oozing, is incurable, and that which
has lasted for a year \se{yāpya}{can be mitigated}.\footnote{The
    sentence appears to describe the condition of the skin, but the word
    “skin” is not expressed.}
  

\item[50cd--51]

When aggravated wind enters into all the \se{dhamanī}{ducts}, the
wind, which moves repeatedly, makes the body \se{ākṣip}{convulse}
quickly and repeatedly. Because of the repeated
\se{ākṣipaṇa}{convulsing} it is traditionally called “\se{ākṣepaka}{The
    Convulsor}.”

	%types of \dev{अपतानक} are being described.

\item[52--56]

	Since a person blacks out (\emph{apatāmyate}) completely, it is
known as a \se{apatānaka}{seizure}.\footnote{Explaining \dev{apatānaka} by 
reference to
    \dev{apatāmyate} is a folk etymology, since the words have different
    etymological roots. 
    
    Gayadāsa, in his commentary on \Su{2.1.52}{265} discussed the
reading \dev{अपताम्यते}, which is also the  reading supported by
witness N but not the vulgate.  This word seems to be unattested
elsewhere.  Gayadāsa defined \dev{अपतानक} as a situation in which
a person sees darkness and loses consciousness (\dev{tamo dṛśyate
mohyate}). Gayadāsa presented a detailed and interesting
discussion of these terms, including citations from earlier
commentators and the texts of Caraka and Dṛḍhabala. Ḍalhaṇa took
up Gayadāsa's discussion and also cited the commentators Jejjaṭa
and Brahmadeva. Brahmadeva was cited often by Ḍalhaṇa and lived
after Gayadāsa and before Cakrapāṇidatta, i.e., in the eleventh
century \pvolcite{1A}[373--374]{meul-hist}.\label{Brahmadeva}}

\item[52cd--53ab] 

If wind that is mixed with a lot of phlegm is present in
the \se{dhamanī}{ducts}, it is called \se{daṇḍāpatānaka}{Stick
    Seizure} because it makes one paralyzed like a stick.\footnote{Against
    \Dalhana{2.1.52}{265}, we read the intensifier \dev{bhṛśam} with 
    \dev{kaphānvita} rather
    than the transitive verb \dev{tiṣṭhati}, for sense.
    
    A verse added in the vulgate at this point asserts that trismus also occurs.}
    
\item[54ab]

The one that bends the body like a bow is technically termed
\se{dhanuḥstambha}{Bow Paralysis}. 

\item[54cd--55cd]

When wind is agitated and located in the fingers, ankles, abdomen, heart, chest, 
or throat and attacks the network of sinews, the person has paralyzed eyes and 
a stiff jaw, their flanks are bent and they vomit phlegm.\footnote{Perhaps 
    the bent flanks, \dev{bhagnapārśva}, are meant to echo the image of the 
    bow, like a scoliosis.}

\item[56]

When a person is caused to bend inwards like a bow,\footnote{It is not
    clear what the qualifier “inwards” is meant to indicate, medically;
    perhaps a form of emprosthotonos.  The verb \dev{nāmyati} is a
    causative, perhaps passive in sense.} then the strong wind causes
    \se{abhyantarāyāma}{internal tension}.

\item[57]

And when the wind is located in the network of external sinews it
causes \se{bāhyāyāma}{external tension}, that breaks the chest, hips.
and thighs. That is untreatable, say experts.

\item[58]

The wind, mixed with phlegm and bile, or even the wind on its own, causes  
another, fourth \se{ākṣepaka}{convulsion} that is caused by 
\se{abhighāta}{trauma}.\footnote{Ḍalhaṇa again cited Brahmadeva's opinion 
on this passage; see note \ref{Brahmadeva} above.}

\item[59]

A \se{apatānaka}{seizure} that arises because of miscarriage,
excessive bleeding and trauma cannot be cured. \footnote{According to
    \Dalhana{2.1.59}{266}, \se{ākṣepaka}{convulsion} is also known as
    \dev{apatānaka}. He further mentioned that even if, fortunately, it is
    cured, it nevertheless cripples the limb.}.

\item[60--61]

When wind that is extremely irritated and strong, proceeds to the
downwards, upwards and horizontal \sepl{dhamani}{pipe}, then,
loosening the bonds of the joints of one side of the body or another,
it destroys that flank. Expert physicians call this
\se{pakṣāghāta}{paralysis}.\footnote{In the \CS, \Ca{6.28.55}{619},
    \dev{pakṣaghāta} “paralysis” was described as \se{ekāṅgaroga}{illness
    of one limb}, which may sometimes have corresponded to the
    contemporary condition monoplegia. Thus, \se{pakṣāghāta}{paralysis}
    may sometimes correspond to conditions that Modern Establishment
    Medicine terms “hemiplegia.” Cf.\ Figure~\ref{2.1.59}{266}.}

\begin{figure}
    \centering
    \includegraphics[width=0.7\linewidth]{"media/Types of paralysis"}
    \caption{Types of paralysis. Image courtesy of 
    \href{https://www.facebook.com/chirosciences/photos/les-différents-types-de-paralysies/599879093836692/}{Chiro
     Sciences}.}
    \label~
\end{figure}
     

\item[62]

If someone is damaged by wind, the whole side of their body is
incapacitated and \se{acetana}{without feeling}.  Then they rapidly
fall down or even die.

\item[32cd--33ab]



% got to here. 
    
    
     



Vitiated wind entered
in the arteries and bends the body like a bow, it is called
\dev{धनुःस्तम्भ} Tetanus. When vitiated wind accumulated in the
regions of finger, ancle, abdomen, heart, chest, and throat
swiftly attack on the group of vain and ligaments, it gets a
person’s eyes stuck, chin stuns, side breaks and vomiting phlegm
he moves inwards like a bow and this situation is known as
\se{antarāyāma}{emprosthotonos}. When vitiated wind attacks on
outside ligaments, body of a person will stretch forward like a
bow. In this situation, if the chest, hip or thigh break, wise men
call it incurable.

\item[58]

	Aggravated phlegm and bile mixed with wind or only vitiated wind causes
	fourth convulsive disease due to trauma.

\item[59]

	Convulsions due to miscarriage, excessive bleeding, and injury are
	incurable \footnote{According to Ḍalhaṇa \se{ākṣepaka}{convulsion} is
	also known as \dev{अपतानक} (Su 1938:266). He further mentions that even
	if fortunately, it is cured, it cripples the limb.}.

\item[60--62]

	When excessively agitated and strong wind flows in the arteries which
	spread downward, upward, and sideways, it loses the joints and kills the
	other side of body. The best of physicians calls it
	\se{pakṣāghāta}{paralysis}.  \footnote{In the ca.6.28.55 \dev{पक्षाघात}
	is described as \se{ekāṅgaroga}{monoplegia}. In that case it damages one
	of the limbs.  In the medical terms \se{pakṣāghāta}{paralysis} is known
	as hemiplegia.} Then half of his entire body becomes inefficient and
	unconscious. Afflicted by wind he suddenly falls or dies.

\item[62.1]

	Bile integrates with wind causes burning sensation, affliction, and
	infatuation. When it integrates with phlegm causes coldness, morbid
	swelling, and heaviness. \footnote{This verse is not available in
	vulgate. It deals with the symptoms when bile and phlegm mix with the
	wind. It is already discussed in su.2.1.38.}. 

\item[63]

	A \se{pakṣāghāta}{paralysis} caused by wind \footnote{Here the term
	\dev{शुद्धवात} suggests the meaning of the wind that is devoid of bile and
	phlegm.} is curable with most difficulty. It becomes curable when caused
	by bile and phlegm mix with the wind. It becomes incurable when caused
	by the loss of bodily constituents.

\item[64--66]

	Verses from 64--66 are not found in the Nepalese manuscripts.  These
	verses discuss the term \se{āpatantraka}{spasmodic contradiction} which
	is the same as \dev{अपतानक}. Ḍalhaṇa commented on ni.1.64-66 (Su
	1938:267) that because of having the similar condition in both
	situations, some scholars do not read the \dev{अपतन्त्रक}. In the verse
	ni.1.59 Ḍalhaṇa commented that the \dev{आक्षेपक} and \dev{अपतानक} is same
	(Su 1938:266) and again he suggested that the \dev{अपतानक} and
	\dev{अपतन्त्रक} both are similar condition. Therefore, \dev{आक्षेपक},
	\dev{अपतानक} and \dev{अपतन्त्रक} should be the same. Gayadāsa further
	commented that the Caraka has not read \dev{आक्षेपक} as \dev{अपतानक} and
	therefore described the \dev{अपतन्त्रक} separately (Su 1938:267).

\item[67]

	This verse also not found in the Nepalese Manuscripts. The verse
	describes \se{manyāsthambha}{rigidity of neck}. According to Ḍalhaṇa,
	rigidity of neck is a prior symptom of spasmodic contradiction. 

	% \se{अर्दित}{spasm of the jaw-bones}
\item[68--72]

	By speaking very loudly, eating hard foods, excessively laughing and
	yawning, lifting heavy loads and sleeping in an awkward position,
	vitiated wind lodges into face painfully and produces
	\se{ardita}{paralysis of the jaw-bones} disease. In that case, half of
	the face and neck become curved, head trembles, speech hindrances,
	deformity occurs in the eys, eyebrows and cheeks.\footnote{Ḍalhaṇa
	suggests \dev{नेत्रादीनाम् इत्यादि शब्दात् भूगण्डादि उपसङ्ग्रहः}} Experts in
	diseases call this disease \se{ardita}{spasm of the jaw-bones}. 

\item[73]

	Spasm of the jawbones cannot be cured when it stays in a person for
	three years, who is very weak, stays without blinking, trembles, and
	constantly speaks gibberish.

\item[74]

	Arteries of Heel and toes stricken by vitiated wind prevents stretching
	of thighs. This disease is known as \se{gṛdhrasī}{sciatica}.

\item[75]

	Arteries which run to the tips of fingers from behind the roots of the
	upper arm affected by vitiated wind terminates all activities of arms
	and back.  This disease is called \se{viśvañci}{paralysis of arms and
	back}. \footnote{Both the MSS N and H read \dev{विश्वञ्चि} instead of the
	vulgate reading \dev{विश्वाची}. There is no such word found in other
	Āyurveda texts.}

\item[76]

	Vitiated wind and blood in the joint of knee causes
	\se{kroṣṭukaśīrṣa}{synovitis of knee join}. In this extremely painful
	situation, the shape of swelling in knee joints seems like a head of
	Jackal. 

\item[77]

	Vitiated wind resides in the waist attacks on the arteries of thigh
	causes \se{khañja}{limpness} and when it attacks on both the thighs a
	person becomes \se{paṅgu}{lame}.

\item[78]

	A person who trembles at the beginning of walking or walks limping and
	whose foot joint has become loose is called
	\se{kalāyakhañja}{lathyrism}.

\item[79]

	Vitiated wind residing in the ankle-joint causes pain when one steps on
	uneven ground. This disease occurs is called \dev{वातकण्टक}.

\item[80]

	Vitiated wind mixed with bile and blood cause burning sensation in feet.
	It should be declared as \se{pādadāha}{burning sensation in feet}.

\item[81]

	A person whose feet tingle and become insensible due to vitiation of
	phlegm and wind is called \dev{पादहर्ष}.

\item[82]

	Vitiated wind lying in the shoulder dries the shoulder joints and it is
	called \dev{अंसशोष}. It also bends the arteries of shoulder, and this
	disease is called \dev{अवबाहुक}. \footnote{Ḍalhaṇa and Gayadāsa both have
	defined two diseases i.e., \dev{अंसशोष} and \dev{अवबाहुक} respectively.}

\item[83]

	Vitiated wind singly or mixed with phlegm cover the channel of ears
	causes deafness.

\item[84]

	Vitiated wind saturated with phlegm covering the arteries which conduct
	the sound of speech makes a person \se{akriya}{inactive},
	\se{mūka}{dumb}. He \se{mimmira}{mumbles} through the nose and
	\se{gadgad}{stammers}.\footnote{Nepalese Manuscripts read \dev{मिर्म्मिर}
	instead of the Vulgate’s reading \dev{मिन्मिण}. Dictionary of MW suggests
	the meaning of \dev{मिर्म्मिर} = having fixed unwinking eyes which is not
	relevant to the disease of tongue.}

\item[85]

	Vitiated wind penetrating into the cheekbones, temporal bones, head and
	neck causes piercing pain in the ears. It is called
	\se{karṇaśūla}{ear-ache}.\footnote{In the medical terms, this disease is
	known as Otitis.}

\item[86--87]

	The pain that arises from the bladder or feces goes down as if it were
	breaking the rectum and…… ? is called \dev{तूनी}, whereas the pain,
	rising upward from the rectum extending up to the region of the
	intestines, is called \dev{प्रतितूनी}.

\item[88--89]

	Retention of vitiated wind inside abdomen causes distension of the
	stomach and flatulence and intense pain and rumbling inside, is called
	\se{ādhmāna}{tympanites}. Vitiated wind mixed with phlegm causes
	\dev{प्रत्याध्मान}. It rises in the stomach anda causes pain in the heart
	and sides. \footnote{There’s an addition in MS N. \dev{नाभेरधस्तात् संजातः
	संचारी यदि वाऽचलः}}

\item[90--91]

	A knotty stone-like tumour caused by wind appearing in the stomach
	having an elevated shape and stretched upward direction which
	obstructing the passage of faeces and urine should be known as
	\dev{वाताष्ठीला}. A tumour of similar shape rose obliquely in the abdomen
	obstructing the passage of wind, faeces and urine should be known as
	\dev{प्रत्यष्ठीला}. 


	Names of diseases discussed in the chapter 2.1

	\se{vātarakta}{Gout} \se{ākṣepaka}{convulsion} \se{pakṣāghāta}{paralysis
	of one side} \se{ardita}{paralysis of the jaw-bones}
	\se{gṛdhrasī}{sciatica} \se{viśvañci}{paralysis of arms and back}
	\se{kroṣṭukaśīrṣa}{synovitis of knee join} \se{kalāyakhañja}{lathyrism}
	\se{vātakaṇṭaka}{vātakaṇṭaka} \se{avabāhuka}{avabāhuka} \se{tūnī}{tūnī} 
	\se{pratitūnī}{pratitūnī}
	\se{ādhmāna}{tympanites} 
    \se{pratyādhmāna}{pratyādhmāna} 
    \se{vātāṣṭhīlā}{vātāṣṭhīlā}
	\se{pratyaṣṭhīla}{pratyaṣṭhīla}




\end{translation}
