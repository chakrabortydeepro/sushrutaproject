% !TeX root = incremental_SS_Translation.tex

\chapter{Śārīrasthāna 1:  On the consideration of all beings}



\section{Literature} 

%Meulenbeld offered an annotated overview of this chapter and a
%bibliography of earlier scholarship to
%2002.\fvolcite{IA}[244--246]{meul-hist}  \citet[chs 6--8]{das-2003} also
%studied topics of this chapter and in chapter 13 provided an overview of
%the conceptual background of ayurveda on the topics discussed in this
%chapter.  

\section{Translation}

\begin{translation}
    
    \item [1] 
    
    So, now we shall explain The Anatomy that is a reflection about all 
    beings.\footnote{The Nepalese version has nouns in apposition 
    (“-\dev{cintā} - \dev{śārīram}”).  The vulgate makes this a single 
    karmadhāraya compound that is slightly easier to parse.} 
    
\item[3]

%The cause of all beings, called “the unmanifest,” is without a cause,
%is characterised by sattva, rajas and tamas,  has eight forms and
%is the reason for the appearance  of this whole world.

That which is called “the unmanifest” is the causeless cause of all living beings,  
having the characteristics of sattva, rajas and tamas, having eight forms, and 
being the reason for the appearance of this whole world.
    
    It is the single basis of the many \se{kṣetrajña}{knowers of the
    field}, just as the ocean is to the beings whose \se{ojas}{power}
is water.\footnote{The Nepalese version differs from the vulgate
    here, introducing the word “power.”  This substantially changes the sense of 
    the passage.}
    
    
\end{translation}
