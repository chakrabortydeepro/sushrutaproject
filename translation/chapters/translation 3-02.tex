% !TeX root = incremental_SS_Translation.tex

\chapter{Śārīrasthāna 2:  On Semen and Menstrual Fluid}

% Jane Allred

\section{Literature} 

Meulenbeld offered an annotated overview of this chapter and a
bibliography of earlier scholarship to
2002.\fvolcite{IA}[244--246]{meul-hist}  \citet[chs 6--8]{das-2003} also
studied topics of this chapter and in chapter 13 provided an overview of
the conceptual background of ayurveda on the topics discussed in this
chapter.  

\section{Translation}

\begin{translation}
    
    \item [1] We shall now explain the anatomy that is the purification of 
    \saneng{śukra}{sperm} and 
    \saneng{śoṇita}{blood}.
%    \q{JG in the light of your reflections, I removed 
%    “women's fertile”. I've put śārīram back in.}
%    
    
    \item [3]  \saneng{retas}{Semen}\footnote{The Nepalese version has
    \dev{-retāṃsi} “semen” (in the plural) as the subject of the
    sentence: “seeds are unable to produce offspring\ldots.”  In the
    vulgate, \dev{-retasaḥ} is a masculine bahuvrīhi, making “men
    whose semen has\ldots” the subject of the sentence.} is
    incompetent to produce offspring if it is [characterized by] wind,
    bile, phlegm, \saneng{śoṇita}{blood},\footnote{Note that the list
        begins with the four entities, wind, bile, phlegm and blood,
        hinting at a four-humour system \citep[see][485--486]{wuja-2000}.}
        \saneng{kuṇapa}{decomposition},
        \saneng{granthi}{clumps},\footnote{\label{granthi}Modern
            Establishment Medicine (MEM) understands that normal ejaculate
            contains coagula which, however, dissolve after about half an
            hour.  But coagula that do not dissolve may sometimes be a sign of
            an underlying disorder (see, e.g., 
            \volcite{2}[614--615]{lamm-1990}; \cite{cohe-1990}).}
            \saneng{pūtipūya}{stinking pus}, \saneng{kṣīṇa}{low volume},
            urine, or feces.
    
 \subsection{Diagnosis by humours}

 \item[4]
 
 \begin{itemize}
     \item  When the dysfunction is caused by wind, there is a colour and a type 
     of pain
     that typically goes with wind problems. 
     \item If  caused by bile the colour and the 
       pain are typical of bile afflictions.  If caused by phlegm the discolouration 
       and 
        suffering are characteristic for phlegm disease.  
        \item And if caused by  
        \saneng{śoṇita}{blood} there will be a colouration due to blood and a 
     sensation of a bile affliction. 
      Moreover, when caused by 
       \saneng{rakta}{blood} there is the \se{kuṇapa}{smell of
     decomposition}.\footnote{Note that the text mentions both \dev{śoṇita} 
     and 
     \dev{rakta}.  This raises the question of whether the author considered 
     these 
     to be different, or whether it is an artefact of textual transmission.}  \item 
     Phlegm 
            with wind causes the appearance of clumps.
         \item   Bile with \saneng{śoṇita}{blood}  causes the appearance of 
         \saneng{pūtipūya}{foul-smelling pus}. 
         \item    Bile with \se{māruta}{wind} cause a weakening of semen.
         \item       \se{sannipāta}{Humoral colligation} causes the smell of urine 
         and 
                feces.\footnote{The expression “humoral colligation,” translating 
             \item       \dev{sannipāta}, refers to the simultaneous 
             \item       disorder of three humors at the same time, a condition that 
             is difficult to 
             treat \citep[see][38 \emph{et 
             passim}]{wuja-2016}.}             
         \end{itemize}     
Cases of foul-smelling sperm, sperm with clumps, and when it reeks of
pus are hard to treat.  But when sperm contains urine or faeces there is no
treatment\sse{asādhya}{incurable}.\footnote{Note that the above
    characterizations presuppose the direct inspection of an ejaculate. 
    The process of collection is not described in the sources in this
    chapter.}
 
 \item[5]
 
 Moreover, \se{ārtava}{seasonal blood} too can become
\se{upasṛṣṭa}{afflicted}, \se{abīja}{seedless} because of the three
humours, and blood as the fourth, taken individually, in pairs or
triples or all together.\footnote{This translates the text of the oldest
    surviving witness, N, and the vulgate.  But MS H, that normally follows
    K very closely, has a negative particle, \dev{na}, reversing the sense
    of the sentence.}
 
 This can also be known by means of the humour, colour and pain.
 
 In these cases, that which displays \se{kuṇapa}{decomposition}, clumps 
 and the putrid smell of pus is \se{asādhya}{incurable}. And otherwise it is 
 \se{sādhya}{curable}.
 
 
%  %
%   Rather it is  the pain 
%  caused or the discoloration of the sperm itself that suggest one of these 
%  afflictions. 
  Among these, the kind which shows decomposition, or coagula, or 
  putrid pus is incurable. The other types, however, can be treated.  
 
 \item[6]
 
 And there is a verse on this. 
 
 \begin{sloka}
      An expert should overcome the first three of these sperm
\sse{doṣa}{pathology}pathologies with special
treatments\sse{kriyā}{treatment} such as unction and sweating,
as well as by means of a \se{uttarabasti}{urethral
    instillation}\sse{basti}{instillation}.\label{uttarabastyantam}%
\footnote{\Dalhana{3.2.6}{345} noted that “unction and sweating”
    indicates the “five treatements”: \dev{vamana, virecana, anirūha,
    anuvāsana} and \dev{uttarabasti}.  He noted that the explicit mention
    of urethral enema in the verse was for the purpose of highlighting its
    priority. However, a natural reading of the verse does not suggest
    that these distinctions were in the author's mind.}\q{find out about
        uttarabasti}
    
 \end{sloka}
 
 \subsubsection{Therapies by humour}
 
 \item[6.1] 
 
In that context, when the sperm is of the nature of wind, there is an
\se{āsthāpana}{enema} consisting of \gls{bilva}, \gls{vidārī} and
milk.\footnote{These three recipes are not present in the vulgate text
    of the \SS.} %
    In the \sse{uttarabasti}{urethral instillation}urethral
    instillations one should use sesame oil well cooked with \gls{madhūka},
    \gls{rāsnā}, \gls{devadāru}, and \gls{sarala}. One can also make
    the patient drink clarified butter with ripe \gls{dāḍima},
    \gls{mātuluṅga} fruit, \gls{saindhava}, a \saneng{kṣāra}{caustic},
    and \gls{vasukavasira}.\footnote{\dev{-vipakva} “well cooked with\ldots” 
    might be interpreted as “with ripe\ldots”.}
 
 \item[6.2]
 
 When the sperm is of the nature of bile, there is  an
\sse{āsthāpana}{enema}enema of milk cooked with curds, \gls{śrīparṇī}
and \gls{madhuka}k. %
One should also apply a \se{kalka}{paste} of \gls{sarja} and
\gls{dhava} in the vagina. %
There is an \se{anuvāsana}{oily enema} of sesame oil cooked with
\gls{madhuka}; in the same way, it should only be applied as a
urethral instillation.\footnote{By specifying “upper (i.e., urethral) instillation”
    the author is clarifying that this is not a rectal enema.}

  One should make him swallow ghee cooked with 
  \gls{kāṇḍekṣu},
  \gls{śvadaṃśtra},
  \gls{guḍūcī},
  \gls{madhuparṇī},
  \gls{bhṛṅga},
  and 
  the \gls{pañcamūla}.
  
 \item[6.3]
 
When the sperm is of the nature of phlegm, there is  an
\se{āsthāpana}{enema} consisting of a \se{kaṣāya}{decoction}  of
\gls{rājavṛkṣa}.  And one should also apply an \se{anuvāsana}{oily
    enema} of sesame oil cooked with \gls{pippalī}, \gls{viḍaṅga} and
honey; and it should only be applied as a urethral instillation. 

One should make him drink a ghee cooked with 
\gls{pāṣāṇabheda},
\gls{kāśmaryā},
\gls{āmalaka},
\gls{pippalī},
\gls{vasuka}, and 
\gls{vasira}.
 
 
\item [3.2.7]
 
  And there are verses about this.
 
 \begin{sloka}
     When there is blood in the sperm, the physician should give the person ghee 
 cooked with 
 flowers of the \gls{dhātakī},
 \gls{khadira},
 \gls{dāḍima},
 and \gls{arjuna}.
 \end{sloka}
 
 \item [3.2.8]
 
\begin{sloka}
     When it smells like a corpse, he should drink ghee cooked with the
\gls{śālasārādi}. %
\dag When clumps appear, it is cooked with stones, or also in ash from a
\gls{palāśa}.\footnote{The Nepalese text and translation of this sentence
    are uncertain. The vulgate text reads, \Su{3.2.8}{345}: \dev{granthibhūte
    śaṭīsiddhaṃ pālāśe vā 'pi bhasmani} “If clumps appear, it is cooked with
    \emph{śaṭī} or in ash from a \emph{palāśa}.”  The vulgate edition notes in a 
    footnote that some vulgate manuscripts add an extra line, \dev{snehādiśca 
    kramaḥ 
    ṣaṭsvetāsu vijānatā}. The Nepalese manuscripts read this line two verses 
    further 
    down.}


 \item[9]
 
And also, when it resembles pus, it is treated with items such as
\gls{parūṣaka} and \gls{vaṭa}.  When the sperm is deficient it should be
treated as was stated before and also as will be
described.\footnote{\Dalhana{3.2.9}{345} noted that “what was stated
    before” refers to the \dev{svayonivardhana} section, i.e., \SS\
    \Su{1.15.10}{69}, and that “what will be described” refers to \SS\
    \Su{4.26}{496}, the chapter on weakness and strength
    (\dev{kṣīṇabalīya}).}
 
 \item [10]
 
 When it looks like feces, he should be made to drink ghee together with
\gls{citraka}, \gls{uśīra} and \gls{hiṅgu}.


 
\item[10.1] 

In these six cases, a wise person should carry out the sequence that
starts with oleation.\footnote{It is difficult to know which six cases 
    the author intended.  \Dalhana{3.2.10}{}
    }\q{to what?}
\end{sloka}


 

\item[10.2--3] 

\begin{sloka}
It deteriorates as a result of not having sex with women for a long
time as well as from the use of actions, and from overusing the drugs
that are astringent, spicy and sharp, that are \se{amla}{acidic},
salty, \se{rūkṣa}{sere}, \se{śukta}{sour} or \se{paryuṣita}{stale},
and because of \se{vegāghāta}{suppressing} the impulses in vaginas and
from \se{gamana}{intercourse}.\footnote{This passage is hard to
    interpret and there are no parallels, commentary or meaningful
    alternate readings.}
\end{sloka}
   
\item[10.4]
\begin{sloka}
   
   When there is a \se{doṣa}{defect} in the \se{ārtava}{menstrual blood}
   one should advise the therapy starting with oleation.
   
   And one should use a \se{uttaravasti}{urethral instillation} exactly as was 
   described before.
    
\end{sloka}


\item[10.5]
\item[10.6]
\item[10.7]
\item[10.8]
\item[10.9]
\item[10.10]
\item[10.11]


\item[10.12]

And there is a verse about this@
\begin{sloka}
To purify the \se{ārtava}{menstrual blood}, one should apply the procedure 
that finishes with a urethral installation.    
    
\end{sloka}

From

\subsection{Therapies for menstrual blood}

\item [12cd]

For purifying the menstrual blood one should follow the procedure, the
last of which is a \se{uttarabasti}{urethral instillation}.\footnote{The
    “procedure ending with a urethral instillation” probably refers to verse
    6 above (see page \pageref{uttarabastyantam}).}
    
    
 \item[13]  
 
 One should use a \se{kalka}{paste} as well as cloths and a salutary
\se{ācamana}{lavages}.\footnote{The word \dev{ācamana}, normally
    “sipping water from the palm” is here translated “lavage” following the
    context and \Dalhana{3.2.13}{345}, who described it as “water for
    washing the vagina” (\dev{yoniprakṣālanodaka}).  This treatment may be
    intended for the condition mentioned in 12cd, but in the vulgate text
    there is a preceding half verse stating that the treatment is for the
    “four disorders of menstrual blood.”}
    
\item[14]

In case of a bad smell and the appearance of pus, or the appearance of
marrow in the blood.
\item [15]

He should drink a \se{kvātha}{decoction} of \gls{bhadraśrī} or a decoction of 
red \gls{candana}.\footnote{The name \dev{candana} may refer to several 
types of sandalwood; presumably one is meant here that is different from 
white sandalwood, i.e., perhaps Pterocarpus santalinus Linn.\ f.  The vulgate 
has an extra half-śloka here.}
 
\item[14ab]
 
 When \se{granthi}{clumps} appear, he should drink \gls{pāṭhā}, 
 \gls{tryūṣaṇa}, and \gls{vṛkṣaka}.\footnote{On \dev{granthi}, see note 
 \ref{granthi}.}
 
 \item[14a] 
 
 He should drink a a \se{niḥkvātha}{decoction} that is the
\se{surasa}{extracted juice} of  a \se{kṣāra}{caustic}, \gls{nāgara},
and \gls{hiṅgu}.
 
\item[\ldots] 
 
\item[24]

Thus a man has unblemished semen and a woman has pure menstrual blood. 
 
 \subsection{During menstruation}
 
 \item[25]
 
During the \se{ṛtu}{season}, starting from the first day onwards, the
\se{brahmacāriṇī}{chaste woman} foregoes bathing, anointments,
ornaments and \se{vilekhana}{grooming}.\footnote{The word \dev{ṛtu}
    “season” in āyurvedic texts can, according to context, refer either to
    the period of menstruation or else to the period of fecundity
    following menstruation \citep[15\,ff., note 27, \emph{et
    passim}]{das-2003}. \Dalhana{3.2.25}{347} noted that the woman's
    abstention should last three days from the first appearence of her
    menses.} She should abstain from sleeping during the day, collyriums,
    \se{aśrupāta}{weeping tears}, massages, cutting her nails, taking
    showers, laughing, telling stories, hearing too much noise and from
    exertion.\footnote{On the similar prohibitions relating to a
        menstruating woman as described in Dharmaśāstra literature, as well as
        the similar defects accruing from disobedience         
        \citep[see][284--287]{lesl-1989}.}
        
For what reason?  By sleeping during the day, the fetus becomes
\diff{deaf}.\footnote{Here, the vulgate reads \dev{svapnaśīlaḥ} “he
    tends to sleep.”} From collyrium he becomes blind.  From weeping, his
    vision is impaired. From bathing and anointing, he becomes badly
    behaved. From massage with oil he gets a \se{kuṣṭha}{pallid skin
        disease}.\footnote{On translating \dev{kuṣṭha} in Āyurvedic texts, see
        \cite[96\,ff]{emme-1984}.} From cutting the nails he gets
        \se{kunakha}{ugly nails}.  From smearing an unguent he becomes bald.
        From habitually exercising in the open air he goes mad. For this
        reason one should avoid these.
    
For three days of ritual food, the husband should \se{\root rakṣ}{protect} the
woman.  She lies on a layer of \gls{darbha}, and eats a different kind of food
from the palm of her hand, or from a plate or from a leaf.\footnote{This
    sentence is hard to construe because \dev{haviṣyaṃ} “ritual food” cannot 
    agree
    with \dev{-bhojinīṃ}.}

On the forth day, one should  show to the husband the woman 
who has
had a purifying bath, is wearing unstitched clothes, is ornamented and who has
chanted a benediction and recited a blessing.\footnote{See \cite[58 and
    fn.\,167]{wuja-2023}.}
    
    What is the reason for that?
     
     
     \item[26]
     And there is a verse on this.
     
     \begin{sloka}
         
         A woman has a bath after her period.  The type of man she sees after 
         that determines the type of son to whom she will give birth. She may then 
         show her son to her husband.     
           
     \end{sloka}
        
        \item[27]
        
\begin{sloka}
            Next, the \se{upādhyāya}{priest} should perform the appropriate 
            ritual 
        for producing a son.  At the end of the ritual, the \se{vicakṣaṇa}{expert} 
        should anticipate the following procedure. 
 
\end{sloka}       

\item [28] Next, after the man has eaten a rice porridge with ghee and
milk in the afternoon, having been celibate for a month, at night he
should sexually approach the woman who has had a diet rich in oil and
mung beans.  He then soothes her in a friendly way and he may go to
her optionally on the fourth, sixth, eighth, tenth or twelfth
day.\footnote{In the Nepalese version, this text presents a general
    rule for lovemaking on even days.  In the vulgate, the word
    \dev{putrakāma} is added, making this a specific rule for conceiving a
    male child.  After this text, sections 29, 30 and 31 of the vulgate
    are not present in the Nepalese version.  These verses state that the
    above-mentioned special days are beneficial, that odd days lead to the 
    conception of a girl child, and finally the vulgate gives a list of the 
    consequences of conceiving a child with a menstruating woman.}
        
\q{29, 30 missing?}
        
 \item [31]

Henceforth, he should approach after a month

[At this point there is a misplaced folio in MS N]
  
\item[32] 

\textcolor{red}{And when conception has occurred in this 
way}\q{\textcolor{red}{Problematic 
passage in the edition.}}

During one of these nights, the pregnant woman should press three or
four drops of juice from one or other of the following:
\gls{lakṣmaṇā}, \gls{vaṭa}, \gls{śuṅgā}, \gls{sahadevā},
\gls{viśvadevā}. Then she should administer them in the right nostril if she
desires a son and in the left if she wants a girl, and she should not
sneeze them out.\footnote{There is a textual problem at the start of this 
passage.}


%\newpage
%
%\begin{tt}
%    \bigskip
%    \raggedright
%    
%
%3.2.32a 
%Here are some more verses.
%
%3.2.11cd 
%On top of that those around her want her smelling sweet as honey, sparkling 
%like a crystal, agile and active, smooth and sweetly perfumed, 
%
%3.2.12ab 
%bright with splendour equally due to the smell of honey as to the smoothness 
%of oil. 
%
%3.2.17 
%It is a token of good health when the menstrual blood is red like a hare’ s 
%blood or like the shine of red lac and when its colour stains can be removed.
%
%3.2.18
%Metrorrhagia or uterine blood loss is diagnosed when there is either 
%excessive 
%bleeding, untimely or irregular bleeding or when symptoms are the opposite 
%of 
%what occurs in a normal menstrual cycle. *
%
%3.2.19 
%Excessive uterine bleeding is always accompanied by aching limbs and with 
%pain. In case blood loss is extremely abundant, symptoms may be weakness, 
%…………………. (bhramamūrcchā), fatigue,
%
%3.2.20 
%And (ca) fever, lamenting pain, anemia, tiredness (tandrā), diseases (rogāś) 
%due to disturbance of Vāta (vātajāḥ), set in motion (hita) by an inhabiting 
%(sevinyā) (disease?) that has just begun (taruṇyā) could become (bhavet) a 
%small (alpa-) disease (-dravam) in a person already labouring under another 
%disease (upa-).
%
%3.2.21cd 
%Because these afflictions have a recurrent character, the woman becomes 
%amenorrhoeic*. 
%
%3.2.22 
%One should impel in such cases in the food meat, Kulattha-pulses, sour 
%Tila-seeds, beans and wine and for drinks (cow)urine, whey and sour curd.*
%
%3.2.23
%In case of thin or abundantly flowing ? menses with features that cannot be 
%treated with drugs, other measures* indicated in case of uterine 
%metrorrhagia 
%must be taken. 
%
%3.2.29 
%The desire is always increasing of knowledge, …… ( dāyura ) and health 
%definitely, of success and power for the husband as well as strength over the 
%days indeed.
%
%3.2.30 
%So then, (a visit to the wife should be made) subsequently on the fifth, 
%seventh, 
%ninth and eleventh day (if) desirous for a female offspring; from the 
%thirteenth 
%day onwards he is to blame.

\item[33]

\begin{sloka}
    For certain, in the presence of these four, a fetus that follows
the rules will come into being, just like a sprout is from a
combination of field, seed, water and grass.\footnote{The Nepalese
    version reads \dev{kṣettrabījodakatṛṇām} “of field, seed, water
    and grass” in contrast to the vulgate's \dev{ṛtukṣetrāmubījānām}
    “of season, field, water and seed.” This gives the two versions
    quite different meanings. In the Nepalese version, the author is
    referring to the four plants mentioned in the previous verse,
    \gls{lakṣmaṇā}, \gls{vaṭa}, \gls{śuṅgā}, \gls{sahadevā}, and
    \gls{viśvadevā}.  Then the author presents a simple agricultural
    simile.  In the vulgate version, the words of the compound each
    have a double meaning: they can refer to the agricultural simile,
    but they can also be construed to mean “menstrual season, womb,
    nourishing bodily fluids, and male and female semen,” a
    parallelism not present in the Nepalese transmission.   This is
    how Ḍalhaṇa interpreted the verse.}
\end{sloka}

\item[34] 

Children born in this manner are beautiful, of noble character and
enjoy long lives.\footnote{We translate \dev{mahāsattvāḥ} as “noble
    character;” Ḍalhaṇa, commenting on the vulgate reading
    \dev{sattvavantaḥ}, refers to the \dev{guṇas}, interpreting the
    expression as “not strongly influenced by \dev{rajas} and
    \dev{tamas}.”}  They provide release from \se{ṛṇa}{obligation} and
    they themselves have children, benefitting their parents.\footnote{Children 
    born in this manner
        fulfil their parent's obligation to have children and they themselves
        have children, thus continuing the family.  The three debts are
        normally understood as being to the gods, the ancestors and to sages.
        But Ḍalhaṇa's phrasing is odd in that he says
        \dev{pitṝṇāmṛṇatrayamokṣaṇaśīlāḥ} “behaving so as to provide release
        from the three debts to the ancestors.”} 

\item[35]

In that context, the element of \se{tejas}{heat} is the most important
factor as far as \se{varṇa}{complexion} is concerned. That being
granted, at the moment the fetus is formed, when the food has water as
its chief element, then the fetus is fair.\footnote{The food of the
    mother, that is.}  When earth is the predominant element, it is
    \se{kṛṣṇa}{dark}. When earth and ether are the chief elements, it is
    \se{śyāma}{dark brown}.\footnote{The terms \dev{kṛṣṇa} and \dev{śyāma}
        often mean more or less the same, a dark blue or black colour. The
        latter can shade into brown or dark green.}  Some people say that the
        \se{prasava}{newborn} has the same colour as the colour of the food
        that the pregnant woman commonly eats. Similarly, creatures like
        snakes, scorpions and \glspl{galagoḍikā} that inhabit black, yellow or
        white habitats are black, yellow or
        white.\footnote{\label{galagodika}Cf.\ also n.\,\ref{godheraka},
            p.\,\pageref{godheraka}. Cf.\ \volcite{IA}[70 and notes]{meul-hist} on
            these poisonous animals as described in the \CS, and
            \cite[455-456]{meul-1974} on the names \emph{kṛkalāsa\slash
            kṛkalāśaka, śaya} and \emph{saraṭa} and the confusion surrounding this
            topic and the indigenous names of some species such as \emph{ṭikṭikī,
            jyeṣṭhi, jyaiṣṭhī}, \emph{girgiṭ}.}
            
            In that context, \se{jātyandha}{congenital blindness} is
caused by the element of \se{tejas}{brilliance} not
reaching the location of \se{dṛṣṭi}{eye}.  Similarly, red
eyes are a consequence of blood, white eyes are a
consequence of phlegm, yellow eyes are a consequence of
bile, and \se{vikṛtākṣa}{dysfunctional eyes} are a
consequence of wind.\footnote{The term \dev{vikṛtākṣa}
    was known to Kātyāyana (\emph{Mahābhāṣya} on P.6.3.3,
    \pvolcite{3}[142]{kiel-1880}).}
  
% got to here 2024-06-14 


\item[35.1--4]

And on this, there are the following:\footnote{The next four verses are 
absent in the vulgate; they were reproduced by the 
editor in a footnote (\cite[348a, n.\,3]{vulgate}).

The phrase “and here are some verses” appears 
in the vulgate before 3.2.36.}

\begin{sloka}
    %1
If a pure wind affects someone's eyes, they become
sunken, blue and dark.

% got to here 2024-08-30
    
    %2
When bile mixed with phlegm, with no impurity, goes into
someone's eyes, their eyes are termed “yellowish-red.”
    
    %3
When phlegm that is free of any impurity moves to the eyes, their
eyes shine with a white circle within a circle.\footnote{Perhaps this
    describes the appearance of arcus senilis.}
    
    %4
When blood mixed with phlegm moves into the eyes, those people
have eyes that become pigeon-blue, or else bloodshot.
  
    \end{sloka}

\item [3.2.36]

Just as the ghee in a pot placed on a fire melts, so the menstrual
blood of a woman may flow out after sex with a man.\footnote{It is
    difficult to know what the author means here, since menstruation is
    not physiologically caused by intercourse.
    
Note that the text actually says “a pot of ghee \ldots\ melts.”  But it's
not the pot that melts, but the ghee.  This may explain the vulgate
reading \dev{ghṛtapiṇḍa} “a lump of ghee.”  The reviser did not like
the imprecise idea of a pot melting.}

\item [3.2.37]

But when the wind splits the \se{bīja}{seed}, two lives 
(\emph{jīva})\sse{jīva}{life} come into the \se{kukṣi}{belly}.
They are called “\se{yama}{twins},” being created from preceding
\se{dharma}{virtue} or its opposite.\footnote{Note the adverbial 
\dev{-purā} at the end of a Bahuvrīhi.  

The commentator Gayadāsa (cited here by Ḍalhaṇa) disagreed with this
interpretation.  He preferred to understand \dev{dharmettara} not as
“dharma and its opposite,” but as “the opposite of dharma.” He
explained that according to both scripture and tradition, twins are
the result of \dev{adharma} “sin,” and that is why penances are
necessary after the birth of twins (on \Su{3.2.27}{348}).

The next two verses are absent in the vulgate; they were reproduced by the 
editor in a footnote (\cite[348b, n.\,3]{vulgate}).}

\item [3.2.37.1]
\begin{sloka}
    When the mixing is happening, if the man's \se{retas}{semen} is plentiful 
    and pure then the pregnant woman gives birth to two boys.
\end{sloka}

\item [3.2.37.2]
\begin{sloka}
    When the mixing is happening, if the woman has a lot of 
    \se{śukra}{semen} then the pregnant woman gives birth to two girls.
    There is no doubt about this. 
\end{sloka}

\item[3.2.38]

The term for men and women who have diminished seed is 
\emph{Āsekya}.\footnote{Etymologically, “to be poured into."}  Without 
doubt, after eating something 
\se{śukla}{white}, his flag is raised.\footnote{\Dalhana{3.2.38}{348} made 
it clear that this is a metaphor for having a penile erection.\label{erection}

“Eating something white” may refer to \dev{śukra} “sperm,” the 
reading of the vulgate, but note that works on aphrodisiacs and fertility 
(\dev{vājīkaraṇa}) in āyurveda and rasaśāstra routinely recommend white 
substances such as milk. See, for example, \Su{4.26.27--31ab}{498} and 
\Ca{6.2, all of sub-chapter 2}{392--394}.

The vulgate has a different reading for the first half of this verse, stating that 
such a man is a product of parents with deficient seed.  Ḍalhaṇa also gave a 
detailed description of a man eating the semen ejaculated by another man, 
and he stated that the terms \dev{ṣaṇḍa} and \dev{mukhayoni} were 
synonyms for such a person.

The term \dev{āsekya} is given in \cite[161]{moni-sans} as “impotent,
a man of slight generative power.”  This is wrong.  It is the
referent of the term, not its meaning. Cf.\ \volcite{1}[98]{josi-maha}.

Some of the features referred to by the term \dev{ṣaṇḍa/ṣaṇḍha} may
have included conditions today covered by
Mayer-Rokitansky-Küster-Hauser syndrome and Morris syndrome.  The
central idea in the Sanskrit usages was that such a person cannot
produce children.}

\item[39]

Someone who is born in a foul womb is termed a \emph{Saugandhika}. 
That person gains strength from smelling a vagina and a
penis.\footnote{Etymologically, “Sweet Smelling."}

\item[40abc]

A man, who has activity in his own anus because of being celibate and
then has activity amongst his own women is known as a
\emph{Kumbhīka}.\footnote{The vulgate adds an avagraha before
    \dev{brahmacaryād}, meaning “because of \emph{not} being celibate."
    \Dalhana{3.2.40abc}{348--349} read the text this way, paraphrasing
    \dev{abrahmacaryāt}, thus inverting the meaning but not clarifying
    what he thought it meant.  But he then cited a passage from “others”
    that read \dev{brahmacaryāt}, i.e., the anal sex followed or was
    caused by celibacy, \dev{brahmacaryāt
    klaibyavaśasaṃjātāpravṛttitvāt} “because of celibacy, that is,
    because of being unable to perform because of the effect of
    impotence."  These unnamed commentators also referred explicitly to
    erectile dysfunction, \dev{śithilenaiva mehanena}, as the result of
    this celibacy and proposed that a man could get an erection through
    abnormal (\dev{viprakṛtyā}) means and as a result could have sex as a
    male with a woman.  Ḍalhaṇa also stated that the origin of a person
    with such a condition was described “in another book”
    (\dev{tantrāntare}), and proceeded to cite \CS\ \Ca{4.2.20}{303}.  
    Ḍalhaṇa then also cited another verse from Gāyadāsa, who himself 
    ascribed it to Kāśyapa \pvolcite{IA}[164--166]{meul-hist}, saying that, “A 
    Kumbhila (\emph{sic}) is born when a man with phlegm for semen 
    has sex with a woman who is not passionate (or not menstruating) during 
    her season, when the love is attached to another.” (Also cited in 
    \volcite{1}[220a--b]{josi-maha}.)
    
    It is noteworthy that the \SS\ is factual and descriptive in
these passages, as befits a medical work, while the commentators
introduce a moralistic and critical tone.}

\item[40d--41abc]

Hear about the next one, the \emph{Īrṣyaka}.  
Someone who has sexual activity after seeing the copulation of other people 
is termed an Īrṣyaka.\footnote{Etymologically “one who envies."  
    
    Here again, \Dalhana{3.2.40--41}{349} cited the opinion of “another 
    book” and cited a passage from \CS\ \Ca{4.2.20}{303} that covers similar 
    ground.  The description of the \CS\ is causally framed in terms of the 
    factors \dev{vāyu} and \dev{agni}.} 

\item [41d--42]

Hear about the fifth, the \emph{Ṣaṇdhaka}.  A man who, out of
delusion, has sexual activity with a \se{kaumārī}{young girl} during
her season as if he were a woman.  In such a case, a male is born who
looks and behaves like a woman.  He is termed a
\emph{Ṣaṇḍha}.\footnote{The vulgate's \dev{bhāryā} “woman, wife” for
    \dev{kaumārī} “girl” is probably bowdlerization.}
    
\item[43]
    
Moreover, if a woman, during her season, has sexual activity like a man, then 
if a girl is born she will have the behaviours of a man.

\item[44]

The \emph{Āsekya}, the \emph{Sugandhin}, the \emph{Kumbhīka} and the
\emph{Īrṣyaka} are known to have semen.  The man with no semen is
termed a \emph{Ṣaṇḍha}.\footnote{It remains a question as to whether the 
authors meant the absence of an ejaculate or the clinical observation of 
childlessness even in the presence of ejaculations.}
    
\item [45]

These two have a semen-carrying vessel that dilates as a result of
unnatural excitement.\footnote{We have emended the Nepalese verb to
    the singular, because witness H clearly has \dev{śukravahā sirā}
    “semen-carry vessel” in the singular. Does Ayurvedic anatomy have a
    single vessel or many? \CS\ \Ca{3.5.8}{250} has a plural,
    \dev{śukravahānāṃ srotasāṃ}.  But the \SS\ \Su{3.9.12}{3.9.12} has a
    clear statement that there are two \se{srotas}{ducts} that carry semen: 
    \dev{śukravahe dve tayur mūaṃ
    stanau vṛṣaṇau ca} “there are two vessels that carry semen.  They are
    rooted in the breasts and the testicles.”  The Ayurvedic Man painting
    has a single \dev{śukramārga} \citep[233, 243]{wuja-2008a}.}  Then
    the flag may be raised.\footnote{On this euphemism, see foonote
        \ref{erection} above.}


    
\item[46]

The \diff{appearance}, behaviour and mentality that is associated with a 
man and a woman is the same as that which their fetus has.\footnote{The 
vulgate has “food” for the Nepalese version's \dev{ākāra} “appearance.”  
The Nepalese version seems more perceptive on this point of heredity.}



    


    
    

%3.2.39 
%He who is born in a sordid vagina is commonly known as a saugandhika. 
%Such 
%a person becomes aroused only after smelling a vagina or a scrotum.*
%
%3.2.40 
%When a man first had same-sex anal intercourse because of a period of 
%sexual 
%abstinence from women and then turns towards his regular partners* again, 
%he 
%should be known as a kumbhīka. And now get it right about what an 
%īrṣyakaṃ 
%is:
%
%3.2.41 
%somebody who has to watch sexual intercourse of others before being able 
%to 
%his own sexual activities should be known as an īrṣyakaḥ.* He who turns 
%towards copulation**
%
%3.2.42 
%during the fertile days of the cycle* but out of pure sexual ignorance 
%ejaculates 
%on the breasts of his virgin wife** will create boys who also exhibit feminine 
%character traits.
%
%3.2.43 
%If a woman in her fertile days* throws herself at the feet of males around her 
%and if she begets a girl, she will also have character traits of a man. *** 
%
%3.2.44 
%Men who do produce sperm but have a pathology can be identified as 
%āsekya, 
%sugandhi, kumbhika or īrṣyaka. Men who do not produce any sperm are 
%called 
%saṇḍha.*
%
%3.2.45 
%The sperm conducting channels of both these (men) should be changed? 
%retaliated? opposed?  By an erection towards openness thus there should be 
%a 
%going towards a rising of the flag. 
%
%3.2.46 
%Just as by similar eating habits both boys and girls (are) connected, just so 
%also 
%foetuses should definitely be(come) in it (the womb).
%
%3.2.47 
%However, whenever a woman and another woman through sexual 
%intercourse 
%succeed in accomodating … yopa (?) each other’ s sperm, then a boneless 
%(being) is born.
%
%3.2.48
%Furthermore a woman could even be carried away into a sexual climax in a 
%dream after having taken her ritual bath. The Vāyu, having taken the egg 
%into 
%the uterus, makes in the belly surely…
%
%3.2.49 
%Month by month in the pregnant woman the signs of pregnancy become 
%apparent. From such a woman a kalala* is born, on account of the absence 
%of 
%paternal qualities.**
%
%3.2.50 
%However, disfigured creatures such as serpents, scorpions and 
%pumpkin-gourd 
%shaped foetuses, they, verily as well as others, should be known (to be) very 
%frequently born from the womb the consequence of sins.
%
%3.2.51
%A womb of a mother (who is) disrespected because of an excess of wind 
%(results in a child that) could be in danger (to become) humpbacked or 
%rather  
%…… (kūnipaṇgur) or dumb.
%
%3.2.52 
%Because of being without religious teacher and (ca) because of misfortunes 
%of 
%the parents or due to the excessiveness of the wind-eaters the child  could 
%obtain disfigurement.
%
%3.2.53 
%Due to the scantiness of bodily excretions, itself due to a disabling of Vāyu 
%with respect to processing of food, the foetus, whilst in the womb, produces 
%(almost)* no urine nor stools. 
%
%3.2.54 
%Due to the dwindling of the Vāyu in the face, the covered parts and the 
%narrowest parts (all) wrapped up by phlegm (kapha-), the foetus while it is in 
%the womb because of obstruction of the going does not weep all the time*.   
%
%3.2.55 
%Thus the foetus, provided with the movements of inhalation and of 
%exhalation, 
%goes on; the coming together of (its) moments of sleep with the movements 
%of 
%inhalation  and of exhalation of the mother.
%
%3.2.56 
%The (adjustment?) of the (limbs of the) body and both the appearance  and 
%the 
%falling of the teeth, even (ca) the very non-appearance of hairs  in the palms 
%of 
%the hands and the soles of the feet, (goes) according to its intrinsic nature ||
%
%3.2.57 
%Men who have uninterruptedly entered one previous existence after another 
%and who have a vast understanding of the scriptures, do remember their own 
%previous births.
%
%This was the second chapter of the śārīrāsthana.
%  
%\end{tt}

\end{translation}
