% !TeX root = incremental_SS_Translation.tex

\chapter{Śārīrasthāna 3:  On Conception and the Development of the 
Embryo}

%First draft, by Jan Gerris, 2023-12-19. 

\section{Literature} 

Meulenbeld offered an annotated overview of this chapter and a
bibliography of earlier scholarship to
2002.\fvolcite{IA}[247--247]{meul-hist}  Important subsequent 
studies of the chapter include those of \citeauthor{das-2003} and of
\citeauthor{krit-2009}.\footnote{\cites[ch.\,8, et 
passim]{das-2003}{krit-2009,krit-2013}; 
see also the valuable terminological study by 
\citet{sune-1991}.}

\newpage
\section{Translation}


\begin{translation}


\item[1] 

We shall now explain the anatomy that is the descent of the embryo. 

%We are now about to begin to explain how the embryo is conceived, 
%nestles and develops* once it arrives in the body.
\subsection{Conception}
\item[3]

Semen is \se{saumya}{of the nature of Soma} and menstrual blood is
\se{āgneya}{of the nature of Agni}.\footnote{On the Saumya--Agni
    classification, see \cites{wuja-2004}{ange-2021}[521--527]{das-2003}.
    The fiery nature of menstrual blood is already stated in
    \Su{1.14.7}{59}, “\ldots but menstrual blood is of the nature of Agni,
    because the embryo is of the nature of fire and water.”} Furthermore,
in this context there also exists a proximity of the other elements
(\emph{bhūta})\sse{bhūta}{element}, by way of a minute special
property\sse{aṇu}{minute}\sse{viśeṣa}{special property}, because they
help one another and they enter into one
another.\footnote{\Dalhana{3.3.3}{350} glossed \dev{aṇunā viśeṣeṇa}
    “by way of a minute special property” as \dev{sūkṣmaprakāreṇa} “in an
    attenuated manner.”\label{anuna-visesena} I am grateful to Christèle
    Barois for drawing attention to the treatment of this topic, and
    specifically the \dev{parasparopakāra} “mutual support” between atoms,
    by the Buddhist author Śubhagupta (fl.~720--780) 
    \cite[126]{sacc-2015}.
        
        \Dalhana{3.3.3}{350} drew attention
        to \Su{3.1.21ab}{343} where the idea of this interpenetration
        (\dev{anupraveśa}) is mentioned.}


\item [4]

In this case, when there is a union of \diff{a husband and wife}, the
wind from the body stimulates the \se{tejas}{heat}.

In that case, because of the \se{sannipāta}{colligation} of fire and
wind, the semen that is ejaculated finds its way to the vagina.

It is commingled with \se{ārtava}{menstrual blood}, then because
of the joining together of Agni and Soma, what is being mingled together
arrives in the receptacle of the fetus. 

He is referred to by names that express synonyms such as, the knower
of the field, the sentient, the toucher, the smeller, the seer, the
hearer, the taster, the human, the goer, the witness, the creator, the
speaker, \diff{the one who is, “who is the one that is life at the
    start?”}\footnote{The last phrase is awkward.  It translates \dev{yaḥ
    ko'sāvādya āyuriti}, which could be paraphrased, “the one who is the
    answer to the question `who is the one who is life at the outset?'” or
    “\ldots `who is that first one who is life?'.” The text differs from
    he vulgate's \dev{yaḥ ko 'sāv iti}, that omits \dev{ādya āyur}
    (\Su{3.3.4}{350}). Most other early editions print \dev{yo'sāviti}
    (e.g., \cites[v.\,1, 320]{gupt-1835}[313]{bhat-1889}[v.\,3,
    30]{bhat-1908}[v.\,2, 635]{sarm-1895}. \citet[v.\,2, 65]{ghan-1936}
    read \dev{yaḥ ko'sāvity}). No other translators translate this
    phrase, nor does Ḍalhaṇa gloss it.} %
    %    vaktā yo'sāv ity evam - Madhusūdana Gupta 1835
    %    vaktā yo'sāv ity evam - Jīvānandavidyāsāgara
    %    vaktā yo'sāv ity evam - Candrākānta Bhattacharyya p.757
    %    vaktā yaḥ ko'sāv ity evam  - Ghanekar 1936
    %    vaktā yo sāv ity evam - Muralidhara Sarma
    %



%Sperm from the male absorbs heat whereas eggs from the female 
%release heat. With respect to this aspect, the way the different basic 
%elements of 
%matter behave depends on how the elements specifically react with one 
%another 
%and how they form bonds with one another.  


Driven by fate, and impelled by wind, the imperishable, unchanging,
inconceivable \se{bhūtātman}{elemental self} enters into the
\se{garbhāśaya}{uterus} together with sattva, rajas and tamas, gods
and demons, and other entities.\footnote{In the vulgate,
    \dev{bhūtātman} “elemental self” is not the subject of the sentence,
    which then reads less clearly overall.}

\item [5]

In that context, a predominance of sperm leads to a male, a
predominance of menstrual blood leads to a female, and equality of the
two leads to a person who is \se{napuṃsaka}{neither male nor female}.

\item[6ab]

In that context, there is a twelve-night period that is the
\se{ṛtu}{season}.\footnote{\citet{slaj-1995} clarified the
    misconception in early Indological scholarship that \dev{ṛtu} referred
    to the period of the menses rather that this longer period of menses
    and ovulation.}

\item[3.3.6.1]

$\dag$In that context, approaching a woman in season for intercourse
during the first day is \se{anāyuṣya}{not conducive to long life}; a
man comes into being.\footnote{This passage appears in the Nepalese
    version at this point, and is absent from the vulgate version.  MS H
    is the sole witness to the Nepalese version at this point and it is
    damaged, making the interpretation of this passage difficult.  In this
    sentence, a nominative would read better than the accusative
    \dev{anāyuṣyam}.} %
    To the extent that the fetus is deposited at that time, because of
    being expelled it is lost.\footnote{In this and the following
        sentences, parts of witness H are damaged and impossible to read.}
        $\dag$
    
    And on the third day, similarly, the body is incomplete and has little duration of 
    life.  For that reason, one should avoid the third night. 
    And seed and menses do not develop the proper quality as expected.     
   $\dag$Just as an object thrown into a river against the flow does not come 
   back.$\dag$  Sperm should be seen the same way.  Therefore the restricted third 
   night should be avoided.  In this context, after seeing the twelve nights of the 
   season, she has no menses. 

\item [6cd]

Some call such women, “having invisible menses."

\item [3.3.9]

And on this:
\begin{quote}
    When the day is over, the lotus inevitably closes.  In the same
way, when the season is over, the woman's uterus
closes.\footnote{The \root \emph{kuc} “close, contract” appears in
    this sense in the \emph{Dhātupāṭha} (1.199 \dev{saṃkocane}) but it
    is not common in literature.  The more common word in this sense
    would be from \root \emph{kuñc} “contract,” although \emph{kuc} is
    probably the primary IE form \pvolcite{1}[361]{EWA}. “Given by the 
    grammarians as two distinct roots, not without some justification,” 
    \cite[19]{whit-root}.}
\end{quote}

\item[3.3.7--8]

One may know that a woman has her season because she has a full, clear
face, a moist body, mouth and teeth, she desires a man, she speaks
nicely, and she has relaxed belly, eyes, and hair.  Her arms, breasts,
loins, navel, thighs, hips and bottom are vibrant and she
has the utmost excitement and eagerness.


%\footnote{The rise of oestrogen \ldots\ }


\item[3.3.10]

At the right time, what has accumulated over a month and has come via
the two \se{dhamanī}{pipes} is led by wind towards the mouth of
uterus.\footnote{“Pipes” (\dev{dhamanī}) are defined in the \SS\ at
    \Su{3.9.8--11}{385}. This verse was discussed by
    \citet[64--66]{das-2003} (see some corrective remarks by
    \citet{voge-2005}.) On the “pipes” and other conduits in the āyurvedic
    body, see also \cite[404--406]{wuja-2022}.}  It is slightly dark and
    smells.\footnote{The reading of the vulgate text contains the object
        of the sentence, \se{ārtava}{menses}, explicitly.  The commentators
        take “at the right time” to indicate the onset of menses in a young
        woman.}

\item[3.3.11]

 From twelve years onwards, blood is present periodically. It ceases
after fifty amongst those whose bodies are old and aged.

\item[3.3.12]

It is declared that there will be a male on even days and a female otherwise. 
Therefore a clean man who wants descendants should approach the woman at the 
time of her flower.\footnote{”Flower” referring to the twelve-day period that has 
been discussed earlier.

\Dalhana{3.3.12}{352} noted the conflict between the idea presented in
passage 5 above and the present idea about odd and even days. He
quoted passages by the ancient authorities Videha (see footnote
\ref{Videha}) and Bhoja (footnote \ref{Bhoja}) that squared the
circle by asserting that there are greater amounts of semen on even
days, and greater amounts of menstrual blood on odd days, etc.  See tr.\ by
\tvolcite{2}[143]{shar-1999}.}

\subsection{Pregnancy}
\item [3.3.13]

In that context, women who have recently become pregnant experience 
tiredness,
fatigue,
thirst,
heaviness of the legs,
flatulence,
clogging of semen and blood,
and a rough pulsation of the vagina.

\item [3.3.14ab]

And about this, there is the following:

\begin{quote}
    The sign of a pregnant woman is said to be: both nipples become
dark and a row of hair appears, there is nausea\sse{praseka}{nausea}
and tiredness\sse{sadana}{tiredness}.
\end{quote}

\item[3.3.16]

From that moment onwards, she should not practice intercourse,
exertion, excessive dieting, sleeping by day, waking at night, grief,
riding in a vehicle, fear, excessive coughing, or therapies like
oleation or bloodletting while alone and at the wrong
time.\footnote{The vulgate passage 3.3.17, which is not present in the
    Nepalese version, presents the doctrine that if a part of the pregnant
    woman's body is assailed by a humour, that same part of the child's
    body in the womb will be damaged.  A similar idea is presented in 
    3.3.21 below and previously in 3.2.25 (p.\,\pageref{3.2.25}).}


\subsection{Fetal development}

\item [18]

In that connection, in the first month, a \emph{kalala} comes into
being.\footnote{On \emph{kalala}, see the useful historical notes by
    \citet[535--536]{das-2003}, that may suggest a meaning such as
    “slime.”  For a discussion of these terms in Buddhist and other
    contexts, and further literature references, see
    \cite{agos-2004,krit-2013,krit-2009,sune-1991}. }
    
In the second, ripening by means of blood, heat and air, a conjunction
of the great elements becomes a \emph{ghana}.\footnote{The word
    \dev{ghana} in the sense “coagulate, lump” is normally masculine in
    this sense, but is neuter in the Nepalese version.}  If it is a
    \se{knot}{granthi}, it is a male; it is a woman if it is a
    \emph{peśī}; it is a neuter if it is an \emph{arbuda}.
        
In the third, the hands feet and head develop into five
\se{piṭaka}{bulges}.\footnote{The word \dev{piṭaka} “bulge” usually
    means “basket."  Here, perhaps, it suggests a small upside-down
    basket. \cite[652]{moni-sans} cites the word from the
    \emph{Carakasaṃhitā} in the sense “blister.”  The vulgate normalizes
    the word to \dev{piṇḍaka} “lump.”}  And the distinction of the limbs
    and \se{pratyaṅga}{minor body parts} is minute.
        
In the fourth, the distinction of the limbs and \se{pratyaṅga}{minor
    body parts} become \se{pravyakta}{apparent}.  
    
In the fifth, the distinction of the limbs and \se{pratyaṅga}{minor
    body parts} become \se{pravyaktatara}{even more apparent}.
    
The \se{cetanādhātu}{element of consciousness} becomes
\se{abhivyakta}{manifest} because of the fact that the heart of the
fetus becomes apparent.\footnote{The Nepalese version of this passage
    is interestingly different from the vulgate and, as usual, contains
    some puzzles.}   
    
    How so?  
    
    Because it (the consciousness) is located
    there.\footnote{The word \dev{kasmāt} “how so?” could, because of
        sandhi, be read \dev{akasmāt} “for no reason, suddenly.”  This would
        radically change the meaning of the passage: “The element of
        consciousness suddenly (or “for no reason”) becomes manifest because
        of the fact that the heart of the fetus becomes apparent.”} 
        
        During the
        fourth month the fetus develops \se{abhiprāya}{intentionality} with
        respect to the objects of sense. And the woman starts to have two
        hearts; she perceives its \se{nimitta}{purposes}.\footnote{The subject
            of the sentence, “she,” probably refers to the woman, but may refer to
            the fetus, “it reveals its goals.”  It is not clear why the focus of
            events has jumped back to the fourth month.} If the dual-hearted
            nature of the woman is ignored, she will give birth to a hunchback
            with a withered arm, \se{ṣaṇḍa}{a man with no semen}, a dwarf with
            \se{vikṛtākṣa}{dysfunctional eyes}, or someone eyeless.\footnote{The
                term \dev{ṣaṇḍha} is discussed on p.\,\pageref{ṣaṇḍha} above.}
                Therefore she should be given whatever she wants. With her
                dual-hearted nature being acknowledged, she will give birth to someone
                heroic and long-lived.
        

\item[19]

The physician should gather and give to the pregnant woman whatever objects of 
sense she wishes to experience, because of the danger of damaging the fetus. 

\item [20]

A woman whose pregnant cravings have been satisfied will give birth to
a son full of good qualities.  And a woman whose pregnant cravings
have not been satisfied causes danger for the fetus or
herself.\footnote{The \dev{garbha} “fetus” could also mean “the womb.”
    \dev{ātman} “(danger for) herself” could mean “for the body (of the
    fetus).”}
    
\subsection{Effects of the mother's experiences on the unborn child}
  
\item[21]

When a woman, \se{dauhṛda}{sharing her heart with the fetus}, is
slighted in respect of one of the objects of sense, she will bring
forth a son who suffers pain in that selfsame sense
organ.\footnote{Note the historical and scribal confusions of forms
    connected with \dev{dvi-hṛd} “two-heart” and \dev{dohada} “pregnant
    longing” (from two-heartedness with the fetus) as opposed to
    derivatives of  \dev{dur-hṛd} “bad-heart,” such as \dev{daurhṛda}
    “bad-heartedness.”    The lexeme \dev{dauhṛda} “having pregnant
    longings (from two-heartedness),” is a  false Sanskritization of the
    MIA \dev{दोहळ}, itself < *\dev{dvaihṛda} \citep[46, 183
    n.\,2]{lued-1940}. Cf.\ further notes, parallels and confusions in
    \cite[\#6690]{CDIAL}.  The expression  “morbid cravings,” appearing in
    translations and dictionaries, is the result of conflating the two
    distinct historical forms.}
    
    \item [22]
   
A woman who has a \se{dauhṛda}{pregnant longing} to see a king gives
birth to a son who is wealthy and very fortunate.
    
\item[23]

A woman \se{dauhṛdā}{sharing her heart with the fetus}, who is in 
fine raiment, undergarments, silk and decoration, 
will produce a charming son who likes ornamentation. 

\item[24]

When she is in an ashram, she gives birth to one who is self-restrained and 
habituated to virtue. 

If she gives birth in the presence of an image of a deity, her child
is like one who \diff{gives joy}.\footnote{The reading of the vulgate,
    \dev{pārṣada-} “is like an attendant," makes better sense than the
    Nepalese \dev{harṣada-} “one who gives joy.”}

If she is within sight of wild species of animals then she gives birth
to one who has violent habits.


\item [25]

The son of a woman who eats \gls{godhā} has an inclination to sleep
and a murderous nature.\footnote{The noun \dev{suṣupsur}, m.,
    “sleepy,” is nominative when it should be accusative (as in the
    vulgate).  Perhaps we have a change of gender as documented for epic
    Sanskrit by \cite[xxxviii--xl, et passim]{ober-2003}.}  If she eats
    the meat of cattle, he is born strong and tolerant of all suffering.
    
\item [26] 
    
Because of pregnant craving for buffalo meat, the son is a hero, has red eyes and is 
hairy.\footnote{At this point, The Nepalese version does not include the vulgate's 
passages on eating boar, deer, and partridge and their consequences for the child.}

\item[28]

Therefore, as regards things that have not yet been mentioned, if a
woman concentrates on feminine  pregnant cravings she will cause a son to be 
born who is the same, in terms of body, diet and behaviour.

\item[29]

What will happen, impelled by the person's karma, recurs repeatedly.  In the same 
way, the effect of fate generates \se{dauhṛda}{pregnant craving} in her heart. 

\item[30]

In the fifth month, the mind becomes more awakened. %
In the sixth, the intellect.  In the seventh, the body becomes
\se{differentiated}{pravyakta} in all parts. %
In the eighth month the \se{ojas}{vital energy} is unstable; one born
at that time does not survive.\footnote{On the concept of \dev{ojas}
    and its translation as “vital energy,” see \cites[xl, et
    passim]{wuja-2003}[530--535]{das-2003}.} 

Then, a \se{bali}{ritual offering} of meat and boiled rice should be
given for him as \se{bh\=agadheya}{tribute} because tribute is due to
Nair\d{r}ta.\footnote{Nair\d{r}ta is a demoness who threatens
    children.  In his commentary on this passage, \d{D}alha\d{n}a cited a
    passage from the \emph{Kum\=aratantra} (\cite[353]{vulgate}).  On this
    work and its genre, see
    \cites[261--264]{wuja-1999}{bagc-1941}{fill-1937}.}

The birth happens on any of the ninth, tenth, eleventh or twelfth months.  If it is 
different than this, there will be something wrong with him. 

\item[31]

As a matter of fact, the \se{garbhanāḍī}{fetal conduit} is connected
to mother's \diff{navel} that supplies \se{rasa}{chyle}.\footnote{In
    the vulgate text, the umbilical is connected to the mother's
    \dev{nāḍī} not \dev{nābhi}.  Also, the vulgate is explicit that the
    umbilicus is connected to the fetus's navel.
    
    From the contemporary physiological view it is the mother's
placenta, not navel, that connects with the umbilical cord.  In
contemporary usage, a navel can only be a post-delivery anatomical
region, and the fluid flowing in the cord is blood, not chyle.} It
supplies his mother's \se{vīrya}{strength} that comes from the
\se{rasa}{essence} of food.\footnote{Or “it supplies the mother's
    \sepl{rasa}{taste} and \se{vīrya}{strength} that come from food.”
    The option here is whether the terms \dev{rasa} and \dev{vīrya}
    should be taken in the technical pharmacological sense (\dev{rasa,
    vīrya, vipāka, prabhāva}, \cite[see][]{meul-1987}), or as generic
    adjectives.  Ḍalhaṇa did not comment on this issue.} Due to this
    \se{upasneha}{infusion}, it grows bigger. That causes it to live,
    even before the differentiation of the limbs has begun, because of
    the infusion of the criss-crossing \sepl{dhamanī}{duct} that carry
    \se{rasa}{chyle} and that from conception onwards run through the
    whole body.

\subsubsection{The formation of the embryo}

\item[32]

And now, the formation of the embryo.\footnote{For a parallel discussion in 
the \CS, compare \Ca{4.6.21}{334}.}

“The head comes into being first of all,” says Śaunaka, “because it is the root of 
it”.\footnote{I.e., the root of the fetus. The \dev{tan} in the compound 
\dev{tanmūlatva} “the root of it” could refer to the head, and that is indeed the 
reading of the vulgate text.  We take it as picking up the genitive 
\dev{garbhasya} at the start of this passage.  
    
 On the medical author (Bhadra)Śaunaka, 
see \volcite{IA}[150--152]{meul-hist}.  The Śaunaka who has an opinion about 
fetal formation appears in the \CS, here in the \SS, and in the 
\emph{Bhelasaṃhitā}.  His views in the \SS\ and the \emph{Bhelasaṃhitā} 
concur but differ from the view expressed in the \CS.  In the \CS, this view about 
the head is proposed by Kumāraśirā Bharadvāja (\Ca{4.6.21}{334}).}

“Amongst the chief organs of sense, the heart is first,” says
Kṛtavīrya, “because it is the location of the intellect and the
mind".\footnote{The phrase “amongst the chief organs of sense" could
    be read with the previous phrase about the primacy of the head.
    \MS{Kathmandu NAK 5-333} has a daṇḍa before the phrase, suggesting
    that it is part of Kṛtavīrya's view, but scribal practice gives this
    low significance. Ḍalhaṇa does not mention this phrase; \citet[353,
    note 3]{vulgate} recorded a variant reading \dev{dehendriyāṇām}
    “amongst the body and the organs of sense.”

On Kṛtavīrya, see \volcite{1A}[370--371]{meul-hist}; note that the view of the 
\SS\ is attributed to Kāṅkāyana the Bactrian in the \CS\ and to Parāśara in the 
\emph{Bhelasaṃhitā}.}

“It is the navel,” says Pārāśarya, “from that, the breath of the embodied person 
expands".\footnote{The reading of the Nepalese version, giving breath as the 
reason for Pārāśarya's view, is more coherent than the vulgate's version.
    
    On Pārāśarya, see \volcite{1A}[174 et passim]{meul-hist}.
    Once again, this person is associated with a different view in the 
    \emph{Bhelasaṃhitā}.  In the \CS, the navel argument is attributed to 
    Bhadrakāpya.}
    
“It is the hand and foot,” says Mārkaṇḍeya, “because
they are the root of its motion”.\footnote{On Mārkaṇḍeya, see 
\volcite{1A}[170]{meul-hist}, 1B: 267 et
passim. Mārkaṇḍeya, like Cyavana, is often an archetype of
longevity and is cited as such in the Bower manuscript
\citep[106--108]{hoer-bowe} and in the alchemical
\emph{Rasendramaṅgala} (\emph{Kakṣapuṭa} 71: \dev{kathayāmi na
sandeho mārkkaṇḍeyena yatkṛtam/ dīrghāyuḥkārakaṃ bhūme rasasiddhe
rasāyane/}).  But this archetype does not seem to be at work in
the present passage.  In the \CS, this view about the hands and feet is 
attributed to Baḍiśa.}


In this context, Subhūtigautama says, “it is the embryo's torso,”
because of the fact that all the limbs originate from a connection
with it.\footnote{On Subhūtigautama, see \volcite{1A}[158 et
    passim]{meul-hist}. His view is not represented in the \CS.}
    
    But this is not correct.  The limbs and smaller body parts appear
at the same time.  Because of the smallness of the embryo at that
moment in time they cannot be perceived.\footnote{Note that the
    vulgate attributes this final summary view to Dhanvantari 
    \pvolcite{1A}[247]{meul-hist}, while the Nepalese
    version does not. Ḍalhaṇa apparently did not have this attribution
    in the text before him, suggesting that it may have been added
    after the twelfth century.  However, in the \CS, this view is attributed to 
    Dhanvantari.  It seems likely that the name Dhanvantari was here added to 
    the \SS\ because of the passage in the \CS.
    
    As has been noted in another context, the phrase \dev{tat tu na samyak} 
    “But this is not correct,” can signal the inclusion of a passage from the \CS\ in 
    the vulgate text of the \SS \citep{wuja-2025}.  It is possible that the present 
    passage entered the \SS\ under the influence of the \CS\ before the ninth 
    century.}
        
        It is like the sprout of a bamboo or the fruit of a mango.
Just as in a ripe mango fruit, the fibres, flesh, stone and
its \se{majjan}{core} can be seen separately because of the
progression of time, so in the same way, those same things are
not perceptible at an \se{taruṇa}{early stage} because they
are so small.\footnote{The list of a mango's parts are
    explicitly paralleled by the parts of the body.} It is time
    that reveals these tiny things such as fibres. In this same
    way the sprout of the bamboo can be explained.
    
    Thus, although in the early stage of the embryo all the limbs and 
    smaller parts cannot be perceived even though they 
    are present, 
    with the progression of time they too become clearly manifest. 
    


    
\footnote{\Dalhana{3.3.32}{354} cited a passage from the author Bhoja
    at this point (on Bhoja, see p.\,\pageref{Bhoja}).}



% got to here


\bigskip


    
\end{translation}
