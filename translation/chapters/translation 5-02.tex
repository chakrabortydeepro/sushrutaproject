% !TeX root = incremental_SS_Translation.tex
\newcommand{\plant}[4]{#1 (\emph{#2})\footnoteA{#3; see #4}}
\let\chemical = \plant
\newcommand{\skt}[2]{#1 (\emph{#2})}
\newcommand{\sskt}[2]{\empty}
%

\chapter{Kalpasthāna 2: Poisonous Plants}

\section{Introduction}

This section begins with several lists of poisonous plants.  The
Sanskrit names for these plants are mostly not standard or familiar
from anywhere in Sanskrit or ethnobotanical literature.  It remains a
historical puzzle why these particular names are so difficult to
interpret. However, we are not the first to encounter these
difficulties. 

In the eleventh century, Cakrapāṇidatta commentated on 
a similar list of poisons in the \CS, and referred to the \SS\ on the 
topic.\footnote{\Cakra{6.23.11}{571}.}   He 
also noted that,
\begin{quote}
    In assigning the names to these plants, the main authorities are
the Kirātas and Śabaras, who know about these things because they
can explain these matters on the basis of a succession of
teachers.\footnote{Cakrapāṇidatta on \CS\ \Su{6.23.11}{571}.}
\end{quote}
\par
\noindent
About a century later, the learned commentator on the \SS, Ḍalhaṇa, 
remarked,
\begin{quote}
    \label{kiratas}
In spite of having made the greatest effort, it has been impossible to
identify these plants. In the Himalayan regions, Kirātas and Śabaras
are able to identify them.\footnote{After \SS, \emph{kalpasthāna} 2.5
    \citep[564]{vulgate}.}
\end{quote}
From the view of Sanskrit authors, Kirātas and Śabaras were tribal
peoples.\footnote{Both communities are mentioned in Sanskrit
    literature from antiquity.  The Kirātas are associated especially with
    Eastern Nepal, the Himalayan and north-eastern regions of South Asia,
    while the Śabara people are mainly associated with Odisha and West
    Bengal.  Representative studies on these communities include
    \citet{chat-1951,sing-2008,roy-1970,elwi-1955,subb-1999,rai-2019a,sing-1990}.}
     
In the tenth or eleventh century, the author Bhikṣu Govinda cast his
alchemical treatise as a dialogue with a Kirāta king called Madana who
was a master of the alchemical art.\footcite[IIA, 620]{meul-hist}  So
there was an awareness amongst Sanskrit medical and alchemical authors
of that period that different populations were a source of specialized
knowledge in these domains, and the Sanskrit authors were open to
these sources and indeed depended on them.

Ḍalhaṇa also recorded variant readings of these poison names from the
manuscripts that he consulted of the lost commentary of Gayadāsa
(fl.\,c.\,\AD\ 1000). The identities of these poisons have thus been
in doubt for at least a thousand years.\footnote{See
    \cite[80--81]{wuja-2003}.} Firm identification has in many cases been
    equally impossible for us today.

One path for exploration in this situation is to attempt to
reverse-engineer some identifications by considering the known toxic
plants of India.\footnote{Valuable reference sources on Indian plant
    toxicology in general include \cite[chs.\,10, 11]{pill-2013} and
    \cite[parts 1.II, 3 and 4]{barc-2008}. More generally \citet[41 et
    passim]{bown-2001} comments usefully of herbs in general that “it
    goes without saying that if they can do good, they must contain
    substances that in excess can poison.”  See for a general list of poisonous 
    plants, see \cite{wiki-2025a}.}

%\section{Manuscript notes}
\subsection{Shock}

An important new topic introduced in this chapter (34--39) is that of “toxic
shock” (\emph{vega}).\sse{vega}{toxic shock}  When a patient has been 
poisoned, the effect of the toxin is expressed in their body in seven waves or 
pulses, \emph{vegas}.  At each stage, symptoms are slightly different and a 
different therapeutic regime is prescribed (40--44).  

The Sanskrit term \emph{vega} has a range of uses, from “impulse” to
“urge, jerk, rush, speed,” or “impetus.”  It appears in the
well-known passage in the \CS\ about avoiding illness not ignoring 
or suppressing “natural urges,” \emph{vegas}, such
as the desire to urinate.\footnote{See \CS\ \Ca{1.7}{49--55}, discussed and
    translated in \cite[7--8, 15--17]{wuja-2003}.}
    
According to the author of the \AS, Ālambāyana\label{alambayana1} was
the ancient authority who declared that the seven \se{vega}{pulses}
of toxic shocks affect, successively, the seven
\se{āśraya}{substrata} of the body, from blood to semen, and
Dhanvantari originated the idea that this applied to victims of
snake-bite.\footnote{\AS\ \As{6.40.35}{844}: \dev{sapteti vegā
    mūrchādyā videhapatinā smṛtāḥ//34// raktamāṃsavasāsnāyu
    tathā'sthyādyāstrayaḥ kramāt/ āśrayāḥ sapta
    saptānāmityālambāyano'bravīt//35//}.  The following verse named
    Dhanvantari as the originator of the idea that toxic pulses are
    experienced specifically by a person bitten by a snake 
    (\dev{vegāndhanvantaristadvatsarpadaṣṭasya manyate/} 36ab).  The
    commentator Indu noted that Dhanvantari was the teacher of Suśruta, i.e., 
    that “Dhanvantari" was shorthand for \SS.    
    On Ālambāyana, see p.\,\pageref{alambayana2}, note \ref{alambayana2}.}
    
The commentator Indu (fl.\,1000--1150) cited verses by Ālambāyana
asserting that the pipes in the body carry poison to the heart, but
that the heart can be protected by ghee. \footnote{\AS\ \As{6.40.60}{}:
    \dev{yāḥ sirāḥ sarvagātreṣu hṛdaye sampratiṣṭhitāḥ| tābhirasya viṣaṃ
    sarvaṃ hṛdayaṃ sampradhāvati// ghṛtena tu praticchannaṃ viṣaṃ nāti
    prapīḍayet/ nirvāṇajananaṃ sarpiḥ prāṇināṃ prāṇavarddhanam//
    hṛdayāvaraṇāstadvadbhakṣyā bhojyāśca sāgadāḥ//}}


\subsection{Literature}

Meulenbeld offered an annotated overview of this chapter and a bibliography
of earlier scholarship to 2002.\fvolcite{IA}[290--291]{meul-hist} 

\newpage

\section{Translation}

\begin{translation}
    
\item[1]

And now I shall explain \diff{\se{vijñānīya}{required knowledge}}
about stationary poisons.\footnote{No reference is made to
    Dhanvantari \citep[see][]{birc-2021}. “Stationary” here is a term
    contrasted with “moving,” and signifies plants as opposed to animals
    and insects.}
  
    \item[3]
    \noindent It is said that there are two kinds of poisons,
    \se{sthāvara}{stationary} and \se{jaṅgama}{mobile}. The former
    dwells in ten sites, the latter in sixteen places.
   
    \item[4]
    Traditionally, the ten are: root, leaf, fruit, flower, bark,
    \se{kṣīra}{milky sap}, \se{sāra}{pith}, \se{niryāsa}{resin}, 
    \sepl{dhātu}{mineral}, and the tuber.

    \item[5]
    
    In that context,\label{poisonousplants}
    \begin{itemize}
        \item[A] The eight items with poisonous roots are:\footnote{Some South 
        Asian
    plants with poisonous roots that we would expect to see in
    this list include \emph{Croton tiglium}, L., \emph{Calotropis}
    spp. (\egls{arka}, etc.), \emph{Citrullus colocynthus} L. Schrad. 
    (\egls{indravāruṇī}), and
    \emph{Ricinus communis} L. (\egls{eraṇḍa}), \citep{pill-2010}.} %
    % \q{Expected \citep{pill-2010}:\\ Croton
    %        tiglium, L. = Naepala, Jayapala, kanakaphala,
    % titteriphala
    % (NL \#720);
    %        Calotropis spp.;\\ Citrullus colocynthus (colocynth);\\
    % Ricinus communis
    %        (castor); }
        \begin{enumerate}
            
        \item  \gls{klītaka},\footnote{Liquorice eaten in excess can
    be poisonous, but it is uncertain whether it is the plant intended
    here.  \citet[124]{gvdb} specifically noted that the poisonous root
    mentioned in this passage, “remains to be identified.” Cf.\ glossary for 
    discussion.}
       
        \item \gls{aśvamāraka},
    
        \item \gls{guñjā},
        
        \item \diff{\gls{subhaṅgurā}},\footnote{The vulgate reads 
      \egls{sugandhā}, which can be poisonous.}
        
       \item \diff{\gls{karaṭā}},\footnote{Conjectural identification with 
       \emph{karahāṭa}; similar-sounding candidates also include 
       \egls{karkaṭaka} and \egls{karaghāṭa}, but since this is a prose passage, 
       there would be no reason to alter the word to fit a metre.} and ending with 
           
\item \gls{vidyutśikhā},

\item \diff{\gls{ananta-poison}},\footnote{\label{ananta-poison}The
    text reads masculine \emph{ananta}, which is not a plant name.  Gayī's
    commentary on \Su{5.2.5}{564} noted a variant reading of feminine
    \emph{anantā} in place of \emph{gargaraka}, earlier in the compound.
    But the feminine \egls{anantā} is not a poisonous
    plant.} and

% got to here 

\item \gls{vijayā-poison},\footnote{\citet[61, n.\,3]{meul-sear} argued that
    our text reads a masculine or neuter noun \emph{vijaya}, which never
    signifies cannabis. However, unlike the vulgate, the unanimous
    readings of the Nepalese manuscripts give feminine \emph{vijayā}. 
    Nevertheless, even the feminine form only started to signify
    \emph{Cannabis sativa} L. after the end of the first millennium
    \citep{meul-sear,wuja-cann,mchu-2021}. The \emph{Sauśrutanighaṇṭu}
    gives a number of synonyms for \emph{vijayā}, almost none of which
    have any poisonous parts \citep[5.77, 10.143]{suve-2000}.  But one of
    them, \emph{viṣāṇī} (also \emph{meṣaśṛṅgī}), is sometimes equated with
    \emph{Dolichandrone falcata (DC.) Seemann} \citep[518]{adps}, a plant
    used as an abortifacient and fish poison \citep[\#862]{NK}.  This
    identification is tenuous.} %
    %        \footnote{Large doses of the root-extract of rauwolfia
    % can be fatal.
    %
    %        In large doses luffa is emetic and a drastic purgative. }
        \end{enumerate}
        \end{itemize}
    
    
    % 5.2.5B
    
        \item[B]
        the leaf-poisons include:
             \begin{itemize}            
        \item \gls{viṣapatrikā},
        \item \diff{\gls{lambaradā}},
%        \item \plant{`choice tree'}{varadāru}{unknown}{?},
        \item \gls{karambha},
        and
        \item big \gls{karambha};
            \end{itemize}
% 5.2.5C
        \item[C]
        the fruits of items like:
        \gls{guñjā},
        \gls{aruṣkara},
        and
        \gls{viṣavedikā}
        are
\begin{itemize}
         \item \diff{\gls{kumudavatī}},	
        \item \diff{\gls{reṇukā}},
    \item \diff{\gls{kuruvaka}},
    \item \diff{\gls{veṇuka}},
    \item \gls{karambha}
    \item \gls{mahākarambha}
    \item \diff{\gls{nandanā}},
    \item \diff{\gls{kāka2}},
\end{itemize}
   
 % 5.2.5D 
        \item[D]
        the flower-poisons include those of:
\begin{itemize}
 \item \diff{\gls{ullika}}, 
 \item \diff{\gls{reṇu}},\footnote{\dev{reṇu} and \dev{reṇuka/kā} are 
 different plants. MS K reads the first; the scribe of MS H added an additional 
 \dev{-ka} in the margin.}
%        \item \gls{vetra},
%        \item \gls{kādamba},        
%        \item \gls{vallīja},
        \item \gls{karambha},
        and
        \item \gls{mahākarambha}.
\end{itemize}

% 5.2.5E        
        \item[E]
        the bark, \se{sāra}{pith} and \se{niryāsa}{resin} of:
\begin{itemize}
        \item \diff{\gls{vallija}},
%        \item \gls{kartarīya},
%        \item \gls{saurīyaka},
        \item \gls{karaghāṭaka},
        \item \gls{karambha},
%        \item \gls{nandanā},
        and
        \item \gls{nārācaka};
            \end{itemize}
 
 % 5.2.5F
        \item[F]
        the  \se{kṣīra}{milky sap} of:
              \begin{itemize}
            % got to here.  2024-09-27
        \item \gls{kumudavati},\footnote{While the identity of this plant is 
            uncertain, the Nepalese version of the \SS\ does not present the 
            hopeless problem of the vulgate's reading \dev{kumudaghnī}.}
%        \plant{purple calotropis}{kumudaghnī $\rightarrow$ arka?}{Calotropis
%            gigantea, (L.) R. Br.}{ADPS 52, AVS 1.341, NK \#427, Potter
%            63},\footnote{The name of this poison, \emph{kumuda-ghnī}, means 
%            `lotus
%        killer'.  In Sanskrit literature, the \emph{kumuda} lotus is associated
%        with the moon, since it blossoms by night.  Since the sun causes this 
%lotus
%        to close, it is therefore an `enemy' of the lotus.  One of the chief words
%        for the sun, \emph{arka}, is also the name of \emph{Calotropis 
%gigantea},
%        which indeed has a milky juice which is a violent purgative, poison and
%        abortifacient.}
%
\item \gls{dantī},
\item \gls{snuhā},
%        \item \plant{oleander spurge}{snuhī}{Euphorbia neriifolia, L., 
%        \textnormal{or}
%            E. antiquorum, L.}{ADPS 448, AVS (2.388), 3.1, NK
%            \#988, IGP 457b},
        %   \marginpar{`The milky juice or gum which flows from the branches
        %     [of \emph{E. antiquorum}] is an acrid irritant\ldots. Internally it is a
        %     powerful emetic and a violent purgative, even in very small quantities'.
        %     --- NK \#982}
and
\item \gls{jālinī}        
%        and 
%        \item \plant{`web-milk'}{jālakṣīri}{unknown}{?};
            \end{itemize}

% 5.2.5G        
        \item[G]
        the  \se{dhātu}{mineral} poisons include:\footnote{These identifications
            are more than usually uncertain.  Note that the vulgate
            text specifies that there are two mineral poisons.}
              \begin{itemize}
\item \gls{haritāla},
\item \gls{phenāśma}, 
\item \gls{bhasma}, and
\item \gls{rakta}.\footnote{If this identification as \egls{rakta} (cinnabar) is 
correct, it is an unexpectedly early mention of the substance.}        
%        \item \plant{`foam-stone'}{phenāśma}{unknown}{?}, and
%        \item \plant{orpiment}{haritāla}{Arsenii trisulphidum}{NK v.\,2,
%            p.\,20\,ff.};\footnote{\citet[38--42]{dutt-1922} conjectured that
%        `foam-stone' may be impure white arsenic obtained by roasting 
%orpiment.}
            \end{itemize}

% 5.2.5H        

% got to here.
        \item[H]
        the tubers poisons are:
        \begin{itemize}
             \item \gls{kālakūṭā},
%        The \emph{Rāja\-nighaṇṭu\-pariśiṣṭa} (9.35) gives \emph{kālakūṭaka} 
% as a synonym for \emph{kāras\-kara}, or \emph{Strychnos nux-vomica}, L., 
%        whose seeds are notoriously poisonous.}
\item \gls{vatsanābha},
%        \item \plant{wolfsbane}{vatsanābha}{Aconitum napellus, L.}{AVS 1.47,
%            NK \#42, Potter 4\,f.},
\item \gls{sarṣapa},
%        \item \plant{Indian mustard}{sarṣapa}{Brassica juncea, Czern. \&
%            Coss.}{AVS 1.301, NK \#378},
\item \gls{pālaka},
%        \item \plant{leadwort}{pālaka $\rightarrow$  citraka}{Plumbago 
%zeylanica
%            (indica? rosea?), L.}{Rā. 6.124, ADPS 119, NK \#1966,
%            1967},
\item \gls{kardama},
%        \item \plant{`muddy'}{kardama}{unknown}{?}, the
\item \gls{vairāṭaka},
%        \item \plant{`Virāṭa's plant'}{vairāṭaka}{unknown}{?},
\item \gls{mustaka},
%        \item \plant{nutgrass}{mustaka}{Cyperus rotundus, L.}{ADPS 316,
%            AVS 2.296, NK \#782},
\item \gls{śṛṅgīviṣa},
%        \item \plant{atis root}{śṛṅgīviṣa}{Aconitum heterophyllum, Wall.
%            ex Royle}{AVS 1.42, NK \#39},
        % \item \plant{liquorice}{prapuṇḍarīka $\rightarrow$ 
        %madhuka?}{Glycyrrhiza
        %  glabra, L.}{AVS 3.84, NK \#1136},\footnote{Non-toxic.}
\item \gls{prapauṇḍarīka},
%        \item \plant{sacred lotus}{prapuṇḍarīka}{Nelumbo nucifera, 
%        Gaertn.}{Dutt 
%        110, 
%        NK
%            \#1698}, 
\item \gls{mūlaka},
%\item \plant{radish}{mūlaka}{Raphanus sativus, L.}{NK 
%            \#2098},
\item \gls{hālāhala},
%        \item \plant{`alas, alas'}{hālāhala}{unknown}{Cf. Soḍhalanighantu 
%p.43 
%        (sub
%            bola) = stomaka = vatsanābha}, 
\item \gls{mahāviṣa},
%\item \plant{`big poison'}{mahāviṣa}{unknown}{?}, 
and 
\item \gls{karkaṭa}
%            \plant{galls}{karkaṭa}{Rhus
%            succedanea, L.}{NK \#2136}.\footnote{Leadwort root is a powerful 
%poison.
%        Nutgrass is tuberous, but non-toxic. Atis has highly toxic tuberous
%        roots. Neither sacred lotus nor galls are toxic. The `alas, alas' poison
%        (\emph{hālāhala}) is the mythical poison produced from the churning of
%        the ocean at the time of creation: it occurs in medical texts such as
%        the present one, and commentators identify it with one or other of the
%        lethal poisons such as wolfsbane or jequirity.
%        \citet[126]{agra-indi} makes the intriguing suggestion
%        that the word \emph{hālāhala},
%        possibly to be identified with Pāṇini's \emph{hailihila} (P.6.2.38),
%        may be of Semitic origin, although his evidence
%        seems uncertain (\citet[1506a]{stei-pers} cites Persian \emph{halāhil}
	%        `deadly (poison)' as a loan from Sanskrit). 
	% \volcite{iii}[585]{KEWA}
%        also cites a claim for an Austro-Asiatic origin for the word.}
            \end{itemize}


    
\subsection{The effects of poisons}

\subsubsection{Symptoms of root poisoning}
    \item[7--10]
    
People should know that root-poisons cause \se{udveṣṭana}{writhing},
\se{pralāpa}{ranting}, and \se{moha}{delirium}, and  leaf-poisons
cause yawning, writhing, and \se{śvāsa}{wheezing}.
    
 Fruit-poisons cause swelling of the
   scrotum, a burning feeling and writhing.  Flower-poisons will
    cause vomiting, \se{ādhmāna}{distension} and \se{svāpa}{sleep}.  
    
The consumption of poisons from bark, \se{sāra}{pith} and
\se{niryāsa}{resin} will cause foul breath, \se{pāruṣya}{hoarseness},
a headache, and a discharge of \se{kapha}{phlegm}.\footnote{At
    \Su{1.2.6 }{11}, Ḍalhaṇa glossed \se{pāruṣya}{hoarseness} as
    \emph{vāgrūkṣatā}, “a rough, dry voice.”}
    
    % 10
    
The \se{kṣīra}{milky sap}-poisons make one froth at the mouth,  cause
loose stool, and make the tongue feel heavy.\footnote{At
    \Su{6.54.10}{773}, Ḍalhaṇa glossed \se{viḍbheda}{loose stool} as
    \emph{dravapurīṣatā}, “having liquid stool.” }  The
    \se{dhātu}{element}-poisons give one a crushing pain in the chest,
    make one faint and cause a burning feeling on the palate.
    
    % 11
    These poisons
    are classified as ones which are generally speaking lethal after a period of time.
    
    \item[11--17]
    
    \subsubsection{Symptoms of tuber poisoning} The tuber-poisons,
though, are severe.  I shall talk about them in
detail.\footnote{See Ḍalhaṇa's comments on the impossibility of
    identifying the following plants, p.\,\pageref{kiratas} above. 
    All the following plant identifications are tentative in the
    extreme; see the glossary for discussion.}
    
    %12
    
    With
    \gls{kālakūṭa},
%    {kālakūṭa}{Abrus precatorius, L.?
%        Cf.\ RRS 21.14.}{AVS 1.10, NK \#6, Potter 168.},
 there is numbness and very severe trembling.

%
    With
    \gls{vatsanābha},
%    {Aconitum napellus, L.}{AVS 1.47,
%        NK \#38, Potter 4\,f.}, 
there is rigidity of the neck, and the faeces,
    and urine become yellow.
    
    %13
With \gls{sārṣapa}, 
%With \plant{Indian mustard
% roots}{sārṣapa}{Brassica juncāea, Czern \&
%    Coss.}{AVS 1.301, NK \#378}
the \skt{wind becomes defective}{vātavaiguṇya}, there is
\se{ānāha}{constipation}, and \se{granthi}{lumps} start to appear.

With \gls{pālaka}, % $\rightarrow$  citraka}{Plumbago zeylanica
%    (indica? rosea?), L.}{Rā. 6.124, ADPS 119, NK \#1966, 1967},
there is weakness in the neck, and speech gets jumbled.\footnote{The
    verse in the Nepalese version ends with a plural verb that does not
    agree with the dual of the sentence subject.}
    
    %14
With the one called \gls{kardama}, there is a
\se{praseka}{discharge}, the faeces pour out, and  the eyes turn
yellow. 

The \egls{vairāṭaka} causes pain in
the body and illness in the head. 

Paralysis of one's arms and legs and trembling are said to be
caused by \gls{mustaka}.%
%    \plant{nutgrass}{mustaka}{Cyperus rotundus, L.}{ADPS 316, AVS
% 2.296,
%        NK \#782} %
\footnote{The substitution in \MScite{NAK 5-333} affecting 15cd is
    caused by an eye-skip to the word \emph{viṣeṇa} in 2.17. 
    
    \emph{Mustaka} commonly refers to Cyperus rotundus, L.; the root is
    used in āyurveda but is not poisonous.  However other dictionaries
    list \emph{mustaka} amongst serious poisons, for example
    \emph{Rājanighaṇṭu} (22 v.\,42) and \emph{Rasaratnasamuccaya} 16,
    v.\,80.  However, its ancient identity is still doubtful.} 

\item[15b] 
% got to here
With \gls{mahāviṣa}, one's limbs grow weak, there is a burning feeling
and swelling of the belly.\footnote{The poisonous root \egls{mahāviṣa}
    is not clearly identifiable, although \emph{viṣā} is commonly aconite.
    Verse 6 above notes that there are several kinds of aconite.}
        
\item[16a] With \gls{puṇḍarīka},  
        %   \plant{sacred lotus}{puṇḍarīka}{Nelumbo nucifera,
        % Gaertn.}{Dutt 110,
        %        NK \#1698},
        one's eyes go red, and one's belly becomes
        distended.\footnote{The word \emph{puṇḍarīka} very commonly means
            white lotus. The entire plant is
            edible and cannot be the poison intended here. \citet[252]{gvdb}
            noted that this poison is unidentified and that it is also listed
            as a poison in \Cs{ci.23.12}{}.}\q{Look up the ca. reference.}

\item[16b] With \gls{mūlaka},    
            % \plant{radish}{mūlaka}{Raphanus sativus, L.}{NK \#2098}es,
            one's body is drained of colour and the limbs are
            paralysed.\footnote{The word \emph{mūlaka} very commonly means
                the radish, \emph{Raphanus sativus}, L. The root is edible and
                cannot be the poison intended here. \citet[317]{gvdb} noted that
                this poison is unidentified.}
    
    %17
    \item[17a]
        
    With \gls{hālāhala}, a man turns a \se{dhyāma}{dark colour}, and
gasps.\footnote{Identification of \emph{hālāhala} is  uncertain. It may simply
be a mythical poison, or its specific identity may have been lost over the
centuries. Late \emph{nighaṇṭu}s identify it as \emph{stomaka} =
\emph{vatsanābha}, i.e., \emph{Aconitum napellus}, L. 
(\emph{Soḍhalanighaṇṭu} p.\,43). 

Ḍalhaṇa on \Su{5.2.17}{564} interpreted our “gasps” as “the man laughs 
and grinds his teeth.”  But this gloss is probably displaced and intended to 
apply to verse 2.18.}

% 5.221 Rājanighaṇṭu

\item[ 17b] 

With \gls{śṛṅgīviṣa}
%{Aconitum
%    heterophyllum, Wall.\ ex Royle}{AVS 1.42, NK \#39}, 
one gets violent
\se{granthi}{knots} and stabbing pains in the 
heart.\footnote{\citet[407]{gvdb} noted that \emph{vatsanābha} and 
\emph{śṛṅgīviṣa} are two different varieties of poisonous Aconites that are 
difficult to distinguish.}
    
    %18
    \item[ 18a]
    With
    \gls{markaṭa}, one leaps up, laughs, and 
    bites.
    
    %{galls}{karkaṭa}{Rhus succedanea, L.}{NK \#2136}
    
    \item[ 18b-19a]
%    Experts said that the thirteen cited highly potent tuber-poisons should be 
%known to have possessed ten features:
%    %()Experts said that one should know that these thirteen cited highly potent 
%%tuber-poisons have ten features:
    %Experts said that these thirteen highly potent tuber-poisons which are 
    %mentioned here consist of ten features.)
      
    There are thirteen tuber-poisons that are said to be fiercely
potent.  These ones that have been stated are connected with ten
\diff{positive} qualities.\footnote{This verse reads
    differently, and scans poorly, in the vulgate. The vulgate's
    \dev{pratyuktāni} “are contradicted” is awkwardly explained by
    Ḍalhaṇa as “are stated individually” (\Dalhana{5.2.18cd}{535}). “Positive” 
    translates \dev{kuśalāni}, which is not present in the vulgate.}
    
    \item[19cd--20ab]
    
    The ten are, traditionally:
    \begin{itemize}
        \item   dry, %\se{rūkṣa}{dry}, 
        \item hot, 
        \item sharp, 
        \item rarefied, %\se{sūkṣma}{rarefied},
        \item     fast-acting, 
        \item pervasive, %\se{vyavāyin}{pervasive}, 
        \item expansive, %\se{vikāsin}{expansive}, 
        \item limpid, %\se{viśada}{limpid},
        \item     light, and 
        \item indigestible.    
    \end{itemize}
    %20b
    \item[ 20b]
    Because of dryness, it may cause inflammation of the wind; because of heat
    it inflames the choler and blood. 
    %21
    Because of the sharpness it unhinges the
    mind, and it cuts through the connections with the \skt{sensitive
        points}{marman}.  Because it is rarified it can infiltrate and distort
    the parts of the body.\footnote{We read the active \emph{vikaroti} with 
    Ḍalhaṇa against the 
    transmitted passive \emph{vikriyeta}, since it must be the parts of the body 
    that are distorted, not the poison.}    
    

\item[22]
Because it is fast-acting it kills quickly, and because of its pervasiveness
it affects one's \skt{whole physical constitution}{prakṛti}.\footnote{Ḍalhaṇa
on \Su{5.2.22}{565} explained this as “\se{akhiladehavyāptirūpam}{takes the
form of pervading the whole body}.”}  Because of its expansiveness it enters
into the \se{doṣa}{humour}s, \se{dhātu}{bodily constiuents}s, and even the
impurities\sskt{impurity}{mala}.  Because it is limpid it overflows, and
because it is light it is difficult to treat.  Because it is indigestible it
is hard to eliminate.  Therefore, it causes suffering for a long time.
    
    \item[ 24]
    Any poison that is instantly lethal, whether it be
    stationary, mobile, or artificial, will be known to 
    have all ten of these qualities.
    
    
  
    
    \subsection{Slow-acting poison}
    \item[25cd--26]  
    \begin{verse}
        A poison that is old or destroyed by anti-toxic medicines, or
else dried up by blazing fire, wind, or sunshine, or which has
just spontaneously lost its features,\footnote{Ḍalhaṇa
    specified that this refers to the ten qualities that are
    mentioned above (\Su{5.2.26}{565}).} becomes a
    \skt{slow-acting poison}{dūṣīviṣa}.\footnote{Ḍalhaṇa cited
        this verse at \Su{1.46.83}{222} while explaining
        \emph{dūṣīviṣa} (see p.\,\pageref{dusivisa}.} Because it has
        lost its potency it is no longer perceived.  Because it is
        surrounded by \se{kapha}{phlegm} it has an aftermath that
        lasts for a very long time.
        
        \item[27] 
        
If he is suffering from this, the colour of his stools changes, he
gets a sour, bad taste and is very thirsty. Speaking nonsensically and
close to death, wandering about, he may feel faint, giddy, and
aroused.\footnote{Similar symptoms of slow-acting poison are described
    at \Su{2.7.11--13}{296} in the context of 
    \se{duṣyodara}{contamination dropsy}.  This this may explain why the
    vulgate inserted reference to this disease at this point.}
        
        
        
%        Also, he has
%        the symptoms of \skt{contaminated
%            dropsy}{duṣyodara}.
%        \footnote{\label{dusyodara}`Contaminated dropsy'
%        (\emph{duṣyodara} or \emph{dūṣyudara}) is described elsewhere as a
%        condition which arises when women of ill-character mix nail clippings,
%        hair, urine, faeces, or menstrual blood with a man's food, in order to
%        gain power over him (2.7.11--13).}



        \item[28]
        If it lodges in his \se{āmāśaya}{stomach}, he becomes sick because of wind 
        and phlegm; if it lodges in his \se{pakvāśaya}{intestines}, he becomes sick 
        because of  wind and 
        choler.  A man's hair and limbs fall away and he looks like a
        bird whose wings have been chopped off.
        \item[29a--c]
        If it lodges in one of the body tissues such as 
        \se{rasa}{chyle}, it causes the diseases arising
        from the body tissues, that have been said to be wrong.\footnote{The 
        expression \emph{ayathāyathoktān} “stated to be unsuitable” is hard to 
        understand here, but is clearly transmitted in the Nepalese version.}
        and it rapidly becomes inflamed on days that are nasty
        because of cold and wind.
        
        \item[29d--31] Listen to its initial \se{liṅga}{symptoms}: it causes
heaviness due to sleep, yawning, \se{viśleṣa}{disjunction} and
\se{harṣa}{horripilation} and a \se{aṅgamarda}{bruising of the
    limbs}.\footnote{Ḍalhaṇa \Su{5.2.30ab}{565} glossed “disjunction” as the
loss of function of the joints in regard to movement.} Next, it causes
\se{annamada}{intoxication from food} and indigestion, \se{arocaka}{loss
    of appetite}, the condition of having a \se{koṭha}{skin disease} with
\se{maṇḍala}{round blotches},\footnote{The last ailment could perhaps be
ringworm.} % 5.2.31
\diff{\se{kṣaya}{dwindling away} of flesh}, swelling of the feet, hands, and
face, \diff{the fever called \textit{pralepaka}}, vomiting and
diarrhoea.\footnote{The \emph{pralepaka} fever was described by Ḍalhaṇa,
at \Su{6.39.52}{675}, as an accumulation of phlegm in the joints.  Its
symptoms are described in 6.39.54} The slow-acting poison might cause
\diff{wheezing, thirst and fever, and it might also cause distension of the
abdomen.}
        
        %Perhaps his colour may drain away and he may faint or have \se{viṣamajvara}{irregular fever}.  It may cause heightened,
        %powerful thirst.
        
        \item[32]
 
            These various disorders are of many different types: one poison may 
            produce
            madness, while another one may cause \se{ānāha}{constipation}, and 
            yet
            another may ruin the semen. One may cause \diff{emaciation}, while 
            another
            \se{kuṣṭha}{pallid skin disease}.
 
    \end{verse}

    
    \item[33]  
    
Something is “corrupted” by repetitively keeping to bad locations,
times, foods, and sleeping in the daytime.  Or, traditionally,
“corrupting poison” (\se{dūṣī-viṣa}{slow-acting poison}) is so called
because it may corrupt (\emph{dūṣayet}) the \se{dhātu}{body tissue}s.
    
    
    \item[34-]
    
    \subsubsection{The stages of toxic shock}
    \label{stagesofshock}

    In the first shock of having taken a stationary poison, a person's tongue 
    becomes dark brown and stiff, he grows faint, and panics.
    
    % FROM HERE Harṣal 35-38
    \item[35]
    
    In the second, he trembles, feels exhausted, has a burning
feeling, as well as a sore throat.  When the poison reaches the
\se{āmāśaya}{stomach}, it causes pain in the \se{hṛd}{chest}.
    
    
    
    \item[36]
    In the third, his palate goes dry, he gets violent \se{śūla}{pain} in the 
    \se{āmāśaya}{stomach}, and his eyes become weak, swollen and yellow.

    \item[37]
    In the fourth shock, it causes the intestines and stomach to
    \se{sāda}{be exhausted}, he gets hiccups, a cough,  a rumbling in the
    \se{antra}{gut}, and his head becomes heavy too.
    
     \item[38]
    In the fifth he dribbles \se{kapha}{phlegm}, goes a bad colour,
    his \diff{\se{parśvabheda}{ribs crack}},  all his humours are irritated, and he
    also has a pain in his \se{pakvādhāna}{intestines}.
   
   
    \item[39a]
    In the sixth, he loses consciousness and he completely loses
    control of his bowels.
    
    \item[39b]
    In the seventh, there are breaks in his shoulders, back and loins, and he  
stops breathing.\footnote{%
%In \Su{1.15.24}{72}, Ḍalhaṇa glossed 
%\emph{kriyā-sannirodha} 
%as “cessation of the activities of the body, speech and mind” 
%(\emph{kriyāṇāṃ kāyavāṅmānasīnāṃ sannirodhaḥ}), while 
Here at \Su{5.2.24}{566} Ḍalhaṇa glossed \emph{sannirodha} as
“complete cessation, i.e., of breath” (\emph{sannirodhaḥ 
samyaṅnirodhaḥ, ucchvāsasya iti śeṣaḥ}).
The manuscripts all read \emph{skanda} where \emph{skandha} must be 
intended; this confusion is known from Buddhist Hybrid Sanskrit 
\pvolcite{2}[608]{edge-1953}.}
    
    % next  40-44
    
    
\subsubsection{Remedies for the stages of slow poisoning}
  \label{dusivisa}
  
    \item[40] In the first shock of the poison, the physician should make the man,
who has vomited and been sprinkled with cold water, drink an
\se{agada}{antidote} mixed with with honey and ghee.
    
    \item[41a] 
    
In the second, he should make the man who has vomited and been
purged drink as before;
    
    \item[41b]
    on the third, drink an antidote and a beneficial
    \se{nasya}{nasal medicine} as well as an \se{añjana}{eye salve}.
    
    
    \item[42a] In the fourth, the physician should make him drink an antidote that
is salt with a little oil.\footnote{At \Su{6.52.30}{769} Ḍalhaṇa noted that
\emph{sindhu} can be interpreted as \se{saindhava}{salt}.}
    
    

    
    \item[42b]
    In the fifth, he should be prescribed the antidote together with a
    \se{kvātha}{decoction} of honey and
    \gls{madhuka}.
   
   
    \item[43] 
    
    \diff{In the sixth, the \se{siddhi}{cure} is the same as
    for diarrhoea. And in the seventh, he perishes}.\footnote{The
    vulgate text here is quite different, recommending that the
    patient have medicated powder blown up his nose. It may be
    possible to detect the evolution of the Nepalese \dev{avasīdet} to
    the vulgate's \dev{avapīḍaś}.  The vulgate version is hard to
    construe, and we see Ḍalhaṇa struggling to interpret it in his
    commentary on \Su{5.2.43ab}{566}.  This sternutatory is, however,
    recommended in the Nepalese version at \Su{5.5.30ab}{576}, for the
    seventh shock of poisoning by a \se{rājimat}{striped snake}.  It
    is possible the text migrated from  that location to this.

\label{kakapada} Another difference at this point is that the Nepalese
version also does not support the vulgate's passage on the
\se{kākapada}{crow's foot} therapy \citep[145, n.\,106]{wuja-2003}.  The
same is the case at \Su{5.5.24}{575} and the clear description at
\Su{5.5.45}{577}, in neither of which is the therapy supported in the
Nepalese version.  This therapy seems unknown to the Nepalese
transmission.  The therapy may have migrated into the vulgate \SS\ from
the \CS\ \Ca{6.23.66--67}{574}.}
    
    % he should have medicated powder blown up his nose, 

%    and
%    after having a `\se{kākapada}{crow's foot}' cut made on his head, he
%    should have a piece of bloody meat put on
%    it.\footnote{\label{su:kakapada}Suśruta explains the term \emph{avapīḍa}
%    `medicated nasal powder' as the procedure either of administering
%    \se{avapīḍa}{nasal drops}, or blowing medicated powder into the nose
%    (4.40.44--46): it is particularly recommended for unconscious or incapable
%    patients.  The `crow's-foot' procedure is also recommended later in the
%    `Section on Procedures' (5.5.24a) in cases of snake-bite. It is also
%    described by Caraka (see p.\,\pageref{sa:kakapada} below).}
%

\item[44] 

\diff{In between any one of these shocks}, once the above treatment
has been done, he should give the patient the following cold
\se{yavāgū}{gruel} together with ghee and honey, that will take away
the poison.
       
    \item[45--46]
    
A \se{yavāgū}{gruel} made of the following items in a
\se{niḥkvātha}{stewed juice} destroys the two poisons:
\gls{kośavatī},\footnote{At \Su{4.10.8}{449} Ḍalhaṇa glossed
    \dev{kośavatī} as \dev{devadālī} and at \Su{4.18.20}{472} as
    \dev{kaṭukośātakī}, vocabulary pointing to \emph{Cucumis cylindrica},
    \emph{Cucumis actangula} or \emph{Luffa echinata}. See glossary under
    \gls{koṣītakī}.} \gls{agnika},\footnote{A plant often cited in \SS,
        but rarely in \CS\ \citep[4]{gvdb}.  Ḍalhaṇa glossed it here,
        \Su{5.2.45}{566}, as \emph{ajamodā}, \gls{ajamodā},  but noted that
        others consider it to be \emph{moraṭa}, \gls{moraṭa}. There is
        considerable complexity surrounding the identification of
        \emph{moraṭa/mūrvā} and related synonyms \citep[314-316]{gvdb}.
        Taking \emph{agnika} as a short reference to \emph{agnimantha}, often
        identified  as \gls{agnimantha}, might be plausible, since that is
        antitoxic or anti-inflammatory, but such a short reference is not
        known elsewhere.} \gls{pāṭhā}, \gls{sūryavallī},\footnote{At
            \Su{5.2.45}{566} Ḍalhaṇa said that this plant has leaves like the
            \emph{paṭola}, \gls{paṭola}, \citet[280, 443]{gvdb} argued plausibly
            that this is a synonym for \emph{arkapuṣpī}, \gls{arkapuṣpī}, as
            Ḍalhaṇa also stated in \Su{1.45.120}{206}, and the leaves of
            Holostemma and Trichosanthes are indeed strikingly similar.  The
            appearance of the plant, a creeper with sun-like flowers, fits the
            name.  But there remains much controversy about the identities of
            these candidates \citep[e.g.,][195--198]{adps}.} %
            %
            %    $\rightarrow$ jīvantī?}{Holostemma ada-kodien,
            % Schultes}{ADPS 195,
            %AVS
            %    3.167, NK \#1242, IMP 3.1619},
            \gls{amṛtā}, \gls{abhayā} \gls{śirīṣa}, and \gls{śelu},
            \gls{kiṇihī}, \diff{the two kinds of
                \gls{haridrā}},\footnote{I.e., \gls{haridrā} and
                \gls{dāruharidrā}.} and the two kinds of
                \gls{bṛhatī},\footnote{I.e., \gls{bṛhatī} and
                    \gls{kṣudrā}.} \gls{punarnavā}, \gls{hareṇu}, \diff{the
                        \gls{tryūṣaṇa}}, %
                    the two kinds of \gls{sārivā}\footnote{I.e.,
                        \gls{anantā} and \gls{pālindī}.} %
                        \diff{and \gls{utpala}}.
  \end{translation}


\subsubsection{The Invincible Ghee}
    
  \begin{translation}
    
    \item[47--49]
    
    \label{ajeya} There is a famous ghee called “Invincible”\sse{ajeya}{invincible}. 
    It rapidly destroys all poisons but is itself unconquered. It is prepared with a
\se{kalka}{mash} of the following plants: %
\gls{madhuka},
\gls{tagara},
\gls{kuṣṭha},
\gls{bhadradāru},
\gls{hareṇu},
\diff{\gls{mañjiṣṭhā}},
\gls{elā}
and \gls{elavālu},
\gls{nāgapuṣpa},
\gls{utpala},
\gls{sitā},
\gls{viḍaṅga},
\gls{candana},
\gls{patra},
\gls{priyaṅgu},
\gls{dhyāmaka},
the two turmerics,\footnote{I.e., \gls{rajanī} and \gls{dāruharidrā}.}
the two Indian nightshades,\footnote{I.e., \gls{bṛhatī} and \gls{kṣudrā}.}
the two kinds of \gls{sārivā},\footnote{I.e., \gls{anantā} and \gls{pālindī}.}
\gls{aṃśumatī},
and 
\diff{\gls{balā}}.
    
    \end{translation}

\subsubsection{Curing the `slow-acting' poison}
    

\begin{translation}    
    
    \item[ 50--52]
    
Someone suffering from “\se{dūṣīviṣa}{slow-acting poison}” should be
well sweated, and purged both top and bottom.  Then he should be made
to drink the following \diff{eminent} antidote which removes
“slow-acting poison:”
    
    Take
    \gls{pippalī},
    \gls{dhyāmaka},
    \gls{māṃsī},
\diff{\gls{lodhra}, \gls{elā}},
\gls{suvarcikā},
\gls{bālaka}, 
\gls{gairika}, as well as \gls{hema},
and
\gls{paripelavā}.

This antitoxin, taken with honey, eliminates slow-acting poison. It is called the
“\se{dūṣīviṣāri}{enemy of slow-acting poison},” and it is not prohibited in other situations.
 
    \item[ 53--54]

If there are any other \se{upadrava}{side-effects}, such as fever, a
burning feeling, hiccups, \se{ānāha}{constipation}, depletion of the
semen, distension, diarrhoea, fainting, skin problems,
\se{jaṭhara}{bellyache}, madness, trembling, then one should treat
each one in its own terms,  using anti-toxic medicines.
    
    \item[ 55] 
    
For a prudent person, the slow-acting poison can be
\se{sādhya}{cured} immediately.  It is \se{yāpya}{treatable} if it is
of a year's standing. Other than this, it should be avoided for the
person who eats unwholesome things.

    \end{translation}


\endinput 
% PLANTS OF GARDEN OR WOODS
% WITH POISONOUS ROOTS AND STEMS
% Arisaema triphyllum
% Colchicum autumnale
% Convallaria majalis
% Dicentra spp.
% Gloriosa superba
% Hyacinthus spp.
% Iris spp.
% Narcissus spp.
% Ornithogalum umbellatum
% Phytolacca americana
% Podophyllum peltatum
% Jack-in-the-pulpit
% Autumn Crocus
% Lily-of-the-Valley
% Bleeding-heart and
% Dutchman’s Breeches
% Glory-lily
% Hyacinth
% Iris, Flags
% Narcissus, Daffodil
% Star-of-Bethlehem
% Pokeweed
% May-apple, Mandrake
%--  DONALD WYMAN
%http://arnoldia.arboretum.harvard.edu/pdf/articles/1966-26--a-few-poisonous-plants.pdf
