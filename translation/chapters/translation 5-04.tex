% !TeX root = ../incremental_SS_Translation.tex
\chapter{Kalpasthāna 4: Snakes and Envenomation}

\section{Introduction} 

The fourth chapter of the Kalpasthāna of the \emph{Suśrutasaṃhitā}
addresses the topic of snake bites and snake venom. Exceptionally for
the Nepalese version of the \SS, the discussion is framed as a
question from Suśruta to the wise Dhanvantari.  Suśruta's questions
are about the number of snakes, how they are classified, the symptoms
of their bites and the pulses or stages of toxic shock experienced by
a victim of snakebite, and related topics.  The taxonomy of snakes is
presented in tabular form in Figures~\ref{snakes1} and
\ref{snakes2}.\footnote{On the idea of notational variants in
    scientific translation, see
    \cites{elsh-2008}{saru-2016}[81--83]{wuja-2021}.} 
    
The \CS\ also addressed this topic of snake taxonomy, but only
included the first three of the \SS's five types, namely Darvīkara,
Maṇḍalī and Rājimān.\footnote{\Ca{6.23.124\,ff.}{577}.} These three
    categories of snakes were framed within a humoral scheme, aggravating
    wind, bile and phlegm respectively, a scheme that was carried forward
    into symptoms and therapy.\footnote{\CS\ \Ca{6.23.165--176}{579}. 
        Note that the \CS\ then described symptoms and therapies without
        reference to the three-humour scheme: \Ca{6.23.177--254}{579--582}.}
        The \SS\ did not use this snake--humour parallelism.  By contrast,
        the system of seven pulses or toxic shocks (\emph{vega}) that was
        central to the \SS's understanding of envenomation is absent from the
        \CS.\footnote{One mention of the term in the \CS\ refers to the peak
            of a tertian fever (\Ca{6.3.70}{404}. In other contexts, it had the
            ordinary-language meaning of a natural “impulse” or “pressure” that
            should not be suppressed (\Ca{1.25.40 et passim}{131--132}).}
    

\section{Literature} 

A brief survey of this chapter's contents and a detailed assessment of
the existing research on it to 2002 was provided by
Meulenbeld.\footnote{\volcite{IA}[292--294]{meul-hist}. In addition to the
    translations mentioned by \tvolcite{IB}[314--315]{meul-hist}, a translation
    of this chapter was included in \volcite{3}[35--45]{shar-1999}. The classic 
    work of \citet[\P93]{joll-1951} offered a short but accurate overview of 
    Indian toxicology.}  There
    also exists a substantial herpetological literature from colonial India
    as well as more recent studies of snakes in the context of cultural and
    religious life.

%Translations of this chapter
% since 2000 have appeared by
%\textcites[131--139]{wuja-2003}[3,
% 1--15]{shar-1999}{srik-2002}.\footnote{For a
%    bibliography of translations to 2002, including Latin (1847),
% English
% (1877),
%Gujarati (1963)
%    and Japanese (1971), see \cite[IB, 314--315]{meul-hist}.}

\citet{chev-1870} gave a characteristically evidential and
gripping nineteenth-century account of death by snakebite in the
context of homicide.  He discussed the specific species of snake most
associated with envenomation and their common geographical
distribution. He also provided numerous vivid case histories of
envenomation as well as murder and execution by deliberate
snakebite.\footcite[368--386]{chev-1870}

The properly ophiological literature of the colonial period began in
the late nineteenth century with the work of Fayrer, whose publication
included striking colour paintings of
snakes.\footnote{\cite{fayr-1874}, first published in 1872.}
    \citeauthor{fayr-1874} provided a biological taxonomy of snakes as
    well as chapters on mortality statistics during the nineteenth
    century, treatment and effects of poison, and experimental data.
    \citet{ewar-1878} included descriptions of appearance and behaviour of
    poisonous snakes and sometimes their local names and reproducing
    Fayrer's illustrations.\footnote{Calling his work a supplement to
        \citet{fayr-1874}, but also being cited by Fayrer, \cite{ewar-1878}
        evidently also collected local indigenous knowledge from his
        “snake-man” (p.\,22).} \citet[75--124]{wall-1913} provided a useful
        analysis of the medical effects of snake envenomation in India
        arranged by the varied symptomatology of different snakes.  He also
        discussed the difference between the symptoms of toxicity and fright
        (69--75) and also the difficulties arising out of uncertainty about
        the effects of snake-bite (124--126).  The \SS\ too recognized the
        emotional and somatic effects of fright (see note \ref{fright} below).
        \citet{wall-1921} provided a wealth of detail of the snakes of Sri
        Lanka, including line drawings.
        
  
        
\citet{seme-1979} traced semiotics of the term \emph{nāga} through
Vedic, Pali and Sanskrit literature.  \citet{doni-2015} provided a
good survey of snakes as protagonists in religious literature from the
\emph{Atharvaveda} through the epics, \emph{Purāṇas} and Buddhist
literature. \citet[31--33 \emph{et passim}]{slou-2016} discussed the
\SS's \emph{Kalpasthāna} as a precursor and influence on later Tantric
traditions of snake-bite interpretation and therapy.  In particular,
the Tantric \emph{Kriyākālaguṇottara} text that Slouber presented
divided snakes into two basic categories, divine and mundane, as the
\SS\ does.\footcite[144--145]{slou-2016}  But unlike the \SS, in the
\emph{Kriyākālaguṇottara} the chief taxonomic principle for both
groups is the four \emph{varṇa}s.  \q{Include info on
    \cite{hida-2019}}
    

A discussion of this chapter specifically in the light of the
Nepalese manuscripts was published by
Harimoto.\footcite[101--104]{hari-2011} After a close comparative
reading of lists of poisonous snakes, Harimoto concluded that, “the
Nepalese version is internally consistent while the [vulgate]
editions are not.”  Harimoto showed how the vulgate editions had been
adjusted textually to smooth over inconsistencies, and gave insights
into these editorial processes.\footnote{The two editions that
    Harimoto noted, \cite{vulgate} and \cite{bhat-1889}, present
    identical texts.}

\subsection{The Seven Stages of Toxic Shock}

A prominent feature the \SS's interpretation of envenomation symptoms
is the concept of seven successive stages or pulses
(\emph{vega}\sse{vega}{pulse}) of toxic shock after a bite. This is
interestingly coordinated with the \SS's concept of the \emph{kalā}s,
which are either seven layers\sse{kalā}{layer} of skin that come into
existence during embryonic development or seven interstitial tissues
that separate the various parts of the
body.\footnote{\label{ka4:kalā}The system of the \dev{kalā} is
    described at \Su{4.4.4--20}{355--357}.  Cf.\
    \volcite{1}[183--184]{josi-maha}, \cite[227--228]{gupt-1983},
    \cite[6]{kutu-1962}, \volcite{1}[247--248 and notes]{meul-hist}. 
    This system of dermal and interstitial \dev{kalā} was not known to
    the \CS\ as such; rather, the \CS\ mentioned six kinds of
    \sed{tvac}{skin} (\Ca{4.7.4}{337}), with different names and
    characteristics, a contradiction discussed by the commentator
    Cakrapāṇidatta (\textit{idem}). It appears in later works such as the
    fourteenth-century \emph{Śārṅgadharasaṃhitā} (1.1.60
    \citep[15]{sast-1931}).}
    
Contemporary clinical studies of snake envenomation and treatment do
not show any awareness of such a seven-stage symptomatology as found
in traditional Indian medicine.\footnote{E.g., \cites {elle-1997}
    {wein-2009} [1747--1749]{pill-2013} [19]{who-2019} {meht-2002}
    {hamz-2021} {desh-2022}.} Exceptionally, the studies by
    \citeauthor{barc-2008} and \citeauthor{oezb-2021}, do identify and
    tabulate three stages of envenomation.\footnote{\cite[1017, Table
        176.3]{barc-2008}, and \cite[7, and Table 1]{oezb-2021}, broadly
        following \citeauthor{barc-2008}.}  The symptoms of these three
        stages are mainly characterized by increasing degrees of edema.  This
        differs from the \SS's detailed characterization of changes in skin
        colour etc.\footnote{I am grateful to Prof.\ Jan Gerris (U. Ghent)
            and Prof.\ Jan Tytgat (KU Leuven) for assistance in finding relevant
            toxicological literature.}


\section{Translation}

\begin{translation}
    
    \item[1] Now we shall explain the \se{kalpa}{procedure} that is
\se{vijñānīya}{required knowledge} concerning the venom in those
who have been bitten by
snakes.\footnote{\label{arunadatta:kalpa}The
    \emph{Sarvāṅgasundarī}, commenting on \AH\ \Ah{1.16.17}{246},
    glossed \dev{kalpa} as \dev{prayoga}.}
    
    \item[3] Suśruta, grasping his feet, questions the wise Dhanvantari, the 
    expert in all the sciences.



    \item[4]
    
    “My Lord, please speak about the number of snakes, and their
divisions, the symptoms of someone who has been bitten, and the
knowledge about the \sse{vega}{toxic reaction}toxic reactions of
poisoning”.\footnote{The expression “toxic reactions” translates
    \dev{vega}, which is other contexts may mean “(natural) urge.”  Here,
    it is rather the discrete stages or phases of physiological reaction
    to envenomation.  Cf.\ the symptoms of cobra poisoning described by
    \citet[80]{wall-1913}.}
        
        
\subsection{[The Taxonomy of Snakes]}
        
    \item[5]
    
    On hearing his query, that distinguished physician spoke.
    
    “The venerable snakes such as Vāsukī and Takṣaka are uncountable. 
    
\item[6--9ab]

“They are snake-lords who support the earth, as bright as the ritual fire,
ceaselessly roaring, raining and scorching. They hold up the earth, with its
oceans, mountains and continents. If they are angered, they can destroy the
whole world with a breath and a look.  Honour to them. They have no role
here in medicine.

“The ones that I shall enumerate in due order are those mundane
ones with poison in their fangs who bite humans.\footnote{The next few
    verses are discussed in detail by \citet[101--104]{hari-2011}, who shows
    that in the taxonomy of snakes, the Nepalese version of the \SS\ has greater
    internal coherence than the vulgate recension.}


\end{translation}

    \begin{figure}
        % The \Tree command calls on the qtree package.
        \centering
        \Tree [.Snakes{ (80)}  
        [.Darvīkara {26 kinds} ]
        [.Maṇḍalin  {22 kinds} ]  
        [.Rājimat  {10 kinds} ]   
        [.Nirviṣa     {12 kinds} ]  
    [.Vaikarañja [.{3 kinds} {7 kinds} ] ]  ]
         \caption{The taxonomy of snakes in the vulgate, \Su{5.4.9--13ab}{571}.}
         \label{snakes1}
\end{figure}
\begin{figure}
\centering          
            \Tree [.Snakes{ (80)}  
            [.Darvīkara {26 kinds} ]
            [.Maṇḍalin  {26 kinds} ]  
            [.Rājimat  {13 kinds} ]   
            [.Nirviṣa     {12 kinds} ]  
            [.Vaikarañja {3 kinds} ]  ]
        \caption{The taxonomy of snakes in the Nepalese version of the \SS.}
        \label{snakes2}
        \end{figure}
    
    \begin{translation}
        \item[9cd--10]    
        
        “There are eighty kinds of snakes and they are divided in five ways:
        Darvīkaras, Maṇḍalins, Rājīmats, and Nirviṣas.  And Vaikarañjas that are
        traditionally of three kinds.\footnote{\citet{hari-2011} translated these
            names as “hooded,” “spotted,” “striped,” “harmless,” and “hybrid.” Figure 
            \ref{snakes1} shows the taxonomy described in the vulgate text; Figure 
            \ref{snakes2} shows the different and more logical division of the Nepalese 
            version of the \SS.}
            
    \item [11] 
    
    “Of those, there are twenty and six hooded snakes, and the same number
of Maṇḍalins are known.\q{Or “There are 20 phaṇins and 6 maṇḍalins.  The
    same number are known. There are 13 Rājīmats.”  Or even, “there are 20
    Phaṇins and six of them are Maṇḍalins.” Are phaṇins really the same as
    darvīkaras?}  There are thirteen Rājīmats.\footnote{The phrasing of
    this śloka is awkward.}
    
    \item [12]
    
    “There are said to be twelve Niriviṣas and, according to tradition, three 
    Vaikarañjas.
    
    
\subsection{[Behaviours]}    

    \item [13--14ef]
    
“If they are trodden on, ill-natured or provoked or even just looking for
food, those very angry snakes will bite.  And that is said to happen in
three ways: \se{sarpita}{serpented}, \se{darita}{torn} and thirdly
\se{nirviṣa}{without venom}.  Some experts on this want to add “hurt by the
snake's body”.\footnote{This might refer to constriction.  The phrase reads
    like a commentarial addition rather than the main text of the \SS.}

\item[15--16]

“The physician can recognize the following as “\se{sarpita}{ophidian}”:
Where a rearing snake  makes one, two or more puncture-marks of its teeth,
when they are deep and without much blood,\footnote{\label{pada-snakes} The
    word \dev{udvṛtta} “aroused” was glossed by Ḍalhaṇa at \Su{5.4.15}{571} as
    \dev{unmoṭya}, a word not found as such in standard dictionaries
    \citep{moni-sans,apte-prac,KEWA,josi-maha}. Semantic considerations
    suggest that the word is not related to $\surd$\emph{muṭ} “break” or
    \emph{mūta/mūṭa} “woven basket.” Perhaps it is related to the Tamil
    \texttamil{மோடி} (\emph{mōṭi},) whose meanings include “arrogance, grandeur,
    display” \citep[\#5133]{DED} or to faintly-documented forms like
    \emph{moṭyate} “is twisted” \citep[\#10186]{CDIAL}. Ḍalhaṇa's \dev{unmoṭya}
    may thus mean “twisting up” or “making an arrogant display.” \par Note that
    \dev{pada} “puncture-mark” (more literally, “footprint”) is being used in
    the same sense as in \Su{1.13.19}{57} when describing the marks on the body
    where a knife scarifies the skin before leeching. See footnote
    \ref{pada-leeches}.} accompanied by a \se{cuñcumālaka}{little ring of
        spots},\footnote{The usual dictionary lexeme is \dev{cañcu}\,, not 
        \dev{cuñcu}
        as in the Nepalese witnesses.  We translate “spots” following Ḍalhaṇa and
        Gayadāsa on \Su{5.4.15}{571}, where they described a group of spots or
        swellings at the site of the bite. On the history of the word \dev{mālaka},
        see \cite{kief-1996}.} lead to degeneration, and are close together and
        swollen.

\item [17]  

Where there are streaks\q{grammar} with blood, whether it be blue or white, the
physican should recognize that to be “\se{darita}{torn},” having a small
amount of venom.

\item[18]

The physician can recognize the locations of the bites of a person in a
normal state as being free from poison, when the location is not swollen,
and there is little corrupted blood.

\item [19]

The wind of a timid person who has been touched by a snake can get
irritated by fear.  It causes
swelling.\footnote{\label{fright}\citet[69]{wall-1913} remarked on the difficulty 
of separating
    toxicity symptoms from the psychosomatic effects of terror:\begin{quoting}
        The gravity of symptoms due to fright does not appear to me to be 
        sufficiently recognised, though there is no doubt in my mind that fatal cases 
        from this cause are abundant, especially among the timid natives of this 
        country.\end{quoting} Wall went on to give several case studies in which 
        patients experienced syncope or even died as a result of bites from 
        toxicologically harmless creatures.}  That is “hurt
    by a snake's body.”

\item [20]

Locations bitten by sick or frightened snakes are known to have little poison.  
Similarly, a site bitten by very young or old snakes has little poison.

\item [21]

Poison does not progress in a place frequented by
eagles,\footnote{Ḍalhaṇa on \Su{5.4.21}{571} identified the \dev{suparṇa}
    as a \dev{garuḍa}. On the bird called \dev{suparṇa}, \citet[72\,ff,
    514]{dave} too noted that it may be a synonym for Garuḍa,
    and in some contexts may refer to the Golden Eagle, Golden
    Oriole, Lammergeyer, etc. \citet[199\,ff, 492]{dave} noted again that the
    Garuḍa is a mythical bird but may refer to the Himalayan Golden Eagle and
    other species of eagle.  He pointed out that historically,
    \begin{quoting}
        The original physical basis for \dev{garuḍa} as the \dev{nāgāśī}
    (snake-eater) was most probably the Sea-Eagle who picks up sea-snakes
    from the sea or sand-beach and devours them on a nearby tree\ldots\  
    \citep[201]{dave}.
    \end{quoting} Dave
    continued with interesting reference to Śrīharṣa's \emph{Nāgānanda}.}
    gods, holy sages, \diff{spirits}, and saints, or in places full of herbs
    that destroy poison.\footnote{For “spirits” the Nepalese version has
        \dev{bhūta} while the vulgate reads \dev{yakṣa}.}

\subsubsection{[Characteristic Features of Snakes]}

\item [22]

Darvīkara snakes are know to have hoods, to move rapidly, and to have rings, 
ploughs, umbrellas, crosses, and hooks on them.


\item [23]

Maṇḍalin snakes are known for being large and slow-moving.  They are 
decorated with many kinds of circles. 
They are like a flaming fire because of their poisons.


\item [24]

Rājimat snakes are smooth and traditionally said to be, as it were,
mottled with multicoloured streaks across and above.

\subsubsection{[Classes of Snake]}

\item[25]

Snakes that are shine like pearls and silver, and that are amber and that
shine like gold, and smell sweet are traditionally thought of as being of
the Brāhmaṇa caste.

\item [26]

Warrior snakes, however, are those that look glossy and get very angry.
The have the mark of the sun, the moon, the earth, an umbrella and
\gls{adrija}.

\item [27]

Merchant snakes may traditionally be black, shine like diamond or have a
red colour or be grey like pigeons.


\item [28]

Any snakes that are coloured like a buffalo and a tiger, with rough skin
and different colours are known as servants.\footnote{Presumably
    “different” from the earlier-mentioned castes.
    
    The sequence of the following three verses is slightly different from the
vulgate (\Su{5.4.29--31}{572}).}

\item[31]

All snakes that are variegated (Rājīmats) move about during the first watch
of the night.  The rest, on the other hand, the Maṇḍalins and the
Darvīkaras, are diurnal.\footnote{The readings of the vulgate, that Rājīmats are 
active in the early night, the Maṇḍalins in the later night, and Darvīkaras in the 
day, seem clearer.}

\item[29]

Wind is irritated by all hooded snakes; bile by Maṇḍalins and phlegm by those 
with many  stripes.

\item [30]

Because of the two classes having greater, lesser or equal class, there is the 
characteristic of irritating two humours.  

And he will explain the opposing view that is to be known as a result of the 
non-union of a male and female.\footnote{The 
sense of the last phrase here is quite different from the vulgate, which says 
only that “details” will be explained below.}


%\item[32]
%\item[33]

\subsection{[Enumeration of Snakes]}
\item[34.1]

In that context, here are the Darvīkaras.
\begin{multicols}{2}
\begin{enumerate}
    \raggedright
    \item \se{kṛṣṇasarpa}{The Black snake};
    \item \se{mahākṛṣṇa}{The Big Black};
    \item \se{kṛṣṇodara}{The Black Belly};
    \item \se{sarvakṛṣṇa}{The All Black};\footnote{Not in vulgate.}
    \item \se{śvetakapota}{The White Pigeon};\footnote{The vulgate adds 
    \se{mahākapota}{The Big Pigeon}.}
    \item \se{valāhako}{The Rain Cloud};
    \item \se{mahāsarpa}{The Great Snake};
    \item \se{śaṃkhapāla}{The Conch Keeper};
    \item \se{lohitākṣa}{The Red Eye};
    \item \se{gavedhuka}{The Gavedhuka};
    \item \se{parisarpa}{The Snake Around};
    \item \se{khaṇḍaphaṇa}{The Break Hood};
    \item \se{kūkuṭa}{The Kūkuṭa};
    \item \se{padma}{The Lotus};
    \item \se{mahāpadma}{The Great Lotus};
    \item \se{apuṣpa}{The Grass Flower};
    \item \se{dadhimukha}{The Curd Mouth};
    \item \se{puṇḍarīkamukha}{The Lotus Mouth};
    \item \se{babhrūkuṭīmukha}{The Brown Hut Mouth};
    \item \se{vicitra}{The Variegated};
    \item \se{puṣpābhikīrṇnābha}{The Flower Sprinkle Beauty};
    \item \se{girisarpa}{The Mountain Snake};
    \item \se{ṛjusarpa}{The Straight Snake};\q{ri- ṛ-?}
    \item \se{śvetadara}{The White Rip};
    \item \se{mahāśīrṣa}{The Big Head}; and
    \item \se{alagarda}{The Hungry Sting};
   
    \end{enumerate}
\end{multicols}

\bigskip

\item[34.2] 

Here are the Maṇḍalins
\begin{multicols}{2}
    \begin{enumerate}
        \raggedright
 \item \se{ādarśamaṇḍala}{The Mirror Ring}; 
 \item \se{śvetamaṇḍala}{The White Ring}; 
 \item \se{raktamaṇḍala}{The Red Ring}; 
 \item \se{pṛṣata}{The Speckled}; 
 \item \se{devadinna}{The Gift of God}; 
 \item \se{pilindaka}{The Pilindaka}; 
 \item \se{vṛddhagonasa}{The Big Cow Snout}; 
 \item \se{panasaka}{The Jackfruit}; 
 \item \se{mahāpanasaka}{The Big Jackfruit}; 
 \item  \se{veṇupatraka}{The Bamboo Leaf}; 
 \item \se{śiśuka}{The Kid}; 
 \item \se{madanaka}{The Intoxicator}; 
 \item \se{pālindaka}{The Morning Glory}; 
 \item \se{tantuka}{The Stretch}; 
 \item \se{puṣpapāṇḍu}{The Pale as a Flower}; 
 \item  \se{ṣaḍaṅga}{The Six Part};
 \item \se{agnika}{The Flame}; 
 \item \se{babhru}{The Brown};
 \item \se{kaṣāya}{The Ochre}; 
 \item \se{khaluṣa}{The Khaluṣa}; 
 \item \se{pārāvata}{The Pigeon}; 
 \item \se{hastābharaṇaka}{The Hand Decoration}; 
 \item \se{tatra}{The Tatra};\footnote{This seems implausible, but 
 otherwise the 
 list of Maṇḍalins would be short.} 
 \item \se{citraka}{The Mark}; 
 \item \se{eṇīpada}{The Deer Foot}.\footnote{The list is short by
     one item.  Perhaps the one of the snakes named in the vulgate, 
     \emph{citramaṇḍala}, \emph{gonasa} or \emph{piṅgala}, should be 
     considered here.}
     
    \end{enumerate}
\end{multicols}
        
        \medskip
        
\item[34.3]

Here are the Rājīmats.\footnote{The following list is one item short.  The 
vulgate text, however, has several names that do not appear in the Nepalese 
Rājīmat list, for example Sarṣapaka and Godhūmaka.}
\begin{multicols}{2}
    \begin{enumerate}
        \raggedright
        \item \se{puṇḍarīka}{The Lotus}; 
        \item \se{rājicitra}{The Stripe Speckle}; 
\item \se{aṅgulirāji}{The Finger Stripe}; 
\item \se{dvyaṅgulirāji}{The Two Finger Stripe}; 
\item \se{bindurāji}{The Drop Stripe}; 
\item \se{kardama}{The Mud}; 
\item \se{tṛṇaśoṣaka}{The Grass Drier}; 
\item \se{svetahanu}{The White Jaw}; 
\item \se{darbhapuṣpa}{The Grass Flower};\footnote{Also in the Darvīkara 
list.}  
\item \se{lohitākṣa}{The Red Eye};\footnote{Also in the Darvīkara 
    list.}  
\item \se{cakraka}{The Ringed}; 
\item \se{kikkisāda}{The Worm Eater};
  
    \end{enumerate}
\end{multicols}

\medskip

\item[34.4]
Here are the Nirviṣas.

\begin{multicols}{2}
    \begin{enumerate}
        \raggedright
\item \se{valāhako}{The Rain Cloud};\footnote{Also in the Darvīkara 
    list.}  
    \item \se{ahipatāka}{Thei Snake Flag}; 
    \item \se{śukapatra}{The White Leaf};
    \item \se{ajagara}{The Goat Swallower}; 
    \item \se{dīpyaka}{The Stimulator}; 
    \item \se{ilikinī}{The Ilikinī}; 
    \item \se{varṣāhīka}{The Year-Snake};
    \item \se{dvyāhika}{The Two-day}; 
    \item \se{kṣīrikāpuṣpa}{The Milk Flower}; 
    \item \se{puṣpasakalī}{The Flower All}; 
    \item \se{jyotīratha}{The Chariot of Light}; 
    \item \se{vṛkṣaka}{The Little Tree};
    \end{enumerate}
\end{multicols}

\medskip

\subsection{[Breeding and Gender]}
\item[34.5]

The Vaikarañjas originate out of contrary unions amongst the three 
\diff{colours}.\q{varṇa means “colour” elsewhere?}\footnote{The word 
\emph{varṇa} in this chapter normally means “colour” not “class.”  (“Class is 
expressed by “jāti.”)  While \emph{kṛṣṇasarpa} is clearly a colour-type, it is less 
obvious that \emph{gonasī} is a special colour, and \emph{rājimat} is a group of 
snakes.}
    Thus:

%\begin{multicols}{2}
    \begin{enumerate}
        \raggedright
\item The \se{mākuli}{Mākuli}; 
\item The \se{poṭagala}{Poṭa Throat}; 
\item The \se{snigdharāji}{Oil Stripe}; 
\end{enumerate}
%\end{multicols}

Amongst those, the \se{mākuli}{Mākuli};  is born when a male Black Snake
mates with a female \se{gonasa}{Cow Snout}, or the reverse.  The
\se{poṭagala}{Poṭa Throat} is born when a male Rājila mates with a female
\se{gonasa}{Cow Snout} or the reverse.  The \se{snigdharāji}{Oily Stripe}
is born when a male Black Snake mates with a female Rājimat, or the
reverse. Their poison is like that of their father, because it is the
superior one out of the two; but others say it is like the mother.   Thus
eighty of these snakes have been described.


\item[35]
Amongst them, males have large eyes, tongues and heads.\footnote{The 
vulgate includes the snake's mouth in this and the next list.}  Females have 
small 
eyes, tongues and heads. Neuters have both characteristics, and are slow to 
exert themselves or be angry.\footnote{The reading \dev{mandaceṣṭākrodhā} 
is an awkward compound; possibly the original reading was \dev{mandaceṣṭāḥ + 
akrodhā} and sandhi was applied twice.}

\item[36] In that context we shall give instruction in a general way
about the sign of having been bitten by any of the 
snakes.

For what reason? 

Because poison acts quickly, like \diff{a fire with an oblation}, a honed
sword, or a thunderbolt.\footnote{Perhaps the image suggested by “a fire
    with an oblation” is that of the Pravargya, in which a large flame rises
    suddenly from the ritual fire.}  And ignored for even a period of time,
    it can drag the patient away. There is not even an opportunity to follow
    the literature.\footnote{The idea seems to be that there is no time to
        consult the verbose āyurvedic teachings.  The
        “\sed{vāksamūhārthavistāra}{extensive meaning of the collection of 
        statements}” is
        singled out as one of Āyurveda's virtues in \Su{5.8.142}{594}.
        Alternatively, perhaps the patient is unable to understand what the
        doctor is saying to him.} 

        And when the symptom of being bitten is stated, there will be three
        ways of treating it because there are three kinds of snake. Therefore
        we shall explain it in three ways. “For this is good for people who are ill,
        and it removes confusion and in this very case it \diff{prevents all 
        symptoms}”.\footnote{In the next passage, the symptoms of snake 
        poisoning are indeed explained under three headings.}

\subsection{[Symptoms of snakebite]}

\item[37] 

In this context, the poison of a Darvīkara causes the skin, nails, eyes,
mouth, urine, feces, and the bite-mark to be black; there is dryness, the
joints hurt and the head feels heavy; the waist, back and neck feel weak;
there is yawning, the voice becomes faint, there is gurgling, paralysis,
dry throat, cough, wheezing, and hiccups; the wind goes upwards, the
patient convulses with sharp pain, black saliva dribbles out, foam
appears, the \se{srotas}{ducts} are blocked and every kind of pain that
is due to wind.\footnote{Cf.\ the similar symptoms of snake venom
    poisoning by the so-called Brahmin warriors of Harmatelia described by
    the classical author Diodorus Siculus (fl.\,ca.\,30--60 \BCE)
    \citep[108]{egge-1975}.}

 %\footnote{The grammar of “\sed{sroto
% 'varodhas}{the
%ducts are blocked}” looks like external sandhi applied inside a
% compound.}

The poison of a Maṇḍalin causes the skin, nails, eyes, teeth, mouth,
urine, feces, bitemark  to be yellow; there is a desire for cold, a
temperature, giving off fumes,\footnote{The term
    “\sed{paridhūpāyana}{giving of fumes}” is not in \cite[596]{moni-sans} as
    such, although \dev{paridhūpana, paridhūmana} and \dev{paridhūmāyana} 
    are
    cited and referred to the \SS. “\sed{paridhūpana}{Giving off fumes}” is
    listed at \SS\ \Su{2.6.13}{291} amongst the symptoms of urinary disease
    caused by phlegm. The editors note a variant reading \dev{paridhūmāyana}
    but do not tell us in which manuscript \citep[291, n.\,3]{vulgate}.
    \Dalhana{2.6.13}{292} glossed \dev{paridhūpana} as
    “\sed{samantatastāpaḥ}{hot all over}” and in our current passage as
    “\sed{sarvāṅgasantāpaḥ}{hot over the whole body}” (\cite[573]{vulgate}).
    See also \volcite{1}[429]{josi-maha}: \dev{dhūmāyana} “\dev{aṅgānāṃ
    dhūmodvamanamiva}” citing the \SS.} a burning feeling, thirst,
    intoxication, fainting, fever, \se{śonitāgamana}{haemorrhaging}, and the
    degeneration of the flesh and fat above and below. There is swelling,
    suppuration of the bite, \se{viparītadarśana}{metamorphopsia}, anger 
    caused by the suffering, and every kind of pain that is due to 
    bile.\footnote{\citet{ghos-2023} describes visual disturbances due to snake 
    envenomation.}

The poison of a Rājīmat causes the skin, nails, eyes, teeth, mouth,
urine, feces, and bitemark to be pale; there  is a cold fever, the hair
stands on end, there is stiffness and swelling of the limbs including the
site of the bite. There is a discharge of viscous phlegm, vomiting, itchy
eyes, and a rattling sound.  The breath is obstructed and there is every
kind of pain due to phlegm.


\item[38]

In that context, “someone bitten by a male gazes upwards,   by a female
horizontally, and by a neuter, downwards.” One bitten by a pregnant snake
has a pale face and becomes \se{ādhmāta}{swollen}. One bitten by a
recently-delivered snake is afflicted with abdominal pain and urinates
with blood. One bitten by a hungry snake craves food.  Those bitten by an
old snake have delayed and slow reactions. And one bitten by a young
snake is fast and keen.  One bitten by a non-venomous snake has the
characteristic mark of non-poisoning.\footnote{The grammar of
    \dev{aviṣaliṅgam} is not quite right; it should be a masculine or plural
    bahuvrīhi.} Some that are bitten by a blind snake become blind.  A
    \se{ajagara}{constrictor} is deadly because it swallows, not because of
    poison.

\subsubsection{[Toxic reactions]}

\item[39]

In that context, all snake toxins have seven toxic reactions\sse{viṣavega}{toxic 
reaction}.\footnote{Cf.\ the same concept in the context of plants, at 
\pageref{stagesofshock}}

\paragraph{[Darvīkaras]}   

Thus, at the first pulse of the Darvīkaras the poison corrupts the blood.
That corrupted blood turns black.  Because of that, blackness and a
feeling of ants crawling about on the body develop.\footnote{Strictly, we
    would expect a dual verb here, instead of the plural of the witnesses.}

In the second pulse, it corrupts the flesh.  That causes extreme blackness and  
lumps. 

In the third, it corrupts the fat. That causes a discharge at the bite,
heaviness of the head and an eclipse of the
vision.\footnote{\Dalhana{5.4.39}{574} glossed the last expression as
    “\sed{dṛṣṭyavarodha}{blockage of the vision}.”}
    
    % got to here 2023-11-22

In the fourth, it penetrates the \se{koṣṭha}{trunk of the body}.  From there, it 
irritates the humors, particularly phlegm. That causes exhaustion and oozing 
phlegm, and dislocation of the joints.

In the fifth pulse, it penetrates the bones.  That causes breaking of the joints, 
hiccups 
and burning.

In the sixth pulse, it penetrates the marrow.  That causes humours in the
\se{grahaṇī}{seat of fire in the gut}, heaviness of the limbs, diarrhoea,
pain in the heart and fainting.\footnote{The “\sed{grahaṇī}{seat of fire
    in the gut}” is an ayurvedic organ in the digestive tract that does not
    correspond to any specific organ known to contemporary anatomy.  For
    discussion, see \cites[v.\,1,
    304]{josi-maha}[619]{meul-1974}[544--545]{das-2003}.}
    
In the seventh, it penetrates the semen and greatly irritates the
\se{vyāna}{vyāna breath}, and  causes the \se{kapha}{phlegm} to run
imperceptibly out of the \se{srotas}{tubes}.  That causes the appearence
of \se{śleṣman}{mucous}, breaking of the hips, back and shoulders, impediment 
to all movements and shortness of breath.
    
    
\paragraph{[Mandalins]}    
Thus, at the first pulse of the Mandalins, the poison corrupts the blood.  
Corrupted by that, it turns yellow. That causes a yellow appearance and a 
\se{paridāha}{feeling of heat all over}.

In the second pulse, it corrupts the flesh.  And that causes the
limbs to be very yellow and an extreme \se{paridāha}{feeling of heat
    all over}, and swelling at the bite.

In the third, it corrupts the fat.  That causes a discharge at the black bite and 
sweating.

In the fourth, it penetrates as before and brings on fever.

In the fifth, it causes heat in all the limbs. 

In the sixth and seventh, it is the same as before. 

\paragraph{[Rājīmats]}

Thus, in the first pulse of the Rājīmats, the poison corrupts the blood. 
Corrupted by that, it turns yellow.  It causes a person to have hair standing on 
end and a pale appearance. 

In the second pulse, it corrupts the flesh. That causes him to become
pale and to become extremely \se{jāḍya}{benumbed}.

In the third, it corrupts the fat.  That causes moistness of the bite and runny eyes 
and nose. 

In the fourth, it is the same as before.  After penetrating, it brings on 
\se{manyāstambha}{stiffness of the neck} and heaviness of the head.

In the fifth, speech is slurred and there is a cold fever.

In the sixth and seventh, it is the same as before. 



\subsection{[Summary Verses]}

\item[40]
There are verses on this.

\begin{sloka}
It is well known that there are seven \se{kalā}{interstitial layers} in
between the \se{dhātu}{bodily tissues}.  Poison passing through these one
by one produces the \se{vega}{toxic reaction}.\footnote{See note \ref{ka4:kalā}
    above.}
\end{sloka}


\item[41]
\begin{sloka}
The interval taken by the \se{kālakalpa}{deadly substance}, \se{\root
    ūh}{propelled} by \se{samīraṇa}{air}, to cut the layers of 
    skin\sse{kalā}{layers of skin} is
known as the “\se{vegāntara}{pulse
    interval}”.\footnote{\Dalhana{5.4.41}{574} glossed \dev{kālakalpa} as
    \dev{mṛtyusadṛśaṃ viṣaṃ} “the poison resembles death.”}
\end{sloka}


\item[42]

\begin{sloka}
    In the first pulse, an animal has a swollen body, is distressed and
broods.\footnote{The verb \root\dev{pradhyai} ”meditate, be
    thoughtful, brood” is unexpected here and in the  second class, an
    epic form. \Dalhana{5.4.42}{574} noted that some manuscripts did not
    include the text about animals from this point on.  The fact that
    these verses are present in the Nepales witnesses testifies to their
    antiquity.}
    
    In the second, it dribbles somewhat,\footnote{The Nepalese witnesses use 
    \dev{lāli-}, not \dev{lālā-}, for “saliva.”} the hair stands up on its 
        body, and it has \se{\root pīḍ}{pain} in the heart.
\end{sloka}


\item[43]

\begin{sloka}
    
    The third stage brings headache and it breaks the \diff{ears} and
necks.\footnote{The scribe of MS H emended the text to read
    \dev{kaṇṭhagrīva} with the vulgate.  Intransitive use of pass.\
    \dev{bhañj}.}  In the fourth, the bewildered creature trembles and gnashing 
    its teeth, it gives up life.
    
\end{sloka}

\item[44--45]

\begin{sloka}
\label{bird-pulse}

    Some experts say that elephants have three toxic reactions\sse{vega}{toxic 
    reaction}.\footnote{On \dev{antaḥsveda} as “elephant,” cf.\
\emph{Arthaśāstra}
    9.1.46 \parencites[v.\,1, 219]{kang-1969}[351]{oliv-2013}: \dev{hastino 
    hyantaḥsvedāḥ kuṣṭhino bhavanti // 46//}.}
  
    
So, at the first toxic reaction, an bird becomes bewildered and is confused
from that point on.  At the second, the bird is distressed and,
crying out, it dies.

Some people claim that where birds are concerned, there is really just a
single \se{vega}{toxic reaction} and that amongst animals like cats and
mongooses, poison does not take much effect.\footnote{See on this
    subject:  \cites[39-40]{brun-1909}[88-89]{mint-1969} (references taken from 
    \volcite{1B}[399, n.\,124]{meul-hist}).}

% dangling chapter-ending. 

\end{sloka}
    % 
    %https://global.oup.com/us/companion.websites/9780190200886/student/chapter10/gline/quotation/
\end{translation}
