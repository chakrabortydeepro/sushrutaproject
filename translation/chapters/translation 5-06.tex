%d !TeX root = ../incremental_SS_Translation.tex
\chapter{Kalpasthāna 6: Mice and rats}
\label{mūṣikā}


\section{Introduction}

This chapter is numbered 6 in the Nepalese version, but 7 in the vulgate.

\subsection{Literature}

In 2004, Umberto Eco published a characteristically subtle and enlightening 
book about translation entitled \emph{Mouse or Rat}.\footcite{eco-2004} 
The title alluded to Eco's discussion of the example of translating words for 
mice and rats across several European languages that do not always 
distinguish these animals from each other, or confuse them in other ways.  In 
Sanskrit too, \emph{mūṣikā}, in the title of this chapter, does not distinguish 
between mouse and rat.  

%A brief survey of this chapter's contents and a detailed assessment
%of the existing research on it to 2002 was provided by
%Meulenbeld.\footnote{\volcite{IA}[295]{meul-hist}. In addition to the
%    translations mentioned by \tvolcite{IB}[314--315]{meul-hist}, a
%    translation of this chapter was included in
%    \volcite{3}[61--66]{shar-1999}.} 



\section{Translation}

\begin{translation}
    
    \item[1] 
    
Now I shall explain the \se{kalpa}{procedure} on the topic of
\se{mūṣikā}{mice}.\footnote{The word \dev{mūṣikā} does not distinguish 
between rats and mice; the same is true for MIA and NIA derivatives
\cite[\#10258]{CDIAL}.}
    
    \item[3] 
    
    
    \end{translation}
