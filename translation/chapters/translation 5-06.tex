%d !TeX root = ../incremental_SS_Translation.tex
\chapter{Kalpasthāna 6: Rats and Rabies}
\label{mūṣikā}


\section{Introduction}

A notable macro-difference between the vulgate and the Nepalese
versions of the \SS\ is that this chapter and the next are reversed
in the vulgate.  In the Nepalese version, this is chapter six and the
chapter on antitoxic drumming is chapter seven.\footnote{See
    p.\,\pageref{kalpa-chapter-sequence} above.}

\subsection{Mouse or Rat?}

In 2004, Umberto Eco published a characteristically subtle and
enlightening book about translation entitled \emph{Mouse or
    Rat?}.\footcite{eco-2004} The title alluded to Eco's discussion of the
example of translating words for mice and rats across several
European languages that do not always distinguish these animals from
each other, or confuse them in other ways.  In Sanskrit too,
\emph{mūṣikā}, the subject and title of this chapter, does not
distinguish between mouse and rat.  The same is true for MIA and NIA
derivatives.\footcite[\#10258]{CDIAL}  It is hard to know quite how
to translate the term since “rodent” is too broad a term.  In what
follows, I have chosen “rat” for \emph{mūṣikā} in order to produce a
working translation of a text about an animal that is viewed as toxic
and threatening.  ``Mouse'' does not have quite these connotations
for a contemporary English speaker.\footnote{\citeauthor{bhis-1907}
    made the same choice \pvolcite{2}[728--736]{bhis-1907}.}

The rodents that may be described as mice or rats in contemporary
South Asia and that are especially associated with the spread of
disease include the house or black rat (\emph{Rattus rattus}, L.),
the brown rat (\emph{R.\ norvegicus}, Berkenhout), the house mouse
(\emph{Mus musculus}, L.) and bandicoots
(\emph{Bandicota}).\footcite[194]{bia} Also present in SA are the
Indian desert gerbille (\emph{Meriones hurrianae}, Jerdon), the
Indian gerbille (\emph{Tatera indica}, Hardwicke), the spiny field
mouse (\emph{Mus platythrix}, Bennett), the Indian field mouse
(\emph{M. booduga}, Gray), the Metad (\emph{Millardia meltada},
Gray), the Indian bush rat (\emph{Golunda ellioti}, Gray), the
longtailed tree mouse (\emph{Vandeleuria oleracea}, Bennett), Royle's
vole (\emph{Aticola roylei}, Gray), the Indian mole-rat
(\emph{Bandicota bengalensis}, Gray \& Hardwicke),\footnote{“Recent
    studies\ldots show that the mole-rat forms 98\% of the total rodent
    population of Calcutta,” \cite[206]{bia}.} the bandicoot rat
    (\emph{B. indica}, Bechstein), the shorttailed bandicoot
    (\emph{Nesokia indica}, Gray \& Hardwicke), the whitetailed wood rat
    (\emph{Madromys blanfordi}, Thomas), the bay bamboo rat
    (\emph{Cannomys badius}, Hodgson), and other similar
    rodents.\footnote{\cite[ill.\ plates \,45, 46 \emph{et passim}]{bia}.
        See also \cite[passim]{meno-2014}.} However, plausibly matching
        these creatures to the Sanskrit names listed in this chapter is hard
        to impossible.\footnote{Mouse-words that we do not see in this
            chapter include the \emph{kirika, giri, girikā} group
            \pvolcite{1}[353, 488, 566]{EWA}.}  Almost no works engage directly
            with the representation or identity of rodents in pre-modern
            India.\footnote{One of the few is \cite[ch.\,3]{geer-2008}.}



\subsection{Literature}

A brief survey of this chapter's contents and reference to the
limited existing research on it to 2002 was provided by
Meulenbeld.\footnote{\volcite{IA}[295--296]{meul-hist}. In addition
    to the translations mentioned by \tvolcite{IB}[314--315]{meul-hist},
    a translation of this chapter was included in
    \volcite{3}[67--77]{shar-1999}. \citet{sekh-2023} omitted mention of
    this type of poisoning, although he discussed rabies, a subsection of
    this chapter.}
    

A rich description of Indian rodents is available by \citet[ch.\,13,
esp.\,205--215]{bia}, including several useful illustration. 
Unfortunately, \citeauthor{bia} rarely gave Indian-language
names.

In Sanskrit literature, the \emph{Arthaśāstra} refers to the problem of rats 
more than once.  To rid a country of the threat of rats, 
\begin{quote}    
    When there is a danger from rats, cats and mongooses should be
released. If these are captured or killed, the fine is 12 Paṇas,
as also for not keeping dogs confined, except in the case of
foresters. He should strew grains smeared with the milk of the
Snuhi-plant or mixed with secret compounds. Or, he should
institute a rat tax; or thaumaturgic ascetics should perform a
pacificatory rite. On the days of the moon’s change \ldots,
moreover, he should have rites of rat worship carried
out.\footnote{\emph{Arthaśāstra} 4.3.20--26, tr.\ \cite[230]{oliv-2013}.}
\end{quote}

\newpage

\section{Translation}

\begin{translation}
    
    \item[1] 
    
Now I shall explain the \se{kalpa}{procedure} relating to 
\se{mūṣikā}{rats}.\footnote{The word \dev{mūṣikā} does not distinguish 
between rats and mice.  See Introduction above.}
    
    \item[3] 
    
    Learn concisely about aforementioned eighteen kinds of rats that have 
    poison in their semen, according to their names, characteristics and the 
    herbal treatments.\footnote{Rats with poisonous semen were mentioned 
    in \Su{5.3.5}{567} (see p.\,\pageref{sukravisa} above).}
    
\subsection{The types of rat}
    \item[4--6]
    
    The eighteen rats are traditionally called,\footnote{\Dalhana{5.6.4}{582} 
    gave no 
    comment on any of these names.  The identifications are 
    mostly guesswork and sometimes whimsical.  The glossary gives lexical 
    discussion of individual names.}
\begin{multicols}{2}
    \begin{enumerate}
        \item \Gls{lālana},
        \item \Gls{putraka},
        \item \Gls{kṛṣṇa},
        \item \diff{\Gls{vasira1}},
        \item \Gls{cikkira}, % Beng Cika = chuchundara, regional
        \item \Gls{chuchundara} % Bengali house shrew Suncus murinus       
        \item \Gls{arala1},% hapax legomenon, Dravidian
\footnote{The word \dev{arala} is a hapax legomenon and has not 
previously been identified as a lexeme because it did not appear in earlier 
editions of the \SS.  It is a loan-word from Dravidian (see glossary).}        
        \item \Gls{kaṣāyadaśana},
        \item \Gls{kuliṅga},
        \item \Gls{ajita},
        \item \Gls{capala},
        \item \Gls{kapila1},
        \item the one called \Gls{kokila1} and 
        \item \Gls{aruṇa},
        \item the large black \gls{unduru}, 
        \item \Gls{śveta1}, together with the
        \item the large \Gls{kapila1},
        \item and the \Gls{kapota1}-like rat.\footnote{The Nepalese list has 
        \dev{vasira} (\Gls{vasira1}) for the 
        vulgate's \dev{haṃsira}.  The terms \dev{ākhu}, \dev{mūṣikā} and 
        \dev{unduru} are here used as generic names of rat/mouse rodents.}
\end{enumerate}
    \end{multicols}
\medskip
    
\item[7]

If a part of the body has their sperm fall on it or if they touch it
with their nails or teeth, etc., that have been touched by sperm,
then the blood is corrupted.\footnote{\label{alambayana2}On this,
    \Dalhana{5.7.7}{582} quoted an authority called Ālambāyana who
    elaborated on this subject (see \volcite{IA}[658]{meul-hist} for
    references to this author of a lost treatise on toxicology). Ḍalhaṇa
    also cited Ālambāyana elsewhere on the topics of insects and spiders
    \pvolcite{IB}[722, note 5]{meul-hist}. See also the \AS's assertion 
    that Ālambāyana was responsible for the doctrine of \se{vega}{toxic 
    pulse}s, p.\,\pageref{alambayana1} above.
    
Ālambāyana, who was already known as “the famous soul of compassion”
in the \emph{Mahābhārata} (13.18.4), was also known in Buddhist
literature. Book 22, tale 543 of the Jātakas includes mention of
an Ālambāyana who claimed to be a doctor and specialist in
snakebite poisons: \emph{nāhaṃ dijādhipo homi, na diṭṭho garuḷo
mayā, āsīvisena vitto ti vejjo maṃ brāhmaṇaṃ vidū ti 793}
(\volcite{6}[181]{faus-1877}, tr.\  \volcite{6}[95]{cowe-1895}).
In the same tale, there is a herbal “Ālambāyana mantra” given to
an ascetic by a Garuḍa who has just caught and eaten a Nāga, thus
invoking the Garuḍa-snake-poison motif
\pvolcite{6}[93--94]{cowe-1895}.  The Jātakas were translated
into Chinese in the third century \CE.
    
    %
    %    Pāli text: \volcite{6}[177\,ff.]{faus-1877}
    %
    %    \pvolcite{6}[93--94, 95--98, 99]{cowe-1895}
    %
    See further discussion by \citet[33--34]{slou-2016}, who calls
    the mantra “Alampāyana,” adopting the reading of the Burmese MS
    Bd against the Fausbøll's critical reading  “Ālambāyana” 
    \pvolcite[see]{2 \& 3}[ Preliminary remarks 3 and 7]{faus-1877}.}
    
\item[8--10ab]

It happens that there are \se{granthi}{lumps}, swellings,
\se{karṇika}{small ear-like growths} and rings, accumulations of
severe \diff{\se{piṭaka}{blisters}}, \se{visarpa}{spreading rashes}
and \se{kiṭibha}{dark, rough patches of
    skin}.\footnote{\label{karṇika}“Little ears” was strikingly described
    by \Dalhana{5.7.8}{582} as looking like the seed pod in the middle of
    a lotus (\dev{kamalamadhyabījakośākṛtiḥ}), a graphic image (see also
    \Dalhana{5.8.136}{594}).  Perhaps similar to hypergranulation. 
    %    lumps of
    % flesh
    % that look like
    %    the ears of a lotus (\dev{karṇikā māṃsakandī,
    %    padmakarṇikākāratvātkarṇkikā bhaṇyate}).
    The Nepalese version has \dev{piṭaka} “blisters” for the
    vulgate's \dev{pīḍaka} “boils” (itself perhaps a typo for
    \dev{piḍaka}).  \dev{kiṭibha} “dark rash” was described by
    \Dalhana{1.11.7}{46} as a kind of \dev{kuṣṭha}, which is
    variously a skin disease of pallor, leucoderma, or leprosy
    \citep{emme-1984}. But it was described in the \CS\ as being dark
    and as rough as a callous to the touch (\Ca{6.7.21cd--22ab}{451})
    \pvolcite{1}[208]{josi-maha}.}  There are severe conditions such
    as pain in the joints, pain, fever, fainting, weakness, loss of
    appetite, exhaustion, nausea and
    horripilation.\footnote{\dev{parvabheda} “pain in the joints” was
        glossed by \Dalhana{5.7.9}{582} as “spots on the joints”
        (\dev{sandheḥ sphoṭaḥ}).  This seems unlikely, since symptoms on
        the surface of the body were described in the previous verse, and
        also because of the obvious etymological meaning of the
        compound.}
        

This is a concise description of the appearance of someone who has been 
bitten.  Now listen to a longer version. 

\subsection{Detailed symptoms}

\begin{itemize}
\item[10cd--11ab]

The \Gls{lālana} causes a flow of saliva, vomiting and hiccups.  For that, one 
should lick a paste of \gls{taṇḍulīyaka} with honey. 

\item [11cd--12]

The \Gls{putraka} causes the limbs to droop and creates a pale
\diff{beauty},\footnote{The expression \dev{-valgu} “beauty” in the
    Nepalese MSS, for the vulgate's simpler \dev{-varṇa} “complexion,” is
    unusual.} and the body is heaped with lumps like the young of a
    rat.\footnote{The grammar here is very loose. \dev{śiśur} cannot
        stand outside the compound, which should read
        \dev{mūṣikaśisusaṃsthitaiḥ}.  The vulgate text has the simpler and
        grammatical \dev{ākhuśāvakasannibhaiḥ} “resembling the offspring of a
        rat.”}  One should lick \gls{śirīṣa}, \gls{iṅgudi} and \gls{patra}
        with honey.\footnote{\Dalhana{5.7.11-12}{582} here cited a passage 
        by
            an unknown author called Nāgārjuna, about the visible symptoms of a
            bite by this kind of rat (cf.\ \cite[45--46]{shar-1982},
            \volcite{IB}[497, note 100]{meul-hist}) as well as variant readings
            by Gayadāsa and Jejjaṭa on the exact formulation of the lickable
            medication.}

\item [13]

The \Gls{kṛṣṇa} causes one to vomit blood, especially when the
weather is bad.  One should drink \gls{śirīṣa} and \gls{patra},  
with \gls{kuṣṭha} and \gls{elā}, with the 
\gls{kiṃśuka} ashes.\footnote{\Dalhana{5.7.13}{583} explained “with the 
ashes of \gls{kiṃśuka}” as “water with the ashes of \gls{kiṃśuka}.”}

\item [14]

The \Gls{vasira1} causes a person have a revulsion for food, to yawn,
and makes their body-hair \diff{leprous}.\footnote{The qualifier
    \dev{kuṣṭhatā} (\dev{romṇāṃ}) is odd; the vulgate's \dev{harṣaṇa}
    “horripilation” reads more easily. \dev{kuṣṭha} has a lesser-known
    meaning “prominent part, mouth or opening” which might perhaps be
    considered here, though it is hard to see how.}  They should drink
    items like \gls{āragvadha} and be quickly made to vomit.

\item[15]

The \Gls{cikkira} causes headache, swelling, hiccups and nausea.  One
should have thorough emesis using
decoctions\sse{kvātha}{decoction} of \gls{jālinī}, and he should drink
the juice of \gls{aṅkolla}.

\item[16cd--ab]

The \Gls{chuchundara} causes constipation, paralysis of the neck, and
\se{vijṛmbhikā}{gasping}.\footnote{\dev{vijṛmbhikā} is one of the
    eighty wind diseases listed in the \emph{Kāśyapasaṃhitā} and glossed
    by Hemarājaśarman as “yawning” (Hindī \dev{jaṃbhāī},
    \Ka{1.27.19--28}{41--42}). However, in the \CS\ it is a term for one
    of the disorders of an improperly treated post-partum umbilical cord
    (glossed by Ḍalhaṇa as \dev{muhurmuhurvṛddhimatī} “growing larger
    moment by moment,” \Ca{4.8.45}{348--349}) and translated by
    \tvolcite{1}[480]{shar-1994} as “umbilical hernia.”
    Cf.\,\volcite{1}[756]{josi-maha}.} %
    In this case, one should administer a caustic made of
    \gls{yavanāla} and \gls{ārṣabhī} as well as the two
    \glspl{bṛhatī}.\footnote{Note that half-verses 16cd and 16ab are
        reversed compared to the vulgate edition.  This makes the caustic
        a remedy for the bite of the \gls{chuchundara}, while the
        earlier \gls{jālinī} remedy is for the \gls{cikkira}, which makes
        betters sense.
        
        The vulgate has text at this point, 17 and 18ab, that are not
present in the Nepalese version.  They are about further
symptoms and treatment of stiffness of the neck, anosmia,
etc., presumably arising from the bite of the
\gls{chuchundara}. \Dalhana{16cd--17}{583} recorded different
readings from Gayadāsa's commentary here (see edition notes);
it seems these verses became slightly confused at an early
period. We would expect symptoms of the bite of the
\gls{arala} at this point in the text, and the Great Antidote
treatment in the next line would be its therapy.}

    
\item[18cd--19]

The \Gls{arala1} causes stiffness of the neck and pain in the area of the bite.
In that case, one should lick \se{mahāgada}{The Great Antidote}, that
is of great \se{vīrya}{potency}, together with honey.\footnote{”The great
    antidote” recipe is described at \SS\ 5.6.63 (p.\,\pageref{mahāgada}
    above).}

\item[19cd--20ab]

The \Gls{kaṣāyadanta} causes sleep and especially emaciation. 
In that case, one should lick the sap and seeds of \gls{śirīṣa} with 
honey.\footnote{The difficult expression \dev{śirīṣasya sāramāṣakān} 
probably accounts for the easier version of the vulgate, with its dvandva 
\dev{sāraphalatvacaḥ}.  Taking \dev{sāramāṣakān} as a dvandva, we can read 
\dev{māṣaka} as in the compound \dev{śirīṣamāṣaka} “\gls{śirīṣamāṣaka}.”}

\item[20cd--21ab]

The \Gls{kuliṅga} causes pains, swelling and lines up to the area of the bite.
In that case, one should lick the two kinds of \gls{sahā}, together with 
\gls{sinduvāra} and honey.  

\item[21cd--22ab]

The \Gls{ajita} causes nauseous fainting, \se{hṛdgraha}{heart-seizure} and 
blackness of the limbs. 
In that case, one should lick \gls{mañjiṣṭhā} mixed with the milky latex of 
\gls{snuhā} and honey. 


\item[22cd--23ab]

The \Gls{capala} causes vomiting and fainting together with thirst.
One should drink \gls{triphalā} with wood-ash, \gls{jaṭā} and 
honey.

\item [23cd--24ab]

The \Gls{kapila1} causes a wound, \se{koṭha}{hives}, fever, and an
outbreak of \se{granthi}{lumps}.\footnote{\dev{koṭha} was a skin
    ailment variously described by authorities as a redness that appeared
    and disappeared rapidly, that was itchy, that was caused by an excess
    of salty items, etc.\ (see \volcite{1}[239]{josi-maha},
    \volcite{IIB}[76, n.\,47]{meul-hist}). It may have referred to
    conditions such as urticaria, allergy, ringworm or vitiligo. “Hives” has a 
    history going back to ca.\,1500, referring to various eruptions in the skin that 
    may feel hot \citep[s.v.\ 
    “\href{https://doi.org/10.1093/OED/3255856400}{hives (n.)}”]{OED}.} In 
    this
    case, \gls{śvetā} or white \gls{punarṇavā} should be licked with
    honey.

\item[24cd--25ab]

The \Gls{kokila1} is said to cause lumps, fever, and an intense 
\se{dāha}{feeling of heat}. 
In that case, one should drink  ghee cooked with an \se{kvātha}{decoction}
of  \gls{nīlā} and \gls{varṣābhu}. 


\subsubsection{The last five, from the \Gls{aruṇa} on}
\item[25cd--26]

The \Gls{aruṇa} causes the wind to be angry, creating illnesses that
originate in wind. The Large Black (\gls{unduru}) causes bile, the
\Gls{śveta1} phlegm, the \Gls{mahākapila} causes blood, and the
\Gls{kapota1} causes all four.\footnote{Note the switch to humoral
    theory with these last five rats in the list, and the assumption of blood as a 
    fourth humour .}

\item[27]

In the bites of these ones there are lumps, rings and
\se{karṇika}{small ear-like growths}.\footnote{On \dev{karṇikā}, see footnote
    \ref{karṇika}.}  There are accumulations of \se{piṭaka}{blisters} on the 
    \diff{body}, and severely painful swellings.

\item[28--31]    
    
A \se{prastha}{half litre} each of curds, milk and ghee are
\diff{measured} out.\footnote{The measure of a \dev{prastha} is
    approximate and different authors have various estimates.} Make a
    broth of \gls{karañja}, \gls{āragvadha}, \gls{vyoṣa}, \gls{bṛhatī},
    \gls{aṃśumatī}, and \gls{sthirā},\footnote{\dev{a{}ṃśumatī} and
        \dev{sthirā} are both normally identified as beggarweed, but when a
        pair are mentioned the second is probably \gls{pṛṣṇaparṇī}.} and once
    again make that broth into one fourth part. %
    One should add    \gls{tṛvṛt}, \diff{\gls{tilva}}, \gls{amṛtā},
    \gls{vakra}, \gls{sarvagandhā}, \gls{agamṛttikā},\footnote{For the
        vulgate's reading \dev{samṛttikā} “with earth,” \Dalhana{5.7.29}{583}
        specified “black earth” and noted that some people read
        \dev{ahimṛttikā} “snake earth” meaning earth taken from anthills,
        while Jejjaṭa read \dev{agavṛttikā}, meaning \dev{śallakī},
        “\gls{śallakī}”  (see also \cite[392]{gvdb}). Jejjaṭa's reading is
        essentially that of the Nepalese MSS, with a \dev{ma}/\dev{va}
        alternant, if Trikamji Ācārya's edition is correct on this.} %
        \gls{kapittha}, \gls{dāḍima}, and \gls{tvac}. Mix all that together
        and cook it over a gentle flame.
This gets rid of the poison of the five rats from \Gls{aruṇa} on.

Alternatively, prepare in the juices of \gls{kākādanī} and \gls{kākamācī}.



\item[32]

Also, you should pierce the affected \se{sirā}{veins} and apply purifications.
As an alternative, one may apply this rule in all cases of rat poisoning.

\item[33--34ab]

One should cauterize the bite, then bleed it and smear the scarified area with a 
paste of \gls{śirīṣa}, \gls{rajanī}, \diff{\gls{vakra}}, \gls{kuṅkuma}, and 
\gls{amṛtā}.\footnote{The vulgate substitutes \dev{kuṣṭha} for 
\dev{vakrā}.}
Emesis is with a \se{kvātha}{decoction} of \gls{nīlinī} with \gls{śuka} and 
\gls{aṅkolla}.\footnote{The vulgate has two and a half more verses at this 
point, expanding the recipe considerably and adding the appropriate verb, “he 
should vomit."}

\item[37]

When doing a purge, \gls{tṛvṛt}, \gls{dantī}, and \gls{triphalā} are
recommended;  when purging the head, either the juice of \gls{śirīṣa} or its 
fruits. Cow-dung with a lot of \gls{kaṭutrika} is good in \se{svarasāñjana}{eye 
collyrium}; an electuary of \gls{kapittha}, cow-dung


\end{itemize}
% got to here 

    \end{translation}
\endinput

%In \As{6.40.35}{844}, Ālambāyana is said to be the authority who 
%declared that the seven \se{vega}{pulses} of toxic shocks affect, 
%successively, the seven \se{āśraya}{substrata} of the body, from blood to 
%semen.    

% Ālambāyana on 5.7.7: “śukrēṇātha purīṣēṇa mūtrēṇāpi nakhais tathā| 
% daṁṣṭrābhir vā kṣipantīha [2] mūṣikāḥ pañcadhā viṣam”
%    tantrāntare: “garbhiṇyā mūṣikayā daṣṭē amlādidōhadaḥ, 
% r̥tumatyā daṣṭē raktamēhanamādhmānaṁ ratiśīlatā ca”

% SS ka 8.24, ka 8. 83-84, ka 8. 120

% Mādhavanidāna 69 viṣaroga 21-24,  Madhukośa, Vijayarakṣita & 
%Śrīkaṇṭhadatta: “naiti raktaṁ kṣatādyasya latāghātairna 
%rājikāḥ| na lōmaharṣaḥ śītādbhirvarjayēttaṁ viṣārditam||”
%
%28, Ātaṅkadarpaṇa “sīdanti kēśalōmāni tasmin pakvāśayaṁ gatē|”

