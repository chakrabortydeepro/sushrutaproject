%d !TeX root = ../incremental_SS_Translation.tex
\chapter{Kalpasthāna 6: Beating Drums}
\label{dundubhi}

\section{Introduction}

\subsection{Literature}

A brief survey of this chapter's contents and a detailed assessment
of the existing research on it to 2002 was provided by
Meulenbeld.\footnote{\volcite{IA}[295]{meul-hist}. In addition to the
    translations mentioned by \tvolcite{IB}[314--315]{meul-hist}, a
    translation of this chapter was included in
    \volcite{3}[61--66]{shar-1999}.}

\section{Translation}

\begin{translation}
    
    \item[1] Now I shall explain the \se{kalpa}{formal procedure}
for drumming.
    
    \item[3] 
     One should take the ash of the following items, mix it with 
     cows' urine and an alkaline compound, take an extract and cook 
     it thoroughly:
     % deep  breath ...
     \gls{dhava},
     \gls{aśvakarṇa},
     \gls{tiniśa}, 
     \gls{picumarda}, 
     \gls{pāṭalī}, 
     \gls{pāribhadraka},\footnote{\label{drum-detox}The ingredients to this point
         are similar to the water-detoxifier described in \SS\ \Su{5.3.9}{568}, 
         p.\,\pageref{water-detox} above.}
     \gls{udumbara}, 
     \gls{karaghāṭaka}, 
     \gls{arjuna},
     \gls{sarjja}, 
     \gls{kapītana}, 
     \gls{śleṣmātakā}, 
     \gls{aṅkoṭha},
     \gls{kuṭaja},
     \gls{śamī}, 
     \gls{kapittha},
     \gls{aśmantaka},
     \gls{arka},
     \gls{ciribilva}, 
     \gls{mahāvṛkṣa}, 
     \gls{arala}, 
     \gls{madhuka},
     \gls{madhukaśigru}, 
     \gls{śāka},
     \gls{gojī}, 
     \gls{bhūrja}, 
     \gls{tilvaka},
     \gls{ikṣuraka},
     \gls{gopaghoṇṭā}, 
     and
     \gls{arimeda}.
     
One should add to this the powder of the following items, together
with an equal quantity of powdered metals: \gls{pippalī},
\gls{pippalīmūla}, \gls{taṇḍulīyaka}, \gls{varāṅga}, \gls{coraka},
\gls{mañjiṣṭhā}, \gls{karañjikā}, \gls{hastipippalī}, \gls{viḍaṅga},
soot\sse{gṛhadhūma}{soot}, \gls{ananta}, soma,\q{Roots page
    number.}\footnote{The literature on the identification of Soma is
    large \citep[passim]{wuja-2003}. To the cited literature, the useful
    historical discussion by \citet[449--455]{gvdb} gives special
    attention to the āyurvedic literature.} \gls{sarala}, \gls{bāhlīka},
    \gls{kuśa}, \gls{āmra}, \gls{sarṣapa}, \gls{varuṇa}, \gls{plakṣa},
    \gls{nicula}, \gls{vardhamāna}, \gls{vañjula}, \gls{putraśreṇī},
    \gls{saptaparṇa}, \gls{ṭuṇṭuka}, \gls{elavāluka},
    \gls{nāgadantī},\footnote{\Dalhana{5.6.3}{580} glossed
        \dev{nāgadantī} as a type of \dev{indravāruṇī} (\gls{indravāruṇī}),
        but he noted that Jejjaṭa had thought it was \dev{dantī}
        (\gls{dantī}).} \gls{ativiṣā}, \gls{bhadradāru}, \gls{marica},
        \gls{kuṣṭha}, and \gls{vacā}.\footnote{\Dalhana{5.6.3}{580} noted
            that Gayadāsa omitted several of the above ingredients, keeping
            thirty.}  Once it has been brought to the boil with the alkali, one
            should take it down and place it in a iron
            pot.\footnote{\label{kṣārapāka2}\Dalhana{5.6.3}{580} explained that
                the above substances, from pepper onwards, should be placed in liquid
                alkali and then cooked until they are neither too runny nor too
                viscous (a phrase he copied from \Su{1.11.11}{47}).  The preparation
                of \dev{pāka} is particularly common in the \SS\ and the \AH.  Cf.\
                the very similar ingredients and procedure in the chapter on alkali
                preparations, \SS\ \Su{1.11.11}{46--47}, p.\,\pageref{kṣārapāka}
                above.}
                
\item[4]    

One should smear this onto a drum as well as onto flags and
\diff{blankets}.\footnote{The vulgate has \dev{toraṇa} “gateways”
    instead of \dev{āstaraṇa} “blankets.”} One is released from all
    poisons as a result of seeing and hearing these.    

\item[5]

This is called “The Caustic Antidote.” It should be given in cases of 
\se{śarkarā}{urinary stones}, \se{aśmarī}{stone}, hemorrhoids, 
\se{vātagulma}{wind-swelling}, 
cough, stabbing pain and \se{udara}{swollen belly}.  It should be given
for indigestion, \se{grahaṇa}{}

\footnote{\volcite{1}[808--809]{josi-maha} gives references for \dev{śarkarā} 
as sugar \citep[see also][5-7--508]{meul-1974}, and also as an ailment.}
    
    % got to here.
\end{translation}
