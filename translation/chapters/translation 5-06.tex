%d !TeX root = ../incremental_SS_Translation.tex
\chapter{Kalpasthāna 6: Rats and Rabies}
\label{mūṣikā}


\section{Introduction}

A notable macro-difference between the vulgate and the Nepalese
versions of the \SS\ is that this chapter and the next are reversed
in the vulgate.  In the Nepalese version, this is chapter six and the
chapter on antitoxic drumming is chapter seven.\footnote{See
    p.\,\pageref{kalpa-chapter-sequence} above.}

\subsection{Mouse or Rat?}

In 2004, Umberto Eco published a characteristically subtle and
enlightening book about translation entitled \emph{Mouse or
    Rat?}.\footcite{eco-2004} The title alluded to Eco's discussion of the
example of translating words for mice and rats across several
European languages that do not always distinguish these animals from
each other, or confuse them in other ways.  In Sanskrit too,
\emph{mūṣikā}, the subject and title of this chapter, does not
distinguish between mouse and rat.  The same is true for MIA and NIA
derivatives.\footcite[\#10258]{CDIAL}  It is hard to know quite how
to translate the term since “rodent” is too broad a term.  In what
follows, I have chosen “rat” for \emph{mūṣikā} in order to produce a
working translation of a text about an animal that is viewed as toxic
and threatening.  ``Mouse'' does not have quite these connotations
for a contemporary English speaker.\footnote{\citeauthor{bhis-1907}
    made the same choice \pvolcite{2}[728--736]{bhis-1907}.}

The rodents that may be described as mice or rats in contemporary
South Asia and that are especially associated with the spread of
disease include the 
house or black rat (\emph{Rattus rattus}, L.), 
the brown rat (\emph{R.\ norvegicus}, Berkenhout), the 
house mouse (\emph{Mus musculus}, L.) and 
bandicoots (\emph{Bandicota}).\footcite[194]{bia} 
Also present in SA are the 
Indian desert gerbille (\emph{Meriones hurrianae}, Jerdon), 
the Indian gerbille (\emph{Tatera indica}, Hardwicke), 
the spiny field mouse (\emph{Mus platythrix}, Bennett),
the Indian field mouse (\emph{M. booduga}, Gray), 
the Metad (\emph{Millardia meltada}, Gray),
the Indian bush rat (\emph{Golunda ellioti}, Gray), 
the longtailed tree mouse (\emph{Vandeleuria oleracea}, Bennett), 
Royle's vole (\emph{Aticola roylei}, Gray), 
the Indian mole-rat (\emph{Bandicota bengalensis}, Gray \& 
Hardwicke),\footnote{“Recent studies\ldots show that the mole-rat forms 
98\% of the total rodent population of Calcutta,” \cite[206]{bia}.} 
the bandicoot rat (\emph{B. indica}, Bechstein), 
the shorttailed bandicoot (\emph{Nesokia indica}, Gray \& Hardwicke),
the whitetailed wood rat (\emph{Madromys blanfordi}, Thomas), 
the bay bamboo rat (\emph{Cannomys badius}, Hodgson),
and other similar rodents.\footcite[ill.\ plates \,45, 46 \emph{et passim}]{bia}
However, plausibly matching these creatures to the Sanskrit names listed in 
this chapter is hard to impossible.\footnote{Mouse-words that we do not see 
in this chapter include the \emph{kirika, giri, girikā} 
group \pvolcite{1}[353, 488, 566]{EWA}.}


\subsection{Literature}

A brief survey of this chapter's contents and reference to the
limited existing research on it to 2002 was provided by
Meulenbeld.\footnote{\volcite{IA}[295--296]{meul-hist}. In addition
    to the translations mentioned by \tvolcite{IB}[314--315]{meul-hist},
    a translation of this chapter was included in
    \volcite{3}[67--77]{shar-1999}. \citet{sekh-2023} omitted mention of
    this type of poisoning, although he discussed rabies, a subsection of
    this chapter.}
    

A rich description of Indian rodents is available by \citet[ch.\,13,
esp.\,205--215]{bia}, including several useful illustration. 
Unfortunately, \citeauthor{bia} rarely gave Indian-language
names.

\newpage

\section{Translation}

\begin{translation}
    
    \item[1] 
    
Now I shall explain the \se{kalpa}{procedure} relating to 
\se{mūṣikā}{rats}.\footnote{The word \dev{mūṣikā} does not distinguish 
between rats and mice.  See Introduction above.}
    
    \item[3] 
    
    Learn concisely about aforementioned eighteen kinds of rats that have 
    poison in their semen, according to their names, characteristics and the 
    herbal treatments.\footnote{Rats with poisonous semen were mentioned 
    in \Su{5.3.5}{5.6.7} (see p.\,\pageref{sukravisa} above).}
    
\subsection{The types of rat}
    \item[4--6]
    
    The eighteen rats are traditionally called,\footnote{\Dalhana{5.6.4}{582} 
    gave no 
    comment on any of these names.  The identifications are 
    mostly guesswork and sometimes whimsical.  The glossary gives lexical 
    discussion of individual names.}
    \begin{enumerate}
        \item \Gls{lālana},
        \item \Gls{putraka},
        \item \Gls{kṛṣṇa},
        \item \diff{\Gls{vasira1}},
        \item \Gls{cikkira}, % Beng Cika = chuchundara, regional
        \item \Gls{chuchundara} % Bengali house shrew Suncus murinus       
        \item \Gls{arala1},% hapax legomenon, Dravidian
\footnote{The word \dev{arala} is a hapax legomenon and has not 
previously been identified as a lexeme because it did not appear in earlier 
editions of the \SS.  It is a loan-word from Dravidian (see glossary).}        
        \item \Gls{kaṣāyadaśana},
        \item \Gls{kuliṅga},
        \item \Gls{ajita},
        \item \Gls{capala},
        \item \Gls{kapila},
        \item the one called \Gls{kokila1} and 
        \item \Gls{aruṇa},
        \item the large black \Gls{unduru}, 
        \item \Gls{śveta1}, together with the
        \item the large \Gls{kapila},
        \item and the \Gls{kapotābha} rat.\footnote{The Nepalese list has 
        \dev{vasira} (\Gls{vasira1}) for the 
        vulgate's \dev{haṃsira}.  The terms \dev{ākhu}, \dev{mūṣikā} and 
        \dev{unduru} are here used as generic names of rat/mouse rodents.}
\end{enumerate}
    
\item[7]

If a part of the body has their sperm fall on it or if they touch it
with their nails or teeth, etc., that have been touched by sperm,
then the blood is corrupted.\footnote{On this, \Dalhana{5.7.7}{582}
    quoted an authority called Ālambāyana who elaborated on this subject
    (see \volcite{IA}[658]{meul-hist} for references to this author of a
    lost treatise on toxicology). Ḍalhaṇa also cited Ālambāyana elsewhere
    on the topics of insects and spiders \pvolcite{IB}[722, note
    5]{meul-hist}. Book 22, tale 543 of the Jātakas includes mention of
    an Ālambāyana who claimed to be a doctor and specialist in snakebite
    poisons: \emph{nāhaṃ dijādhipo homi, na diṭṭho garuḷo mayā, āsīvisena
    vitto ti vejjo maṃ brāhmaṇaṃ vidū ti 793}
    (\volcite{6}[181]{faus-1877}, tr.\  \volcite{6}[95]{cowe-1895}). In
    the same tale, there is a herbal “Ālambāyana mantra” given to an 
    ascetic by a Garuḍa who has just caught and eaten a Nāga, thus
    invoking the Garuḍa-snake-poison motif
    \pvolcite{6}[93--94]{cowe-1895}.  The Jātakas were translated into
    Chinese in the third century \CE.
%    
%    Pāli text: \volcite{6}[177\,ff.]{faus-1877}
%    
%    \pvolcite{6}[93--94, 95--98, 99]{cowe-1895}
%    
    See further discussion by \citet[33--34]{slou-2016}, who calls the mantra 
    “Alampāyana,” adopting the reading of the Burmese MS Bd against the 
    Fausbøll's critical reading  “Ālambāyana”  \pvolcite[see]{2 \& 3}[ 
    Preliminary remarks 3 and 7]{faus-1877}.}
    
\item[8--10ab]

It happens that there are \se{granthi}{lumps}, swellings,
\se{karṇṇikā}{small ear-like growths} and rings, accumulations of
severe \diff{\se{piṭaka}{blisters}}, \se{visarpa}{spreading rashes}
and \se{kiṭibha}{dark, rough patches of skin}.\footnote{“Little ears”
    was strikingly described by \Dalhana{5.7.8}{582} as looking like the
    seed pod in the middle of a lotus (\dev{kamalamadhyabījakośākṛtiḥ}),
    a graphic image (see also \Dalhana{5.8.136}{594}). %    lumps of
    % flesh
    % that look like
    %    the ears of a lotus (\dev{karṇikā māṃsakandī,
    %    padmakarṇikākāratvātkarṇkikā bhaṇyate}).
    The Nepalese version has \dev{piṭaka} “blisters” for the
    vulgate's \dev{pīḍaka} “boils” (itself perhaps a typo for
    \dev{piḍaka}).  \dev{kiṭibha} “dark rash” was described by
    \Dalhana{1.11.7}{46} as a kind of \dev{kuṣṭha}, which is
    variously a skin disease of pallor, leucoderma, or leprosy
    \citep{emme-1984}. But it was described in the \CS\ as being dark
    and as rough as a callous to the touch (\Ca{6.7.21cd--22ab}{451})
    \pvolcite{1}[208]{josi-maha}.}  There are severe conditions such
    as pain in the joints, pain, fever, fainting, weakness, loss of
    appetite, exhaustion, nausea and
    horripilation.\footnote{\dev{parvabheda} “pain in the joints” was
        glossed by \Dalhana{5.7.9}{582} as “spots on the joints”
        (\dev{sandheḥ sphoṭaḥ}).  This seems unlikely, since symptoms on the 
        surface of the body were described in the previous verse, and also 
        because of the obvious etymological meaning of the compound.}
        

This is a concise description of the appearance of someone who has been 
bitten.  Now listen to a longer version. 

\subsection{Detailed symptoms}

\begin{itemize}
\item[10cd--11ab]

The \Gls{lālana} causes a flow of saliva, vomiting and hiccups.  For that, one 
should lick a paste of \gls{taṇḍulīyaka} with honey. 

\item [11cd--12]

The \Gls{putraka} causes the limbs to droop and creates a pale
\diff{beauty},\footnote{The expression \dev{-valgu} “beauty” in the
    Nepalese MSS, for the vulgate's simpler \dev{-varṇa} “complexion,” is
    unusual.} and the body is heaped with lumps like the young of a
    rat.\footnote{The grammar here is very loose. \dev{śiśur} cannot
        stand outside the compound, which should read
        \dev{mūṣikaśisusaṃsthitaiḥ}.  The vulgate text has the simpler and
        grammatical \dev{ākhuśāvakasannibhaiḥ} “resembling the offspring of a
        rat.”}  One should lick \gls{śirīṣa}, \gls{iṅgudi} and \gls{patra}
        with honey.\footnote{\Dalhana{5.7.11-12}{582} here cited a passage 
        by
            an unknown author called Nāgārjuna, about the visible symptoms of a
            bite by this kind of rat (cf.\ \cite[45--46]{shar-1982},
            \volcite{IB}[497, note 100]{meul-hist}) as well as variant readings
            by Gayadāsa and Jejjaṭa on the exact formulation of the lickable
            medication.}

\item [13]

The \Gls{kṛṣṇa} causes one to vomit blood, especially when the
weather is bad.  One should drink \gls{śirīṣa} and \gls{patra},  
with \gls{kuṣṭha} and \gls{elā}, with the 
\gls{kiṃśuka} ashes.\footnote{\Dalhana{5.7.13}{583} explained “with the 
ashes of \gls{kiṃśuka}” as “water with the ashes of \gls{kiṃśuka}.”}

\item [14]

The \Gls{vasira1} causes a person have a revulsion for food, to yawn,
and makes their body-hair \diff{leprous}.\footnote{The qualifier
    \dev{kuṣṭhatā} (\dev{romṇāṃ}) is odd; the vulgate's \dev{harṣaṇa}
    “horripilation” reads more easily. \dev{kuṣṭha} has a lesser-known
    meaning “prominent part, mouth or opening” which might perhaps be
    considered here, though it is hard to see how.}  They should drink
    items like \gls{āragvadha} and be quickly made to vomit.

\item[15]

The \Gls{cikkira} causes headache, swelling, hiccups and nausea.  One
should have thorough emesis using
decoctions\sse{kvātha}{decoction} of \gls{jālinī}, and he should drink
the juice of \gls{aṅkolla}.

\item[16]

The \Gls{chuchundara} causes 

\end{itemize}
% got to here

    \end{translation}
\endinput

%In \As{6.40.35}{844}, Ālambāyana is said to be the authority who 
%declared that the seven \se{vega}{pulses} of toxic shocks affect, 
%successively, the seven \se{āśraya}{substrata} of the body, from blood to 
%semen.    

% Ālambāyana on 5.7.7: “śukrēṇātha purīṣēṇa mūtrēṇāpi nakhais tathā| 
% daṁṣṭrābhir vā kṣipantīha [2] mūṣikāḥ pañcadhā viṣam”
%    tantrāntare: “garbhiṇyā mūṣikayā daṣṭē amlādidōhadaḥ, 
% r̥tumatyā daṣṭē raktamēhanamādhmānaṁ ratiśīlatā ca”

% SS ka 8.24, ka 8. 83-84, ka 8. 120

% Mādhavanidāna 69 viṣaroga 21-24,  Madhukośa, Vijayarakṣita & 
%Śrīkaṇṭhadatta: “naiti raktaṁ kṣatādyasya latāghātairna 
%rājikāḥ| na lōmaharṣaḥ śītādbhirvarjayēttaṁ viṣārditam||”
%
%28, Ātaṅkadarpaṇa “sīdanti kēśalōmāni tasmin pakvāśayaṁ gatē|”

