%d !TeX root = ../incremental_SS_Translation.tex
\chapter{Kalpasthāna 7: Beating Drums}
\label{dundubhi}

\section{Introduction}

This chapter is numbered 7 in the Nepalese version, but 6 in the vulgate.

\subsection{Literature}

A brief survey of this chapter's contents and a detailed assessment
of the existing research on it to 2002 was provided by
Meulenbeld.\footnote{\volcite{IA}[295]{meul-hist}. In addition to the
    translations mentioned by \tvolcite{IB}[314--315]{meul-hist}, a
    translation of this chapter was included in
    \volcite{3}[61--66]{shar-1999}.} 
%    Magic practices (kṛtyā) with a drum (dundubhi) are already mentioned in 
%the
%    \emph{Atharvaveda} (5.31.7).\footnote{\volcite{IB}[400]{meul-hist}.}
%    
%    See on thedundubhi: V.R.R. Dikshitar (1987): 379-380; S.R. Kul-
%    shrestha (1994): 113-114.

\section{Translation}

\begin{translation}
    
    \item[1] 
    
Now I shall explain the \se{kalpa}{procedure} on the topic of
sounding the \se{dundubhi}{kettle drum}.\footnote{This title suggests
    that the chapter may once have begun with the words “the drums are to
    be sounded" or at least that this is the subject of the chapter
    (Pāṇini 4.3.87).
    
On the translation “kettle drum” see \cites[318]{hopk-1889}{ross-2014}.}
    
    \item[3] 
    
One should take the ash of the following items, mix it with cows'
urine and an \se{kṣāra}{caustic} compound, take an extract and cook
it thoroughly: % deep  breath ...
\gls{dhava}, \gls{aśvakarṇa}, \gls{tiniśa}, \gls{picumarda},
\gls{pāṭalī}, \gls{pāribhadraka},\footnote{\label{drum-detox}The
    ingredients to this point are similar to the water-detoxifier
    described in \SS\ \Su{5.3.9}{568}, p.\,\pageref{water-detox} above.}
    \gls{udumbara}, \gls{karaghāṭaka}, \gls{arjuna}, \gls{sarjja},
    \gls{kapītana}, \gls{śleṣmātakā}, \gls{aṅkoṭha}, \gls{kuṭaja},
    \gls{śamī}, \gls{kapittha}, \gls{aśmantaka}, \gls{arka},
    \gls{ciribilva}, \gls{mahāvṛkṣa}, \gls{arala}, \gls{madhuka},
    \gls{madhukaśigru}, \gls{śāka}, \gls{gojī},
    \gls{bhūja},\footnote{Note the unanimous Nepalese MS reading
        \dev{bhūja}, the Middle Indo-Aryan form of Sanskrit \dev{bhūrja}
        \citep[\#9570]{CDIAL}.} \gls{tilvaka}, \gls{ikṣuraka},
        \gls{gopaghoṇṭā}, and \gls{arimeda}.
     
One should add to this the powder of the following items, together
with an equal quantity of metals: \gls{pippalī}, \gls{pippalīmūla},
\gls{taṇḍulīyaka}, \gls{varāṅga}, \gls{coraka}, \gls{mañjiṣṭhā},
\gls{karañjikā}, \gls{hastipippalī}, \gls{viḍaṅga},
soot\sse{gṛhadhūma}{soot}, \gls{ananta}, soma,\footnote{The
    literature on the identification of Soma is large and continuing
    \parencites[76--78, 125--131]{wuja-2003}{clar-2017}. To the cited
    literature, the useful historical discussion by
    \citet[449--455]{gvdb} gave special attention to the āyurvedic
    literature.  Its presence in this recipe may add special value or
    power to the resulting compound.} \gls{sarala}, \gls{bāhlīka},
    \gls{kuśa}, \gls{āmra}, \gls{sarṣapa}, \gls{varuṇa}, \gls{plakṣa},
    \gls{nicula}, \gls{vardhamāna}, \gls{vañjula}, \gls{putraśreṇī},
    \gls{saptaparṇa}, \gls{ṭuṇṭuka}, \gls{elavāluka},
    \gls{nāgadantī},\footnote{\Dalhana{5.6.3}{580} glossed
        \dev{nāgadantī} as a type of \dev{indravāruṇī} (\gls{indravāruṇī}),
        but he noted that Jejjaṭa had thought it was \dev{dantī}
        (\gls{dantī}).} \gls{ativiṣā}, \gls{bhadradāru}, \gls{marica},
        \gls{kuṣṭha}, and \gls{vacā}.\footnote{\Dalhana{5.6.3}{580} noted
            that Gayadāsa omitted several of the above ingredients, keeping
            thirty.}  Once it has been brought to the boil with the alkali, one
            should take it down and place it in a iron
            pot.\footnote{\label{kṣārapāka2}\Dalhana{5.6.3}{580} explained 
            that
                the above substances, from pepper onwards, should be placed in 
                liquid
                alkali and then cooked until they are neither too runny nor too
                viscous (a phrase he copied from \Su{1.11.11}{47}).  The 
                preparation
                of \dev{pāka} is particularly common in the \SS\ and the \AH.  Cf.\
                the very similar ingredients and procedure in the chapter on alkali
                preparations, \SS\ \Su{1.11.11}{46--47}, p.\,\pageref{kṣārapāka}
                above.}
                
\item[4]    

One should smear this onto a drum as well as onto flags and
\diff{carpets}.\footnote{The vulgate has \dev{toraṇa} “gateways”
    instead of \dev{āstaraṇa} “carpets.”  On the meaning of the latter
    term, see \cite[31, 33 \emph{et passim}]{bail-1970} and the remarks
    of \tvolcite{1}[390--391, note 171]{rotm-2008}.} One is released from
    all poisons as a result of \diff{seeing and hearing}
    these.\footnote{The vulgate adds “and touching” \Su{5.6.4}{580}. Note
        the ditransitive (\dev{dvikarmaka}) \dev{-mucyate}; cf.\
        \emph{Meghadūta}, uttaramegha 33 \citep[\dev{71}, 120]{kale-1947}.}

\item[5--6]

This is called “\se{kṣārāgada}{The Caustic Antidote}”.\footnote{Cf.\
    \Ca{4.23.95--104}{575--576}.} It should be given in cases of
    \se{śarkarā}{small urinary stones}, \se{aśmarī}{urinary
        stones},\footnote{\dev{aśmarī} and \dev{śarkarā} are described in
        \SS\ \Su{2.3}{276--280}, the latter being smaller and more easily
        expelled (\Su{2.3.13cd--14}{279};  cf.\ \volcite{1}[67--68,
        808--809]{josi-maha}). The commentators Cakrapāṇidatta and Ḍalhaṇa
        discussed the lack of a firm distinction between these categories.}
        hemorrhoids, \se{vātagulma}{wind-swelling}, cough,
        \se{śūla}{abdominal gripes} and \se{udara}{swollen belly}.  It should
        be given for indigestion, \se{grahaṇīdoṣa}{humours of the
            abdomen},\footnote{On the organ called \dev{grahaṇī}, see the useful
            summary by \tvolcite{2}[20--21, 96 \emph{et 
            passim}]{rama-1985}.} 
            and
            severe \se{bhaktadveṣa}{aversion to food},\footnote{A sign of
                impending death according to \SS\ \Su{1.32.4}{142}.} in swelling,
                \se{sarvasara}{mouth ulcer},\footnote{See 
                \volcite{1}[888]{josi-maha}
                    and \SS\ \Su{2.16.65--66}{336} and \Su{4.23.3}{}.} and 
                    persistent
                    \se{śvāsa}{asthma}. %
                    % give
                    %    references for \dev{śarkarā} as sugar \citep[see
                    %    also][5-7--508]{meul-1974} but also as an
                    % ailment that
                    % is a
                    %    subdivision of \dev{aśmarī}.}

\item[7]
                    This is to be employed in all cases where someone is
                    suffering as a result of any poison.  Thus, it is the
                    antidote that is  the \se{sarpāṅkuśa}{Snakes'
                        Controlling Hook} even for the snakes led by
                    Takṣaka.\footnote{\dev{takṣaka} is an ancient name
                        for a Nāga, mentioned in the \emph{Kauśikasūtra}
                        \citep[28.1 \emph{et passim},][78]{bloo-1890}.
                        Takṣaka is mentioned briefly in the \emph{Rāmāyaṇa}
                        \citep[292, n.\,13]{poll-1991} and more in later
                        works. See further, \cite[22, 26, 37, \emph{et
                        passim}]{slou-2016}.  The
                        \emph{Kriyākālottaratantra}, edited by Slouber,
                        contains a similar sentence (7.26cd, p.\,232): “Even
                        someone bitten by Takṣaka will be rapidly cured of
                        poison.”}\textsuperscript{,}\footnote{There follow
                            four verses in the vulgate, 8--11, that are not
                            present in the Nepalese version.  These list
                            ingredients that form a ghee called
                            \se{kalyāṇaka}{The Salutary}. This ghee recipe with
                            the same name is also present in the
                            \emph{Uttaratantra} at \Su{6.39.229--232}{689}, where
                            it is a treatment for mostly similar ailments:
                            chronic fever, asthma, cough, swelling, madness and a
                            \se{gara}{toxic potion} (defined at
                            \Su{5.8.24cd--25ab}{587} as something manufactured,
                            \dev{kṛtrima}).  However, in the Nepalese version at
                            6.39.232, the vulgate statement of this name
                            “\dev{etatkalyāṇakaṃ nāma sarpirmāṅgalyamuttamam}” is
                            not present. Thus, in the Nepalese version,
                            \se{kalyāṇaka}{The Salutary} is not named. The same
                            named ghee also appears in the \CS\ at
                            \Ca{6.9.35--42ab}{471}, where it is presented as a
                            treatment for \se{unmāda}{madness} as well as many
                            other ailments including those mentioned above in the
                            \SS\ (excluding swelling); it is possible that this
                            is a case where a text from the \CS\ was added to the
                            \SS\ after the Nepalese version.}
        
    
\item [12--13]    

Grind \gls{apāmārga} seeds and the beans of \gls{śirīṣa}, the two 
\glsplural{śvetā} and \gls{kākamācī} with cows' urine.\footnote{On the BHS 
form \dev{pīṣayet}, see \volcite{2}[346]{edge-1953}, \volcite{1}[\S 
28.4, p.\,220]{edge-1953}.}  A ghee mixed with 
these is the most effective means of soothing poison.  It is famous under the 
name “Immortal (Amṛta).”  It can revive even the dead.


\item [14--23] 

Collect together the following requisites:

\gls{candana}, \gls{aguru},  \gls{kuṣṭha},  \gls{tagara},  
\gls{tailaparṇika}, 
\gls{prapauṇḍarīka},  
\gls{nalada},  
\gls{sarala},  
\gls{devadāru}, 
\gls{bhadraśriya},  
\gls{yavaphalā},  
\gls{bhārgī}, 
\gls{nīlī},  
\gls{sugandhikā}, 
\gls{kāleyaka},  
\gls{padmaka},  
\gls{madhuka},  
\diff{\se{sanakha}{thorny}} \gls{jaṭā}, 
\gls{punnāga},
\gls{elā},  
\gls{elavālu},  \gls{gairika},  \gls{dhyāmaka},  
\gls{toya},  
\gls{sarjarasa}, 
\gls{māṃsī},  
% got to here
\gls{śatapuṣpā},  
\gls{hareṇukā}, 
\gls{tālīsapatra},   
\gls{kṣudrailā},   
\gls{priyaṅgū},   
\gls{sakuṭa},   
\gls{naṭā},  
\gls{tilapuṣpa},   
\gls{saśaileya},   
\gls{patra},  
\gls{kālānusārivā},  
\gls{kaṭutrika},   
\gls{śītaśiva},   
\gls{kāśmarya},   
\gls{kaṭuro Lhiṇī},  
\gls{somarājī},   
\gls{ativiṣā},  
\gls{pṛthvīkā},  
\gls{indravāruṇī}, 

\gls{uśīre dve},  
\gls{varuṇaka}, ṃ 
\gls{kustumvurya},  
\gls{nakhāni},  


\gls{tvac},  
\gls{taskarasāhya},  
\gls{granthilā},  
\gls{saharītakī}, 


\gls{śvete},  
\gls{haridre}, 
\gls{ sthoṇeya},  
\gls{lākṣā},   

\gls{lavanāni}, 


\gls{kumuda},  
\gls{utpala},  
\gls{padma},  
\gls{puṣpa},  
\gls{arjaka}, 


\gls{campaka},  
\gls{aśoka},  
\gls{sumanā},  
\gls{tilakaprasavāni}, 


\gls{pāṭalī},  
\gls{śālmalī},  
\gls{śelū},  
\gls{śirīśāṇā},  


\gls{surasyā},  
\gls{tṛṇaśūlya},  
\gls{sinduvāra}, 


\gls{dhavā},  
\gls{aśvakarṇṇa},  and 
\gls{tiniśa}.

Having collected these ingredients, have a fine powder of them made and 
place them in a horn together with cow's bile, honey and ghee.

\item [24]

This foremost antidote can rescue a man, whose back is bent and whose 
eyes are rolling, from within the jaws of death.

\item [25]

This antidote is like fire, irresistible to the angry, infinitely ardent 
\diff{progress} of 
all the snakes.  It destroys even Vāsuki's poison.\footnote{This Nepalese MSS 
unanimously read \dev{sarvanāgagati} “the progress of all the snakes” for 
the vulgate's \dev{viṣaṃ nāgapater} “the poison of the king of snakes.”  The 
latter reading is much easier but is not supported by the Nepalese witnesses.}

\item [26]

Out of all the royal antidotes, this one, called The Great Perfume
(\emph{Mahāsugandha}), assembled out of eighty-five components, 
should always be in the king's hand.

\item [27]

A king anointed \diff{with this} will become beloved of all the people.  He 
becomes refulgent even when surrounded by his enemies. 

\item [28]

For those afflicted by poison, the expert should apply a therapy that
avoids heat.  The exception is insect poison, because coldness
makes that grow.\footnote{Verses 29 and 30 of the vulgate, giving dietary 
advice, are not present in the Nepalese version.}

\item[31]

Someone suffering from poison should avoid 
sleeping during the day, sexual intercourse, exercise, anger, the heat of the 
sun, \se{surā}{wine}, \gls{tila} and 
\gls{kulattha}.\footnote{\Dalhana{5.6.31}{581} took the “and” in this 
sentence to mean the inclusion of a list of additional avoidances, from 
\gls{pippalī} to 
\glspl{śiśumāra} and \glspl{kūrma}.}


\item[32]

A physician can recognize that a person is free of poison if their 
humours are clear, if their \se{dhātu}{tissues} are in a normal state, 
if they have an appetite, if their urine and feces are \se{sama}{normal}, and
if \diff{the movement of their senses and mind are clear}.\footnote{This 
verse is much clearer in the Nepalese version.  The vulgate seems to have 
acquired corrupted readings before the time of Ḍalhaṇa.}

\end{translation}
