% !TeX root = ../incremental_SS_Translation.tex
\chapter{Kalpasthāna 8: Poisonous insects}

\section{Introduction} 

This is the last chapter of the \emph{Kalpasthāna}.  Since all the  colophons 
of the Nepalese manuscripts commonly end with the statement, “here ends 
the \SS\ together with the Uttaratantra,” we can presume that an older 
version of the \SS\ ended with the present chapter.
    
    

\subsection{Literature} 
A brief survey of this chapter's contents and a detailed assessment of
the existing research on it to 2002 was provided by
Meulenbeld.\footcite[IA, 296--299]{meul-hist} 


\section{Translation}



\begin{translation}

\item[1]
 And now I shall explain the \se{kalpa}{procedure} about 
 insects\sse{kīṭa}{insect}.
 
 
\item [3] 

Insects originate from the semen, feces, urine, the rot of corpses, and eggs 
\diff{of snakes}.  Their natures are traditionally divided into \diff{three}: 
wind, fire, and water.

\subsection{Taxonomy of insects}

\item[3--17ab] xx 

\subsection{Symptoms}

\item[17cd--24] xx

\subsection{Taxonomy according to symptoms and prognosis}

\item[25--27] xx

\item [28]  \gls{godheraka}    \label{godheraka}
    
\item [29] \footnote{See n.\,\ref{galagodika}, 
p.\,\pageref{galagodika}.}


\item[30--41] xx

\subsection{Therapies}

\item[42--56abcd] xx
 
\subsection{Taxonomy of scorpions}
 
 \item [56ef--66] xx
 
 \subsection{Therapies for scorpion-sting}
 
 \item[67--74] xx
 
 \subsection{Symptoms of spider poisoning}
 
 \item[75--89] xx
 
 \subsection{Origin story for spiders}
 
 \item[90--93] xx
 
 \subsection{Taxonomy of spiders}
 
 \item[94--100ab] xx
 
 \subsection{Specific symptoms and treatment for spider poisoning}
 
 \item[100cd--120] XX
 
 \subsection{Untreatable spider poisons}
 
 \item [121--127] xx
 
 \subsection{Curable and incurable}
 
 \item[128--129] xx
 
 \subsection{Therapies for spider poisoning}
 
 \item [130--134] xx
 
\subsection{General therapies for poisoning}

\item [135--139] xx

\subsection{End of the Suśrutasaṃhitā}

\item[140--143] xx
 
\end{translation}