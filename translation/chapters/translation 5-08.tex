% !TeX root = ../incremental_SS_Translation.tex
\chapter{Kalpasthāna 8: Poisonous insects}

\section{Introduction} 

This is the last chapter of the \emph{Kalpasthāna}.  Since the
chapter-colophons of the Nepalese manuscripts of the whole
\emph{Suśrutasaṃhitā} commonly end with the statement, “here ends the
\SS\ together with the \emph{Uttaratantra},” we can presume that an older
version of the \SS, sans \emph{Uttaratantra}, ended with the present chapter.
Added to this, the beginning of the next section of the work, the
\emph{Uttaratantra}, reads,
\begin{quote}
It being declared in the preceding 120
chapters, from here on, in the latter section, I shall explain the
meanings in detail, fully.\footnote{Note that this is not the reading
    of the vulgate, which says that the \emph{Uttaratantra} will explain
    everything that was \emph{not} completely explained before.}  Now, I
    shall explain the treatise called “the latter” where diseases
    in their diversity are fully revealed.
    \end{quote}
It is often the case with evolving works that new chapters are added
at the start or, especially, at the end of a work.  This has been true
since the \emph{Ṛgveda}.  The \emph{Kalpasthāna} has a different character
from the rest of the \SS, for example eschewing theoretical
considerations in many situations.  It may therefore itself have once
been an addition to an even earlier medical work consisting of four main
divisions.

\subsection{Insect names}

It is more than usually difficult to equate the Sanskrit names of
insects with contemporary creatures.  In fact, it is mostly
impossible. This is partly, at least, because historical entomology is
non-existent as a discipline. Furthermore, entomology as a science in
South Asia is dramatically undeveloped when compared, for example,
with botany.\footnote{\citet{desm-1992} devoted a book of 368 pages to
    the early history of Indian botany; \citet[338--345]{dove-1922}
    described the history of Indian entomology in seven pages.}  There are
few general surveys of insects in India and virtually none that record
historical names or literary references.  In the twelfth century,
Ḍalhaṇa made the following remark about the commentators who lived
before his time:
\begin{quote}
These different types of insects are not described by commentators
like Suvīra, Nandin,Varāha, Jejjjaṭa and Gayadāsa, so they have to be
identified from the people of different
localities.\footnote{\Dalhana{5.8.4}{586}: \dev{ete kīṭakabhedā
    nānādeśīyalokādavagantavyāḥ, yataḥ
    suvīranandivarāhajejjaṭagayadāsādibhiḥ ṭīkākārairna vyākhyātāḥ}.
    (Varāha is called Vārāha by \Dalhana{2.13.3}{318}.) Cf.\
    \tvolcite{IA}[387--388]{meul-hist} on Suvīra and \emph{mutatis
mutandis} on
    the other commentators}
\end{quote}
Thus, even pre-modern Sanskrit authors were not expert regarding the
identities of the insects discussed in the
\SS.\footnote{\cite{moni-sans} includes 191 insect names, almost none
    of which are identified.}  
    
In general the names listed in passages 5--14 are the  least
recognizable.  Most seem never to appear elsewhere in Sanskrit
literature or even elsewhere in the \SS.  The names mentioned from
passages 25 onwards are mostly recognizable and do appear elsewhere
Sanskrit literature.\footcite[E.g.,][]{mitr-2005} This chapter
therefore gives the appearance of having two distinct parts.  First,
there is a taxonomy arranged according to humoral characteristics,
containing otherwise unknown insect names. Second follows a
concatenated treatise with more recognizable ordinary-language
nomenclature coupled with creature-by-creature nosology and therapy.

%\ignoreargument{agniprabhā, agniraja,
%        agnirajas, agnika, apacit, aparājita, abhīrājī, arimedaka, alpaśayu,
%        avamūtrita, avalgulī, avaskaraka, āgneya, āvartaka, indragopa,
%        utkleśaka, uttiṅga, utpādikā, upādika, urabhrasārikā, ṛṇamatkuṇa,
%        ejatka, eṇīpadī, aindrādṛśa, aindrāliśa, kaṇḍūmakā, kambala,
%        karṇakīṭā, karṇakīṭī, karṇasūci, kaṣkaṣa, kaṣāyavāsika, kāṣāyavāsika,
%        kāṣṭhakīṭa, kiṇa, kiṃtanu, kīṭa, kīṭī, kīṭagardabhaka, kīṭāvapanna,
%        kīṭaka, kuṇa, kunta, kumārikā, kumbhin, kumbhīnasa, kuṣumbha,
%        kuṣumbhaka, kṛmi, kṛmikara, kṛmitā, kṛmirāga, kṛmisarārī, kṛṣṇā,
%        kṛṣṇagodhā, kṛṣṇatuṇḍa, keśakīṭa, keśāda, kaiṭa, kokila, koṭika,
%        koṭira, kośakārī, kośastha, kauṇḍilyaka, klīta, kṣārakīṭa, kṣīrakīṭa,
%        khaga, khajyotis, khadyotā, kharju, kharjū, garādhikā, guhyadīpaka,
%        golikā, ghuṇa, ghuṇapriyā, ghuṇavallabhā, ghuṇākṣara, cicciṭiṅga,
%        citraśīrṣaka, cipiṭa, jatukṛt, janturasa, jalamakṣikā, taṇḍurīṇa,
%        tarda, talpakīṭa, tālaka, tuṅganābha, tuṅgīnāsa, tuṇḍikerin,
%        tailakīṭa, toṭaka, trikaṇṭaka, troṭaka, daṃśa, daṃśanāśinī,
%        dadrunāśinī, darbhakusuma, darśanī, divī, dundubhika, dehikā,
%        dyotiriṅgaṇa, dhundhumāra, nīlamīlika, pañcakṛṣṇa, pañcaśukla,
%        pañcāla, pañcālaka, pataṃga, padmakīṭa, pallavita, pākamatsya,
%        pāṭalakīṭa, picciṭa, picciṭaka, pulaka, puṣpakīṭa, pūtikīṭa, pūtyaṇḍa,
%        prajāpati, prabhākīṭa, pralūna, prāvārakīṭa, pluṣi, balabha, bindula,
%        bhadrapīṭha, makara, maṇḍalapucchaka, madyakīṭa, madhukeśaṭa,
%        mayūrikā, yavāṣa, yāva, yevāṣa, raktarāji, raktarājī, raktavarṇa,
%        rasaja, lākṣā, lākṣātaru, lomakīṭa, vajrakīṭa, vaṭi, vaṭibha, vāhaka,
%        vāhyakī, vināsaka, vināsika, vināsikā, vicilaka, viḍbhuj, vedamukhya,
%        vaidala, śakragopa, śakragopaka, śakunta, śatadāruka, śatapattraka,
%        śambuka, śaluna, śādvalābha, śūkā, śūkadoṣa, śūkavṛnta, śūrakīṭa,
%        śvetakṛṣṇā, ṣaṭpada, ṣaḍbindu, ṣaḍvindhyā, sarabhaka, sarāva,
%        sarvaśvetā, sarṣapakī, sarṣapika, sārikāmukha, simīka, sīmika,
%        sūkṣmatuṇḍa, sūkṣmaṣaṭcaraṇa, sūcīmukha, sūcīka, saukṣmaka, saurasa,
%        sauvarṇikā, halagolaka, hastikakṣa}

\subsection{Literature} A brief survey of this chapter's contents and
a detailed assessment of the existing research on it to 2002 was
provided by Meulenbeld.\footcite[IA, 296--299]{meul-hist} 

The early history of entomology in India was fragmented until the
study of \citet{maxw-1909} who provided a comprehensive and well
illustrated reference compendium. \citet{dove-1922} gave an overview
of the early years of the field, though he admitted that, “I have not
the linguistic attainments to discuss the mention of various insects
in ancient Sanskrit works.”   Entomological studies focussed on south
India include those of \citet{bain-1914} and \citet{rama-1963}.
\tvolcite{IB}[402]{meul-hist} provided short bibliographies on Indian
scorpions (note 214) and on spiders (note 222). Some insects were
included by \citet{ball-1888} in his study of the Indian flora and fauna
known to classical Greek authors. \citet{kaur-2018} provided a unique
but very brief historical sketch of some arthropod references in
Sanskrit literature.

\newpage
\section{Translation}



\begin{translation}

\item[1]
 And now I shall explain the \se{kalpa}{procedure} about 
 insects\sse{kīṭa}{insect}.
 

\subsection{Taxonomy of insects}

%\item[3--17ab] 

\item [3] 

Insects originate from snakes' semen, feces, urine, the rot of corpses,
and eggs.\footnote{\tvolcite{3}[78]{shar-1999} omitted “snakes'\,” making it 
sound as if insects are just born of any semen, etc.}  Their
\sse{prakṛti}{character}characters are traditionally divided
into \diff{three}: wind, fire, and water.

\item[4]

Yet others hold the opinion that they are connected with the
\sse{prakṛti}{character}characters of all of the humours. And those
insects are also very fierce and all of them are divided into four
groups.\footnote{The insects named in the following lists are all
    unidentifiable at the present time.  The English translations are
    based mostly on the etymologies of the Sanskrit names.  Future
    ethno-linguistic studies of insect-names in South Asia may solve some
    cases.}

\subsubsection{Wind}

\item[5--6]
\begin{multicols}{2}
\begin{enumerate}
    \item \Gls{uṇḍunābha},
    \item \Gls{tuṇḍikerī},
    \item \Gls{śṛṅgī}, and
    \item \Glspl{śatakulimbhaka},
    \item \Gls{ucciṭiṅga},
    \item \Gls{agni-insect},
    \item \Gls{alpavāca},
    \item \Glspl{viciṭiṅga},
    and
    \item \Glspl{masūrika-insect}.
    \item \Gls{āvarttaka}, and
    \item \Gls{urabhra-insect},
    \item \Gls{śārikāmukha}, and
    \item \Gls{vaidala},
    \item \Gls{śatakurda},
    \item \Gls{abhirājī},
    \item \Gls{paruṣa},
    \item \Gls{citraśīrṣaka}.\footnote{The list is deficient in the Nepalese version.  
    The
        vulgate text has another half-verse here listing two more names,
        \dev{śatabāhu} “hundred-arm” and \dev{raktarāji} “red-stripe.” It does
        not include the Nepalese version's \dev{alpavāca} “little voice."}
\end{enumerate}
\end{multicols}

\item[7cd--8ab]

These eighteen insects, being of airy character, irritate the
wind.
The diseases of people bitten by one of these are caused by wind.

\subsubsection{Fire}

\item[8cd--11ab]
\begin{multicols}{2}
    \begin{enumerate}
        \item \Gls{kauṇḍinya-insect},
        \item \Gls{kaṇabha},
        \item \Gls{svarga-insect}, and
        \item \Gls{vāraṇī},
         \item \Gls{patravṛścika},
        \item \Gls{vināsikā},
        \item \Gls{brahmaṇīkā},
        \item \Gls{bindula},
        \item \Gls{bhramara},
        \item \Gls{bāhyaka}.
        \item \Glspl{picciṭā}, 
        \item \Gls{kumbhīvarcas}, 
        \item \Gls{kīra-insect}, 
        \item \Gls{arimedaka},
        \item \Gls{padmakīṭa},
        \item \Gls{dundubhaka},
        \item \Gls{maśaka},
        \item \Gls{śatapādaka},
        \item \Gls{pañcālaka},
        \item \Gls{pākamatsya},\label{pakamatsya}
        \item \Gls{kṛṣṇatuṇḍa},
        \item \Gls{gardabhī-insect}.\\
                These are the insects, as well as the 
        \item \Gls{krimisarāvī},\\ and the 
        other one that is known as the
        \item \Gls{śleṣmaka-insect}.       
        \end{enumerate}
    \end{multicols}

\item[11cd--12ab]

These are the twenty-four insects that have the character of fire. 
The diseases of people bitten by one of these are caused by bile.


\subsubsection{Phlegm}

\item[12--15ab]

\begin{multicols}{2}
    \begin{enumerate}
        \item \Gls{vaiśvambhara},
        \item \Gls{pañcaśukla},
        \item \Gls{pañcakṛṣṇa},
        \item \Gls{kokila-insect},
        \item \Gls{śairyaka-insect},
        \item \Gls{pravalāka},
        \item \Gls{bhaṭābha},
        \item \Gls{kiṭibha},
        \item \Gls{aṭakī},
        \item \Gls{sucīmukha},
        \item \Gls{kṛṣṇagodhā},
        \item \Gls{kuṣṭa-insect},
        \item \Gls{kaṣāyavāsika},
        \end{enumerate}
\end{multicols}

These are the thirteen \se{saumya}{watery} insects that irritate the phlegm.
The diseases of people bitten by one of these are caused by phlegm.

\subsubsection{All three humours}

\item[15cd-17ab]
\begin{multicols}{2}
    \begin{enumerate}
        \item \Gls{tuṅgīnāsa},
        \item \Gls{valabhika},
        \item \Gls{tolaka},
        \item \Gls{nāhana},
        \item \Gls{koṇṭāgīrī},
        \item \Gls{krimikara},
        \item \Gls{maṇḍalapuṣpaka},
        \item \Gls{tuṇḍavakra},
        \item \Gls{sarṣapaka},
        \item \Gls{spoṭaka},
        \item \Gls{śambuka},
        \item \Gls{agnikīṭa},
       \end{enumerate}
\end{multicols}

These are the twelve terrible ones that are born of all three humours.

\subsection{Symptoms}

\item[17cd, 20--24] 

The knowledge about the stages of \se{vega}{toxic shock} of those
bitten by one of these is the same as with snakes.\footnote{Two verses
    appear at this point in the vulgate that are not in the Nepalese
    version.  They introduce a categorization of insect poisons into severe versus 
    mild, a scheme that the Nepalese version does not reference.}

The following are found in the area of a bite, or in a body
\se{ākula}{permeated} with poison: an eruption of blisters, swelling,
lumps and circles, \se{dardru}{ringworm},\footnote{More usually
    \dev{dadru}, a skin disease like \dev{kuṣṭha}, i.e., leprosy or
    vitiligo, caused by an excess of bile and phlegm
    \citep[390]{josi-maha}, although the form \dev{dardru} is mentioned in
    the \emph{Uṇādisūtra} commentary by Śvetavanavāsin (fl.\,tenth to
    fifteenth century), “\dev{dardrūḥ kuṣṭhabhedaḥ}” (I.88). Translated
    here as “ringworm” because that is prominent amongst the NIA usages of
    the lexeme and derivatives
    \pvolcite{1}[\#6142]{CDIAL}.}\sse{dadru}{ringworm} % Ca.1.3.7
\se{karṇikā}{small ear-like growths}, \se{visarpa}{spreading rashes},
and \se{kiṭibha}{dark, rough patches of skin}.\footnote{These symptoms
    are the same as those listed at \Su{5.7.8}{582} as being caused by rat
    poisoning, and similar to the list at \Su{1.11.7}{46}.  See footnote
    \ref{karṇika}, p.\,\pageref{karṇika}.}

\subsection{Taxonomy according to symptoms and prognosis}

\item[25--27] xx

\item [28]  \gls{godheraka}    \label{godheraka}
    
\item [29] \footnote{See n.\,\ref{galagodika}, 
p.\,\pageref{galagodika}.}


\item[30--41] xx

\subsection{Therapies}

\item[42--56abcd] xx
 
\subsection{Taxonomy of scorpions}
 
 \begin{figure}
     \centering
     \includegraphics[width=0.7\linewidth]{"media/Scorpions Smithsonian"}
     \caption{Husain, Shaykh, Shaykh Ali and Shaykh Hatim, “Asavari Ragini: 
     Cropped Image of Scorpions” \citep{husa-1591}. Courtesy of the Smithsonian 
     Institution.}
     \label{fig:scorpions-smithsonian}
 \end{figure}
 
 
 \item [56ef--66] xx
 
 \subsection{Therapies for scorpion-sting}
 
 \item[67--74] xx
 
 \subsection{Symptoms of spider poisoning}
 
 \item[75--89] xx
 
 \subsection{Origin story for spiders}
 
 \item[90--93] xx
 
 \subsection{Taxonomy of spiders}
 
 \item[94--100ab] xx
 
 \subsection{Specific symptoms and treatment for spider poisoning}
 
 \item[100cd--120] XX
 
 \subsection{Untreatable spider poisons}
 
 \item [121--127] xx
 
 \subsection{Curable and incurable}
 
 \item[128--129] xx
 
 \subsection{Therapies for spider poisoning}
 
 \item [130--134] xx
 
\subsection{General therapies for poisoning}

\item [135--139] xx

\subsection{End of the Suśrutasaṃhitā}

\item[140--143] xx
 
\end{translation}