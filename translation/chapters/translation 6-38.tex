% !TeX root = incremental_SS_Translation.tex

\chapter{Uttaratantra 38: Diseases of the Female Reproductive System}

\section{Introduction} 

The chapter talks about various diseases of the female reproductive
system and, in doing so, combines both aspects that go into a
representation of diseases in āyurvedic literature: signs, symptoms and
pathogenesis (\emph{nidāna}), on the one hand, and medical treatment
(\emph{cikitsā}), on the other. In chapters of the \emph{Uttaratantra},
these two aspects are sometime dealt with in two different chapters
X-\emph{vijñānīya} and X-\emph{pratiṣedha}. There are, however, many
examples where this distinction is not made.


\section{Literature}

The chapter is summarized, with notes on vocabulary and references to
further research literature, in \volcite{IA}[313]{meul-hist}.
\citep{tiva-1990} dedicated a monograph to this topic, and
\citet{selb-2005,selb-2005b} has explored gyencological narratives in
ayurveda.

\section{Placement of the Chapter}

In the vulgate text (\cite{vulgate}) the current chapter, 6.38, is found
after the Uttaratantra's subsection on paediatrics, the
\emph{Kumāratantra}, see Table~\ref{uttara-sections}.\footnote{Or 
\emph{Kumārabhṛtya} as this section is
    named in \MScite{Kathmandu KL 699}.}  But in the Nepalese version, this
    is chapter 6.58 of the Uttaratantra. And it is also counted as chapter 23
    of the subsection \emph{Kāyācikitsā}.

Several things are noteworthy in this regard:
\begin{itemize}
    \item In the placement of the vulgate, this chapter follows upon 6.37 
    \emph{Grahotpatti} (6.35 in the Nepalese version), a chapter that talks about the 
    origination of nine \se{graha}{demons} that are responsible for all children's 
    diseases described in previous chapters of the \emph{Kumāratantra}. In this way, 
    the 
    current chapter retains the general focus on the \se{kaumārabhṛtya}{child bearing}, 
    but, at the same time, marks a change to a distinct, less mystical approach to the 
    topic at hand (that could originate in a cultural milieu different from that of the 
    preceding eleven chapters). Ḍalhaṇa explained how the chapter fits its context in the 
    following way: 
    \begin{quote}
        It is appropriate that, for the sake of treating the disorders of
the female reproductive system, the chapter called
“Countermeasures Against Disorders of the Female Reproductive
System” is taught immediately after the chapter called “The
Origination of \se{graha}{Demons}.”  It is because (1) there is
an explicit mention of the word “\emph{yoni}” in the statement 
“born in the \se{yoni}{womb} of animal and human” [in
\Su{6.37.13bc}{667}] and because (2) the disorders of the female
reproductive system are the causes for the inborn disorders of
children.\footnote{Ḍalhaṇa on \Su{6.38.1}{668}:
    \dev{grahotpattyadhyāyānantaraṃ ‘tiryagyoniṃ mānuṣaṃ ca}'
    \dev{iti vacanena yonernāmasaṃkīrtanāt
    kumārajanmavikārakāraṇatvācca yonervyāpaccikitsitārthaṃ
    yonivyāpatpratiṣedhādhyāyārambho yujyata [\ldots]/}%
    }
        \end{quote}
   \begin{table}[t] 
       \centering
       \caption{Subdivisions of the Uttaratantra, in the vulgate.}  
       \label{uttara-sections}
       \medskip
       \begin{tabular}{lll}
           \toprule
           Section & Chapters & Internal count  \\
           \midrule
           Śālakyatantra &  1--26   & 1--26\\
           Kumāratantra    & 27--38  & 1--12 \\
           Kāyacikitsātantra & 39--59   & 1--21 \\
           Bhūtavidyātantra  & 60--62   & 1--3\\
           Tantrabhūṣaṇādhyāya & 63--66 & 1--4   \\
           \bottomrule
       \end{tabular}
   \end{table}     
\item In the placement of the Nepalese version,
\begin{itemize}
    \item 6.\emph{Yonivyāpatpratiṣedha} is preceded by 
    \item 6.56 \emph{Mūtrāghātapratiṣedha} (6.58 in \cite{vulgate}) and 
    \item 6.57 \emph{Mūtrakṛcchrapratiṣedha} (6.59 in \cite{vulgate}), two 
    chapters
    dealing with the diseases of the urinary tract.  
    \end{itemize}
The current chapter carries on with the topic of diseases that affect
genitalia. In its Nepalese version, the chapter opens with two verses
that explain the reasons for treating the particular set of diseases.
These lack any reference to the inborn disorders of children, mentioned
by Ḍalhaṇa, and instead highlight the importance of curing female
diseases for the satisfaction of male partner.
        
\item SS.1.3 in both \cite{vulgate} and the Nepalese version lists the
chapter at the place where it is found in the vulgate.\footnote{See 
\Su{1.3.37ab}{15}:
\dev{naigameṣacikitsā ca grahotpattiḥ sayonijā//}.}
        
        \item Parallel chapters in the \emph{Aṣṭāṅgasaṃgraha} and the
        \emph{Aṣṭāṅgahṛdayasaṃhitā} form a part of the \emph{Śalyatantra} section of 
        each
        text.
    \end{itemize} 
    %In the Nepalese version, this is chapter 6.58 (\emph{Kāyācikitsā} 23) that follows 
    %upon 6.56 \emph{Mūtrāghātapratiṣedha} (6.58 in the vulgate) and 6.57 
    %\emph{Mūtrakṛcchrapratiṣedha} (6.59 in the vulgate). In the vulgate, on the 
    %other 
    %hand, this chapter concludes another section, the \emph{Kumāratantra} 
    %(\emph{Kumārabhṛtya} in \MScite{Kathmandu KL 699}), and follows upon 6.36 
    %\emph{Naigameṣapratiṣedha} (6.34 in the Nepalese version) and 6.37 
    %\emph{Grahotpatti} (6.35 in the Nepalese version).
    
    \section{Parallels}
    
The current chapter is parallel in its content to \emph{Aṣṭāṅgasaṃgraha}
6.38 and 6.39 as well as \emph{Aṣṭāṅgahṛdayasaṃhitā} 6.33 and 6.34
(\emph{Guhyarogavijñāna} and \emph{Guhyarogapratiṣedha} respectively).%,
%which form a part of the \emph{Śalyatantra} section of each text.
    
    A close literary parallel to the first part of the chapter is found
in \emph{Mādhavanidāna} (\cite{madhava}) 62, or at least its version
printed in \citet[361]{madhava}. The readings of the \cite{madhava}
as it stands now usually side with the vulgate version rather than
with the Nepalese. In addition to the basic text, there are several
valuable pointers made in the \emph{Madhukośa}, an early commentary
on the \cite{madhava}. This part of the text is authored by
Śrīkaṇṭhadatta, who was most like a direct student of Vijarakṣita.
The latter wrote the first part of the \emph{Madhukośa}, up to
chapter 32, and, what is more, can be dated to the late eleventh or
early twelfth centuries.\footcite[22--26]{meul-1974}
    
    Another most interesting parallel is found in \emph{Carakasaṃhitā} 6(Ci).30.
    
    
    \section{Philological notes}
    
    \subsection{Metrical alterations}
    
The first two verses in the Nepalese version, 6.38.2.1 and 6.38.4.1, are
written in a classical variety of the \emph{upajāti} metre:
\underline{$\cup$} $\_ \cup\ \_\ \_\ \cup\ \cup\ \_\ \cup\ \_\ \_$ In
content, they are only approximately parallel to three hemistichs in
\emph{anuṣṭubh} metre found in the vulgate.\footnote{\SS\
    \Su{6.38.3--4ab}{668}.} The latter verses lack the apologetic explanation
    concerning the reasons for this chapter being taught.
        
\subsection{The original opening verses}

From verse \SS\ 6.38.5.1 onwards, the Nepalese version of the text
continues with three hemistichs in the same classical \emph{upajāti}
metre (the syllabic pattern above).\footnote{The metre of these verses is
    not perfect.} By contrast, the vulgate contains two complete verses (four
    hemistichs) in the \emph{anuṣṭubh} metre, again with only loosely-related
    content.\footnote{\SS\ \Su{6.38.4cd--6ab}{668}.} The three final
        hemistichs of this group are borrowed verbatim from the
        \CS.\footnote{\CS\ \Ca{6.30.7cd--8}{634}.} We can be sure of the
            direction of borrowing because one of these shared verses says that the
            twenty kinds of diseases of the female reproductive system “have already
            been indicated in the \emph{\se{rogasaṃgraha}{Compendium of
                    Diseases}}”.\footnote{\SS\ \Su{6.38.5ab}{668}: 
                    \dev{viṃśatirvyāpado
                yonernirdiṣṭā rogasaṃgrahe//} $\leftarrow$ \CS\ 
                \Ca{6.30.7cd}{634}.} This
                statement does not make any sense  in the context of the \SS, where 
                no
                such Compendium exists.\footnote{The remark was not commented 
                on by
                    Ḍalhaṇa.}  By contrast, in the \CS\ this reference points back to 
                    chapter
                    \Ca{1.19}{109--112}, which calls itself “The Compendium of
                    Diseases”.\footnote{\CS\ \Ca{1.19.9cd}{112}: \dev{rogādhyāye
                        prakāśitāḥ}.} This Compendium lists all the diseases dealt with 
                        in later
                        sections of the text, and specifically mentions the twenty 
                        diseases of
                        female reproductive system.\footnote{\CS\ \Ca{1.19.3}{110}:
                            \dev{viṃśatiryonivyāpadaḥ/}}  Even the vocabulary and 
                            wording of this
                            passage is identical to the later verses. It is beyond doubt that 
                            this
                            passage originated in the \CS\ and was borrowed by the 
                            editors of the
                            vulgate text of the \SS.\footnote{The above three hemistichs in
                                \emph{anuṣṭubh} are also repeated in the \cite{madhava} 
                                62.1--2ab. Given
                                that the subsequent verses in the \cite{madhava} stem 
                                from the \SS, it is
                                likely that \cite{madhava} 62.1--2ab too was borrowed 
                                from from the \SS\
                                and not from its original location in the \CS).}
    
    \newpage
    \section{Translation}
    
    \begin{translation}
        
        \item [1] And now I shall explain the countermeasures against
\se{yonivyāpat}{disorders of the female reproductive
    system}.\footnote{On this broad understanding of the term
    \emph{yoni} as “female reproductive system” see \cite[pp.\
    572--5]{das-2003}.}
            
            
            \item [*3] For good men, a woman is the most pleasurable
thing. Therefore a physician should diligently attend to
the diseases located in the \se{yoni}{female reproductive
    system}, because he is entirely devoted to it (that is, to
curing these diseases) for the sake of (people's)
happiness.%
% DW: because he really is dependent ... for the sake
% of...
% yasmāt pramāda .... , ataḥ (= tasmāt) vaidyaḥ...
% samuprakrameta, yasmāt ...
%tadadhīna...
% Jason suggests: to parse ``yasmāt sukhārtham [asti],
% [tasmāt] tadadhīnaḥ''
\footnote{%
    As our translation indicates, the sentence
    construction does not allow an unambiguous
    identification of who or what is the referent of the
    pronoun \emph{tad} in the compound form
    \emph{tadadhīna} ‘devoted to it.’ Our current
    understanding is that \emph{tad} refers to the ‘most
    pleasurable thing’ mentioned in pāda a. It could,
    however, also refer to ‘them,’ that is, the ‘good
    men.’%
    }
                
                \item [*4] A corrupted \se{yoni}{female reproductive system} cannot 
                consume 
                \se{bīja}{semen}, and therefore, the woman cannot take a fetus (that is, 
                become pregnant). She gets severe 
                \se{arśas}{prolapses}, 
                \se{gulma}{abdominal lump} and similarly many other 
                \se{roga}{diseases}.
                % think about ``praduṣṭa-'' (!!!):
                %% Martha suggested ``ruined'' 
                %%% COMM (ak): it fits really well here, but what to do about praduṣṭa- doṣas?
                %% spoiled, corrupted, “vitiated”?
         
         
    
    \item [*5] \se{doṣa}{Humours}, \se{vāta}{wind}, etc., corrupted due to
\se{mithyopacāra}{faulty medical treatment},\footnote{In our translation
    of the compound \dev{mithyopacāra}, we decided for the technical meaning
    of the term \dev{upacāra}, that is, “medical application” or
    “treatment.” The combination \dev{mithyā}+\dev{upa-}$\surd$\dev{car} is
    attested several times in medical literature. At least once, at \CS\
    \Ca{3.3.38}{245}, it is given an explicit gloss by Cakrapāṇidatta:
    \dev{mithyopacaritāniti asamyak cikitsitān} “\ldots\ given improper
    therapy”. In the \SS\ \parencite{vulgate}, it is used once in a passage
    (\Su{6.18.30}{635}) where it refers specifically to the wrong
    application of \se{tarpaṇa}{irrigation} and \se{puṭapāka}{roasting},
    both of which are mentioned in the previous verse. Another use of the
    compound in a similar meaning is found in a citation from Bhoja's work
    quoted by Gayadāsa at \SS\ \Su{2.5.17}{287}: \dev{śvitraṃ tu dvividhaṃ
    proktaṃ doṣajaṃ vraṇajaṃ tathā/ tatra mithyopacārāddhi vraṇasya vraṇajaṃ
    smṛtam//} “\ldots\ arises from wrong treatment of the wound.” In
    contrast to this, the parallel verse in \SS\ \Su{6.38.5ab}{668} = \CS\
    \Ca{6.30.8}{634} = \cite{madhava} 62.1 reads \dev{mithyācāra} “wrong
    conduct.” All commentators (Cakrapāṇidatta on the \CS, Śrīkaṇṭhadatta on
    the \cite{madhava}, and Ḍalhaṇa on the \SS) explain that the wrong
    conduct stands here specifically for unwholesome diet. The parallel in
    \AH\ \Ah{6.33.27}{895} = \AS\ \As{6.38.34}{829} plainly reads
    \dev{duṣtabhojana} “corrupted food” instead.} sexual activity, fate, and
    also \se{doṣa}{defects} of \se{ārtava}{menstrual blood} and
    \se{bīja}{semen}, produce various diseases in the \se{yoni}{female
        reproductive organ}. These 20 diseases are taught here distinctly and
    one by one along with their \se{bheṣaja}{treatment}, \se{hetu}{causes}
    and \se{cihna}{signs}. %%
    % mithyopacāra-/ mithyācāra
    % -->technically, this usually refers to “wrong
    %%treatment”/ “faulty
    %% mithyācāra- is glossed in Caraka + Mādhavanidāna --- asamyagāhārācara
    %% Aṣṭāṅga... use ``duṣṭabhojana-''
    %%% Mithyopacāra in Caraka Vi.3.38 is glossed with “asamyak cikitsita-”
    % Cf.:
    %% \emph{Mādhavanidāna} 62.1-2ab: \emph{viṃśatir vyāpado yonau nirdiṣṭā
    %%rogasaṃgrahe/ mithyācāreṇa tāḥ strīṇāṃ praduṣṭenārtavena ca// jāyante
    %%bījadoṣāc ca daivāc ca śṛṇu tāḥ pṛthak}
    % Cf.:
    %% \emph{Aṣṭāṅgahṛdaya} 6.33.28ab-29ab = \emph{Aṣṭāṅgasaṃgraha}
    %%6.38.34:
    %% \emph{{viṃśatir vyāpado yoner jāyante duṣṭabhojanāt/
    %%viṣamasthāṅgaśayanabhṛśamaithunasevanaiḥ/ duṣṭārtavād apadravyair
    %%bījadoṣeṇa daivataḥ //}
    % NOTE: Several other interpretations are possible:
    %% - suratakriyāyāḥ can be Genitive
    %% - mithyopacāra- can wrong treatment specifically
    %% - chose ``amorous'' rather than ``sexual'' to capture the feeling of
    %%\emph{surata} better.
    %% the intro verses are identical with Carakasaṃhitā 6.30.7--8. There,
    % in
    % Cakra
    %%comments on ``rogasaṃgraha'' saying that it refers to 1.19.3 !!!


    \item [*6.1] Because of \se{vāta}{wind}, \se{yoni}{female reproductive organ} 
    becomes:
    %All feminines refer to {yoni}, and hence are various kinds of adjectival forms 
    %(ktānta-s, bahuvrīhi-s, or other kinds of adjectives). Perhaps, instead of “occur” 
    %one could say smth like “\se{yoni}{} becomes...”, There is actually “bhavet” in 
    %6.
    
    \begin{enumerate}
        \item \se{udāvartā}{udāvartā},
        \item called \se{vandhyā}{Infertile}, and
        \item \se{plutā}{Sprung},
        \item \se{pariplutā}{Flooded}, and
        \item \se{vātalā}{Windy}.
    \end{enumerate}
    
    \item [*6.2] And because of \se{pitta}{choler}, occur:
    \begin{enumerate}
        \item \se{raktakṣayā}{With bloodloss},
        % This is, according to the analysis given in a later verse, a rare case of a 
        %``genetive bahuvrīhi'' (A reads it as a grammatically more correct Locative 
        %bahuvrīhi!).
        \item \se{vāminī}{Vomiting}, and
        \item \se{sraṃsanī}{Causing a Fall},
        % this one is most likely formed with some kind od LyuṬ (or smth like that) + 
        %ṄīP. See similar formations like jananī, hananī, śāmanī, śamanī
        % the Aborting One (?)
        \item \se{putraghnī}{Child-murderess}, and also
        \item \se{pittalā}{Bilious / Choleric}.
    \end{enumerate}
\item [*7.1] And because of \se{kapha}{phlegm} occur:
\begin{enumerate}
    \item \se{atyānandā}{Extremely Excited},
    \item \se{karṇinī}{Protuberant}, and
    % @@NOTE a conjunctive error (errores coniunctivi)
    \item[3.\ \& 4.] two \se{caraṇī}{Caraṇī}, and
    \item[5.] other \se{śleṣmalā}{Phlegmatic}.
\end{enumerate}

\item [*7.2] And similarly there are other (kinds of morbid female reproductive 
system) involving all \emph{doṣa}s:
\begin{enumerate}
    \item \se{śaṇḍhī}{Impotent},
    \item \se{aṇḍīnī}{With testicles},
    \item two \se{mahatī}{Huge},
    \item \se{sūcīvaktrā}{With a needle-like opening},
    \item \se{sarvātmikā}{Sarvātmikā}.
\end{enumerate}
\end{translation}

