\documentclass [11pt]{book}%[14pt]{extarticle}
\usepackage{SS-critical-edition} % all the TeX setup is in SS-critical-edition.sty
\addtolength{\textheight}{-5ex}

\title{\textenglish{Suśrutasaṃhitā, sūtrasthāna, adhyāya 16}}
\date{Draft of \today\\ \copyright\ \href{http://sushrutaproject.org}{Suśruta Project}}

\setcounter{page}{47}
\setcounter{chapter}{4} % one less that the next \chapter


\begin{document}
    \include{sanskrit-hyphenations}
    \Large
\begin{sanskrit}
%        \raggedright
\sloppypar
  
\input{sanskrit-hyphenations}

\lineation{section} % or page?
\begingroup
\beginnumbering

\chapter{\textenglish{Critical edition of Sūtrasthāna 16}}

%\emph{\footnotesize The first layer of the apparatus gives the available witnesses, when 
%they 
%change.  The next layer gives the variants of the Nepalese MSS
%    (K, \N\ and H); the second layer gives those of the vulgate edition, Trivikramaja
%    Ācārya 1938 (A). When we adopt a reading of A against the manuscripts, we note
%    that in the first layer of variants.  A wavy underline (\uwave{xyz}) marks an
%    unclear reading.}
%
%\bigskip

\pstart
\edlabel{K-start}%
\edtext{athātaḥ}{
    \linenum{|\xlineref{K-start}|||\xlineref{K-end}||}
    \lemma{athātaḥ--°viddhaliṅgam}
    \Bfootnote{\textenglish{MSS \K, \H, and \N}}}
\edtext{karṇṇavyadhavidhiṃ}{\Dfootnote{karṇavyadhabandhavidhimadhyāyaṃ  
\A.}
} \edtext{vyākhyāsyāmaḥ}{\lemma{vyākhyāsyāmaḥ}\Afootnote{vyā 
\K.}\Dfootnote{\add\ 
yathovāca bhagavāndhanvantariḥ// \A.}
}//1//\vulgate{1}%
\pend

\pstart
rakṣābhūṣaṇanimittambālasya karṇṇau \edtext{vyadhayet}{
\Dfootnote{vidhyete  \A.}
}/ tau ṣaṣṭhe \edtext{māse}{
  \Dfootnote{māsi  \A.}
} \edtext{saptame}{
  \Afootnote{\omit\ \N.}
} vā śuklapakṣe praśasteṣu tithikaraṇamuhūrttanakṣatreṣu 
\edtext{kṛtamaṅgalasvastivācanaṃ}{
\lemma{°maṅgalasvastivācanaṃ}\Afootnote{\A; °maṅgalaṃ svastivācanaṃ  \K, \H, \N.}
}\Note{The compound \textsanskrit{kṛtamaṅgalasvastivācanaṃ} is an emendation
based on the similar text at \Su{3.2.25}{346}.} \edtext{dhātryaṅke}{\Afootnote{dhātryaṅko 
\K.}\Dfootnote{\add\  kumāradharāṅke vā  \A.}
} \edtext{kumāra}{\lemma{kumāra}
  \Afootnote{kumārakam \N.}
}\edtext{mupaveśyābhisāntvayamāno}{
  \Dfootnote{upaveśya bālakrīḍanakaiḥ pralobhyābhisāntvayan  \A.}
}\Note{The \emph{ātmanepada} participle is a permitted 
form, although the vulgate has the 
\emph{parasmaipada}. Ḍalhaṇa, on \Su{1.16.3}{76}, recorded the alternative reading 
\textsanskrit{bhakṣyaviśeṣairvā} before \textsanskrit{bālakrīḍanakaiḥ pralobhya} in the 
vulgate.} 
bhiṣagvāmahastenākṛṣya \edtext{karṇṇa}{\lemma{karṇṇan}
  \Dfootnote{karṇaṃ  \A.}
}ndaivakṛte \edtext{chidre}{
  \Dfootnote{chidra  \A.} \Dfootnote{\add\ ādityakarāvabhāsite śanaiḥ śanair  \A.}
  } dakṣiṇahastena \edtext{ṛju}{
  \Afootnote{rjum \N\ \H.}
  \Dfootnote{rju \A}
} \edtext{vidhyet}{
 \Dfootnote{\add\ pratanukaṃ sūcyā bahalamārayā  \A.}
 }/  dakṣiṇaṃ kumārasya vāmaṅkanyāyāḥ\edlabel{SS.1.16.3--32}\edtext{/}{
  \linenum{|\xlineref{SS.1.16.3--32}}\lemma{kanyāyāḥ}\Dfootnote{kumāryāḥ  
  \A.}
} \edtext{pratanuṃ}{
  \Afootnote{pratanū \N\ \H.}
} sūcyā\edlabel{SS.1.16.3--35} bahalamā\edtext{rayā}{\lemma{ārayā}\Dfootnote{\add\ 
tataḥ picuvartiṃ praveśayet// \A.}
}\edlabel{SS.1.16.3--37}//2//\vulgate{3}
 \pend

\newpage

\pstart
śoṇita\edtext{bahutve}{\Dfootnote{bahutvena 
\A.}}\edtext{'tivedanāyāṃ}{\Dfootnote{vedanayā \A.}}
cānyadeśaviddhamiti jānīyāt/ \edtext{nirupadravatā}{\lemma{°dravatā}
  \Dfootnote{°dravatayā  \A.}
} taddeśaviddhaliṅgam\edlabel{SS.1.16.4--6} \edtext{//3//}{
  \linenum{|\xlineref{SS.1.16.4--6}}\lemma{°viddhaliṅgam}\Dfootnote{°viddhamiti  \A.}
}\edlabel{K-end}\vulgate{4}
\pend

\pstart
\edtext{tatra\emph{\edlabel{SS.1.16.5--0}}}{
  \Dfootnote{tatrājñena  \A.}
  \Bfootnote{\textenglish{MSS \H\ and \N.  From here to the end of 1.16, MS \K\ is missing 
  a folio.}}
} \edtext{yadṛcchāviddhāyāṃ}{
  \Dfootnote{yadṛcchayā viddhāsu  \A.}
} \edtext{sirāyāma}{\lemma{sirāyām}
    \Dfootnote{sirāsu \A.}
    \Dfootnote{\add\ kālikāmarmarikālohitikāsūpadravā bhavanti/ tatra kālikāyāṃ \A.}
}jñena \edtext{jvara}{\lemma{jvara°}\Dfootnote{jvaro}
}\edtext{dāha}{\lemma{°dāha°}\Dfootnote{dāhaḥ}
}\edtext{śvayathu}{\lemma{°śvayathu°}
  \Afootnote{\emended; °śvaya\uwave{thur} N; °śvayathur \H.} 
  \Dfootnote{śvayathur \A.}
}\edtext{vedanā}{\lemma{°vedanā°}
  \Dfootnote{\add\  ca bhavati marmarikāyāṃ vedanā jvaro  \A.}
}\edtext{\-granthi}{\lemma{°granthi°}
  \Dfootnote{granthayaś \A.} \Dfootnote{\add\ ca lohitikāyāṃ \A.}
}\-manyā\-stambhā\-patānaka\-śirograhakarṇṇaśūlāni \edtext{bhavanti}{
\lemma{bhavanti}  \Dfootnote{\add\  teṣu yathāsvaṃ pratikurvīta// 
kliṣṭajihmāpraśastasūcīvyadhādgāḍhataravartitvād  \A.}
}//4//\vulgate{5}
\pend

%\newpage % the large paragraphed notes make page-breaking hard

\pstart
doṣasamudayāda\edlabel{SS.1.16.6--0}praśastavyadhādvā\edtext{}{
    \lemma{vā}
    \Dfootnote{\add\  yatra saṃrambho vedanā vā bhavati  \A.}
} tatra varttima\edtext{pahṛtya}{
    \lemma{apahṛtya}
    \Dfootnote{upahṛtyāśu  \A.}
} 
\edtext{yavamadhukamañjiṣṭhāgandharvvahastamūlai}{\lemma{°gandharvvahastamūlair}
    \Afootnote{°gandarvahastamūlai \N.}
    \lemma{yavamadhuka°} 
    \Dfootnote{madhukairaṇḍamūla° \A.}
    \lemma{°mañjiṣṭhāgandharvvahastamūlair} 
    \Dfootnote{°mañjiṣṭhāyavatilakalkair \A.}
}rmmadhu\-ghṛta\-pragāḍhairālepayet\edtext{/}{\lemma{ālepayet}
  \Dfootnote{\add\  tāvadyāvatsurūḍha iti//  \A.}
} surūḍhañcainampunarvvi\edtext{dhyet}{
    \lemma{vvidhyet}
    \Dfootnote{\add\  vidhānaṃ tu pūrvoktameva//  \A.}}//5//\Note{Ḍalhaṇa,
    on \Su{1.16.6}{77}, stated that some do not read 
    \textsanskrit{surūḍhañcainampunarvidhyet}.}\vulgate{6}
\pend

\pstart
\edtext{samya}{\lemma{samyag°}\Dfootnote{tatra samyag° 
\A.}}gviddhamā\edlabel{SS.1.16.7--1}matailapariṣekeṇopacaret\edlabel{SS.1.16.7--2} 
\edtext{/}{
  \linenum{|\xlineref{SS.1.16.7--2}}\lemma{°pariṣekeṇopa°}\Afootnote{°pariṣekaṇopa° 
  \H.}\lemma{āmatailapariṣekeṇopacaret}\Dfootnote{āmatailena pariṣecayet \A}
} tryahāttrya\edtext{hādva}{
    \lemma{tryahād}
    \Dfootnote{\add\  ca  \A.}
}rttiṃ \edtext{sthūlatarīṅ}{\lemma{°tarīṅ}
  \Dfootnote{°tarāṃ \A}
  \Afootnote{°tarīṃ \N}}\Note{The unusual form \textsanskrit{sthūlatarīṃ} is supported 
  by both manuscripts and we have retained it in spite of only meagre evidence for the 
  form in epic Sanskrit.}
  \edtext{kurvvīta}{
  \Dfootnote{dadyāt  \A.}
} pariṣekañca tameva//6//\vulgate{7}
\pend

\newpage

\pstart
 atha\edlabel{SS.1.16.8--1} vyapagatadoṣopadrave 
 karṇṇe\edtext{'laṃpravarddhanārthaṃ}{
      \Dfootnote{\omit\ \A.}
} \edtext{laghupravarddhanaka}{
    \lemma{laghupravarddhanakam}
    \Afootnote{\uwave{la} pravardhanakā\uwave{mo} N; 
 pravardhanakā\uuline{mā} “{mai}”  \H.}
\Dfootnote{laghuva° \A}
}māmuñcet\edlabel{SS.1.16.8--6} \edtext{//7//}{
 \linenum{|\xlineref{SS.1.16.8--6}}\lemma{āmuñcet}
    \Afootnote{\emended; muñcet  \N\ \H.}
    \Dfootnote{kuryāt// \A}
}\vulgate{8}
\pend\medskip

%\newpage

\pstart
\begin{verse}
 evaṃ\edlabel{SS.1.16.9--1} \edtext{samvarddhitaḥ}{
  \Dfootnote{vivardhitaḥ  \A.}
} karṇṇaśchidyate tu dvidhā nṛṇām\edlabel{SS.1.16.9--7}\edtext{/}{
  \linenum{|\xlineref{SS.1.16.9--7}}\lemma{nṛṇām}\Afootnote{nṛṇā  \N.}
} \\
\edtext{}{
  \Afootnote{\A; doṣaṭo \N\ \H.}
}doṣato vābhighātādvā \edtext{sandhānā}{
    \lemma{sandhānān}\Dfootnote{sandhānaṃ  \A.}
}ntasya me śṛṇu//8// \vulgate{9}
\end{verse}
%\pend\bigskip


\pstart
 tatra\edlabel{SS.1.16.10--1} samāsena \edtext{pañcadaśasandhānākṛtayo}{\lemma{°sandhānā°}
  \Afootnote{°sandhā° \N.}
  \lemma{°daśasandhānākṛtayo}
  \Dfootnote{°daśakarṇabandhākṛtayaḥ \A.}
} bhavanti\edlabel{SS.1.16.10--4}\edtext{/}{
  \linenum{|\xlineref{SS.1.16.10--4}}
  \lemma{bhavanti}
  \Dfootnote{\omit\  \A.}
}\Note{Cakrapāṇidatta, on \Cakra{1.16.9–13}{128}, and Ḍalhaṇa, on 
\Su{1.16.10}{78}, pointed out that others read 
\textsanskrit{pañcadaśakarṇakṛtayaḥ} (instead of 
\textsanskrit{pañcadaśasandhānākṛtayaḥ}). At the same place, Ḍalhaṇa 
also mentioned that some read \textsanskrit{samunnatasamobhayapāliḥ} 
(instead of \textsanskrit{vṛttāyatasamobhayapālir}) and others do not read 
\textsanskrit{saṃkṣiptādayaḥ pañcāsādhyāḥ}.} tadyathā/ 
nemīsandhānakaḥ\edlabel{SS.1.16.10--9}\edtext{/}{
    \linenum{|\xlineref{SS.1.16.10--9}}
    \lemma{nemī°}
    \Dfootnote{nemi° \A.}
} utpalabhedyakaḥ/ vallūrakaḥ/ āsaṅgimaḥ/ gaṇḍakarṇṇaḥ/ āhāryaḥ/ nirvvedhimaḥ/ 
vyāyojimaḥ/ kapāṭasandhikaḥ/ 
arddhakapāṭasandhikaḥ\edtext{/}{
    \lemma{arddhakapāṭasandhikaḥ}
    \lemma{arddhakapāṭasandhikaḥ}
    \Afootnote{\omit\ \N.}
    \lemma{°sandhikaḥ/ arddha°}
    \Dfootnote{°sandhiko 'rddha°\A}
} saṅkṣiptaḥ/ hīnakarṇṇaḥ/ vallīkarṇṇaḥ/ 
yaṣṭīkarṇṇaḥ\edtext{/}{\lemma{yaṣṭī°}\Dfootnote{yaṣṭi°  \A.}
}
kākauṣṭhaḥ\edtext{/}{\lemma{kākauṣṭhaḥ}\Afootnote{kākauṣṭha\uwave{bhaḥ} 
\H.}\Dfootnote{kākauṣṭhaka \A.}
} iti\edtext{/}{\lemma{iti}\Afootnote{ti \H.}
} teṣu \edtext{tatra}{
  \Dfootnote{\omit\  \A.}
} \edtext{pṛthulāyatasamobhayapāli}{\lemma{°yatasamo°} \Afootnote{\A; °yasamo° \H; 
°tasamo \N.}
}rnemīsandhānakaḥ\edtext{/} {\lemma{nemī°}\Dfootnote{nemi°  \A.}
} vṛttāyatasamobhayapālirutpalabhedyakaḥ 
\edtext{/}{
        \lemma{°bhedyakaḥ}
        \Afootnote{°bhedyaḥ N; °bhedakaḥ \H.}
} hrasvavṛttasamobhayapālirvallūrakarṇṇakaḥ\edtext{/}{
    \lemma{vallūra°}
    \Afootnote{valūra° \N.}
    \Dfootnote{vallūrakaḥ  \A.}
} abhyantaradīrghaikapālirāsaṅgimaḥ/ \edtext{bāhya}{
    \Afootnote{\A; bāhyaika \N\ \H.}
}dīrghaikapālirggaṇḍakarṇṇakaḥ/ apālirubhayato'pyāhāryaḥ/ 
\edtext{pīṭhopamapāli}{\lemma{pīṭhopamapālir}\Dfootnote{\add\ ubhayataḥ 
kṣīṇaputrikāśrito  \A.}
}rnirvvedhimaḥ/ 
\edtext{aṇusthūlasamaviṣamapāli}{
    \lemma{aṇusthūla°}
    \Afootnote{aśusthūla° \H}
    \Dfootnote{sthūlāṇu°  \A.}
}rvyāyojimaḥ/ abhyantaradīrghaikapāliritarālpapāliḥ 
kapāṭasandhikaḥ\edtext{/}{
    \lemma{kapāṭa°}
    \Afootnote{kavāṭā° \H.}
} bāhyadīrghaikapāliritarālpapāliścārddhakapāṭasandhikaḥ\edtext{/}{
    \lemma{cārddhakapāṭa°}
    \Afootnote{\emended; vārddhakavāṭa° \H; 
 cārddhakavāpa° \N.}
    \lemma{cārddha°} 
    \Dfootnote{ardha° \A.}
} \edtext{tatraite}{
  \Dfootnote{tatra  \A.}
} \edtext{daśakarṇṇasandhivikalpā}{
  \Dfootnote{daśaite karṇabandhavikalpāḥ  \A.}
} bandhyā bhavanti\edtext{/}{
    \lemma{bandhyā bhavanti}
    \Dfootnote{sādhyāḥ  \A.}
} \edtext{teṣā}{
    \lemma{teṣān}
    \Dfootnote{teṣāṃ  \A.}
}\edtext{nnāmabhi}{
    \lemma{nāmabhir}
  \Dfootnote{svanāmabhir  \A.}
}revākṛtayaḥ prāyeṇa vyākhyātāḥ/ saṃkṣiptādayaḥ pañcāsādhyāḥ/ tatra 
\edtext{śuṣkaśaṣkuli}{
    \lemma{śuṣkaśaṣkulir}
  \Dfootnote{\add\  utsannapālir  \A.}
}ritarālpapāliḥ saṃkṣiptaḥ/ anadhiṣṭhānapāliḥ 
\edtext{paryantayośca}{
    \Afootnote{\omit\ \N.}
    \lemma{ca}  
    \Dfootnote{\omit\  \A.}
} kṣīṇamāṃso hīnakarṇṇaḥ/ \edtext{tanuviṣamapāli}{
    \lemma{°ṣamapālir}
    \Dfootnote{°ṣamālpapālir  \A.}
}rvallīkarṇṇaḥ/ 
\edtext{granthitamāṃsaḥ}{
    \lemma{granthitamāṃsaḥ}
    \Afootnote{°granthitamānsaḥ \N\ \H.}
    \Dfootnote{grathitamāṃsa° \A.}
} 
\edtext{stabdhasirātatasūkṣmapāli}{
        \lemma{°sirātatasūkṣma°}
        \Dfootnote{°sirāsaṃtatasūkṣma° \A.}
}ryaṣṭīkarṇṇaḥ/ 
\edtext{nirmmāṃsasaṃkṣiptāgrālpaśoṇitapāliḥ}{\lemma{nirmāṃsa°}\Afootnote{\A; 
nimāsa° N; nirmmānsa° \H.}
} \edtext{kākauṣṭha}{\Dfootnote{kākauṣṭhaka  \A.}
} iti/ baddheṣva\edtext{pi}{
    \lemma{api}
    \Dfootnote{\add\  tu śopha  \A.}
} \edtext{dāha}{\lemma{°dāha°}\Dfootnote{\add\  °rāga°  \A.}
}\edtext{pāka}{\lemma{°pāka°}
  \Dfootnote{\add\  °piḍakā°  \A.}
}\edtext{srāva}{\lemma{°srāva°}
  \Afootnote{°śrāva° \H.}
}\edtext{śopha}{\lemma{°śopha°}
  \Afootnote{°sopha° \N.}\Dfootnote{\omit\ \A.} 
}yuktā na siddhimupayānti//9//\Note{The vulgate passage inserted between 9 and 10 
(from
\textsanskrit{bhavanti cātra} to \textsanskrit{śāstravit}) was probably
also absent in the version of the \emph{Suśrutasaṃhitā}
commented on by Cakrapāṇi, who cited it in his commentary as being
“read by some” in regard to the joins (\textsanskrit{sandhāna})
that they describe (\Cakra{1.16.9–13}{128}).}\vulgate{10}
 \pend
% \vspace{-.8\baselineskip}
\newpage

\pstart
\edtext{}{
  \lemma{\insertedpassage}\Dfootnote{bhavanti cātra/ yasya pālidvayamapi 
  karṇasya 
  na bhavediha/ karṇapīṭhaṃ same madhye 
  tasya viddhvā vivardhayet// bāhyāyāmiha dīrghāyāṃ 
  sandhirābhyantaro bhavet/ ābhyantarāyāṃ dīrghāyāṃ 
  bāhyasandhirudāhṛtaḥ// ekaiva tu bhavetpāliḥ sthūlā 
  pṛthvī sthirā ca yā/ tāṃ dvidhā pāṭayitvā tu chittvā 
  copari sandhayet// gaṇḍādutpāṭya māṃsena 
  sānubandhena jīvatā/ karṇapālīmāpālestu 
  kuryānnirlikhya śāstravit// \A.}
}
%\vulgate{11--14}
\pend

\pstart
 \edtext{\edlabel{SS.1.16.15--1}}{
     \label{ato}
  \Afootnote{tato \N.}
}\edtext{ato'nyatamasya}{
    \label{'nyatamasya}
  \Dfootnote{'nyatamaṃ  \A.}
} bandhañcikīrṣuḥ 
\edtext{agropaharaṇīyoktopasambhṛtasambhāraḥ}{\lemma{°paharaṇīyo°}
  \Afootnote{°pasaṃharaṇīyo° \N.}\lemma{°sambhāraḥ}\Dfootnote{°sambhāraṃ  \A.}
} viśeṣataścā\edtext{tropaharet}{
    \lemma{cātropaharet}
    \Afootnote{\A; cāgropaharaṇīyāt \N\ \H.}
}\Note{\textsanskrit{viśeṣataścāgropaharaṇīyāt} of the MSS has been emended to 
\textsanskrit{viśeṣataścātropaharet} to make sense of the list of ingredients, which is in the 
accusative case. Also, the repetition of \textsanskrit{agropaharaṇīyāt} in the 
Nepalese version suggests that its second occurrence, which does not make 
good sense here, is a dittographic error.} \edtext{surāmaṇḍakṣīra}{
    \lemma{surāmaṇḍakṣīram}
    \Dfootnote{surāmaṇḍaṃ kṣīram  \A.}
}mudakaṃ 
\edtext{dhānyāmlakapālacūrṇṇañce}{\lemma{dhānyāmlakapāla°}\Dfootnote{dhānyāmlaṃ
 kapāla° \A.}
}ti/ tato'ṅganāṃ \edtext{puruṣa}{	
    \lemma{puruṣam}
  \Afootnote{puruṣañ \N.}
}mvā grathitakeśāntaṃ laghubhuktavantamāptaiḥ suparigṛhītaṃ 
\edtext{ca kṛtvā}{
    \lemma{ca kṛtvā}\Afootnote{\A; \omit\ \N\ \H.}
} \edtext{bandhā}{
    \lemma{bandhān}
  \Dfootnote{bandham  \A.}
}\edtext{nupadhārya}{
    \lemma{upadhārya}
  \Afootnote{upapādya \H.}
} \edtext{chedyabhedyalekhyavyadhanai}{\lemma{°vyadhanair}
  \Dfootnote{\add\  upapannair  \A.}
}rupapādya karṇṇaśoṇita\edlabel{SS.1.16.15--27}\edtext{mavekṣyaita}{
    \lemma{avekṣyaitad}
    \linenum{|\xlineref{SS.1.16.15--27}}
    \lemma{°śoṇitamavekṣyaitad}
    \Afootnote{°śoṇitata avekṣyetad \N.}
    \lemma{avekṣyaitad}  
    \Dfootnote{avekṣya  \A.}
}dduṣṭama\edtext{duṣṭa}{
    \lemma{aduṣṭam}
  \Afootnote{aduṣṭaś \N.}
}\edtext{mveti}{
    \lemma{veti}
    \Afootnote{\A; ceti \N\ \H.}
} \edtext{tato}{
  \Dfootnote{tatra  \A.}
} vātaduṣṭe \edtext{dhānyāmlodakābhyāṃ}{\lemma{dhānyāmlo°}
  \Afootnote{dhānyāvlo° \N} \lemma{dhānyāmlodakā°} \Dfootnote{dhānyāmloṣṇodakā° \A.}
} pittaduṣṭe \edtext{śītodakapayobhyāṃ}{\lemma{śītodaka°}
  \Afootnote{śītodako° \N.}
} śleṣmaduṣṭe \edtext{surāmaṇḍodakābhyāṃ}{\lemma{°maṇḍodakā°}
  \Dfootnote{°maṇḍoṣṇodakā°  \A.}
} prakṣālya \edtext{karṇṇa}{\lemma{karṇṇam}
  \Dfootnote{karṇau  \A.}
}mpunaravalikhet\edlabel{SS.1.16.15--42}/ \edtext{anunnata}{
    \lemma{anunnatam}
    \linenum{|\xlineref{SS.1.16.15--42}}
    \lemma{avalikhet}
    \Afootnote{avalikheta \N.}
    \lemma{avalikhet/ anunnatam}
    \Dfootnote{avalikhyānunnatam \A.}
}mahīnamaviṣamañca \edtext{karṇṇasandhi}{\lemma{°sandhin}
   \Afootnote{°sandhiṃ \N\ }\Dfootnote{°sandhiṃ \A\ }
}\edtext{nniveśya}{\lemma{niveśya}
  \Dfootnote{sanniveśya  \A.}
} sthitaraktaṃ \edtext{sandarśya}{
  \Dfootnote{sandadhyāt \A.} \Dfootnote{\add\ tato  \A.}
} madhughṛtenābhyajya picuplotayora\edtext{nyatareṇāvaguṇṭhya}{
    \lemma{°guṇṭhya}
    \Afootnote{°guṇṭhyo \H.}
} \edtext{nātigāḍha}{\lemma{°gāḍhan}
    \Afootnote{°gāḍhaṃ \N.}
    \lemma{nātigāḍhan}
    \Dfootnote{sūtreṇānavagāḍhaman \A.}
}\edtext{nnātiśithilaṃ}{\lemma{nāti°}
  \Dfootnote{ati°  \A.}
} \edtext{sūtreṇāvabadhya}{\lemma{°badhya}
  \Afootnote{°baddha \N.}
  \lemma{sūtreṇāvabadhya}
  \Dfootnote{ca baddhvā \A.}
} kapālacūrṇṇenāvakīryācārikamupadiśet/ 
dvivraṇīyoktena\edlabel{SS.1.16.15--61} 
cānnenopacaret\edlabel{SS.1.16.15--62} \edtext{//10//}{
  \linenum{|\xlineref{SS.1.16.15--61}}
    \lemma{cānnenopacaret}
    \Afootnote{\uwave{upapocaret} \N.}
  \linenum{|\xlineref{SS.1.16.15--62}}\lemma{cānnenopacaret}
  \Dfootnote{ca vidhānenopacaret  \A.}
  \lemma{//10//}
    \Afootnote{\add\ bha \N.}
  \Dfootnote{\add\ bhavati cātra/ \A.}
}\vulgate{15}
\pend
%\medskip

\newpage

\pstart
\begin{verse}
%\edtext{}{\lemma{\inserted}
%  \Afootnote{bha \N.}
%  \Dfootnote{bhavati cātra/ \A.}
%}%
%\pend\bigskip
%
%
%\pstart
\edtext{vighaṭṭana\edlabel{SS.1.16.16x-1}}{
    \lemma{vighaṭṭanan}
  \Afootnote{vighaṭṭanaṃ \N.}\Dfootnote{vighaṭṭanaṃ \A.}
}ndivāsvapnaṃ vyāyāmamatibhojanam |\\  
vyavāyama\edtext{gnisantāpa}{
    \lemma{agnisantāpam} 
    \Afootnote{āgnisantāpa \N.}
}mvākśramañca\edlabel{SS.1.16.16x-10} \edtext{vivarjjayet}{
  \linenum{|\xlineref{SS.1.16.16x-10}}\lemma{vivarjjayet}\Afootnote{varjayet \N.}
}//11//\vulgate{16}
\end{verse}
\pend\medskip

\pstart
 \edtext{nātiśuddharakta\edlabel{SS.1.16.17--1}}{
    \lemma{°śuddha°}
    \Afootnote{°suddha° \N.} 
    \lemma{nāti°}
    \Dfootnote{na cāśu° \A.}
}\edtext{matipravṛttaraktaṃ}{
    \lemma{°vṛttaraktaṃ}
    \Afootnote{°vṛttaṃ raktaṃ \N.}
} kṣīṇaraktaṃ vā sandadhyāt/ sa hi vātaduṣṭe \edtext{raktabaddho'rūḍho}{
    \lemma{raktabaddho'rūḍho}
    \Afootnote{\emended; raktavaddho ruḍho N; raktabaddho rūḍho \H.}
    \Dfootnote{rakte rūḍho'pi \A.}
} \edtext{paripuṭanavā}{
    \lemma{°puṭanavān}
    \Afootnote{°puṭavām N; °puṭanavām \H.}
}nbhavati\edlabel{SS.1.16.17--12}\edtext{/}{
  \linenum{|\xlineref{SS.1.16.17--12}}\lemma{bhavati}\Dfootnote{\omit\  \A.}
} \edtext{pittaduṣṭe}{\lemma{°duṣṭe}
  \Afootnote{°duṣṭai \N.}
} gāḍhapākarāgavān\edlabel{SS.1.16.17--15}\edtext{/}{
    \linenum{|\xlineref{SS.1.16.17--15}}
    \lemma{gāḍhapākarāgavān}
    \Dfootnote{dāhapākarāgavedanāvān \A.}
} \edtext{śleṣmaduṣṭe}{\lemma{śleṣma°}
  \Afootnote{śleṣa° \N.}
} \edtext{stabdhakarṇṇaḥ}{\lemma{°karṇṇaḥ}
  \Afootnote{°varṇṇaḥ \N.}
  \lemma{stabdha°}
  \Dfootnote{stabdhaḥ \A.}
} kaṇḍūmāna\edtext{tipravṛttasrāvaḥ\edlabel{SS.1.16.17--20}}{
    \lemma{°srāvaḥ}
    \Afootnote{°śrāvaḥ \H.}
} \edtext{śophavā}{
  \linenum{|\xlineref{SS.1.16.17--20}}
  \lemma{°vṛttasrāvaḥ śophavān}
  \Dfootnote{°vṛttarakte śyāvaśophavān  \A.}
}\edtext{nkṣīṇālpamāṃso}{
    \lemma{kṣīṇālpa°}
    \Afootnote{kṣīṇo lpa° \N.}\Dfootnote{kṣīṇo'lpa° \A.}
} na vṛddhimupaiti//12//\vulgate{17}
\pend

\pstart
\edtext{}{
  \lemma{\insertedpassage}\Dfootnote{āmatailena trirātraṃ 
  pariṣecayettrirātrācca picuṃ parivartayet/ \A.}
}sa\edlabel{SS.1.16.18--1} yadā \edtext{rūḍho}{
  \Afootnote{ruḍho \N.}\Dfootnote{surūḍho \A.}
} nirupadravaḥ \edtext{karṇṇo}{
  \Dfootnote{savarṇo \A.}
} bhavati tadainaṃ śanaiḥ śanairabhivarddhayet/ \edtext{anyathā}{
  \Dfootnote{ato'nyathā  \A.}
} \edtext{saṃrambhadāhapākavedanāvā}{\lemma{°pākavedanāvān}
  \Afootnote{°pākarāgavedanāvān N; °pākavedanāvām \H.} 
  \Dfootnote{°pākarāgavedanāvān 
  \A.}
}nbhavati\edlabel{SS.1.16.18--14}\edtext{/}{ 
  \linenum{|\xlineref{SS.1.16.18--14}}\lemma{bhavati}\Dfootnote{\omit\  \A.}
} punara\edtext{pi}{
    \lemma{api}
    \Dfootnote{\omit\  \A.}
} chidyeta\edlabel{SS.1.16.18--18} \edtext{//13//}{
    \linenum{|\xlineref{SS.1.16.18--18}}\lemma{chidyeta}\Dfootnote{chidyate 
  vā \A.}
}\vulgate{18}
\pend

\newpage

\pstart
 \edtext{athāpra}{\lemma{athāpra°}\Afootnote{athāsyāḥ pra° \H.}\Dfootnote{athāsyāpra° 
 \A.}}\edtext{duṣṭasyābhivarddhanārtha}{
    \lemma{°duṣṭasyābhivarddhanārtham}\Afootnote{°duṣṭasyāvivardhanārtham
  \N.}
}mabhyaṅgaḥ\edtext{/}{\lemma{abhyaṅgaḥ}
  \Dfootnote{\add\  tadyathā  \A.}
} \edtext{godhāpratudaviṣkirānūpaudakavasāmajjāpayastailaṃ}{\lemma{°majjāpayastailaṃ}
  \Dfootnote{°majjānau payaḥ sarpistailaṃ  \A.}
} gaurasarṣapajañca yathālābhaṃ 
\edtext{saṃbhṛtyārkālarkabalātibalānantāvidārīmadhukajalaśūkaprativāpantai}{
    \lemma{°ārkālarkabalāti°}
    \Afootnote{°ārkālakavalāti° \N.} 
    \lemma{°prativāpan}
    \Afootnote{°prativāpaṃ \N.} 
    \lemma{°balānantā°} 
    \Dfootnote{°balānantāpāmārgāśvagandhā \A.}
    \lemma{°vidārīmadhukajalaśūka°} 
    \Dfootnote{°vidārigandhākṣīraśuklājalaśūkamadhuravargapayasyā°} 
    \lemma{°prativāpan}
    \Dfootnote{°prativāpaṃ \A.}
}\edtext{la}{
   \lemma{tailam}
   \Dfootnote{\add\  vā  \A.}
}mpācayitvā \edtext{svanugupta}{
    \lemma{°guptan}
    \Afootnote{°guptaṃ \N.}
    \Dfootnote{°guptaṃ \A.}
}nnidadhyāt\edlabel{SS.1.16.19--12} \edtext{//14//}{
  \linenum{|\xlineref{SS.1.16.19--12}}\lemma{nidadhyāt}\Afootnote{nidadyāt \N.}
}\Note{Ḍalhaṇa, on \Su{1.16.19}{79}, noted that some read \textsanskrit{rājasarṣapajaṃ} in 
the place of 
\textsanskrit{gaurasarṣapajaṃ}. This reading appears to have been accepted by Cakrapāṇi, 
who glossed \textsanskrit{rājasarṣapaja} as 
\textsanskrit{śvetasarṣapa} (\Cakra{1.16.18–20}{130}). 
Cakrapāṇi also said that some read sarpis in the place of \textsanskrit{payas}. 
In the compound 
beginning with \textsanskrit{arka}, Ḍalhaṇa noted that some read 
\textsanskrit{arkapuṣpī}.}\vulgate{19}
 \pend\bigskip


\pstart
\begin{verse}
 \edtext{svedito\edlabel{SS.1.16.20--1}}{
  \Afootnote{svadito \N.}
} \edtext{marditaṅka}{
    \linenum{|\xlineref{SS.1.16.20--1}}
    \lemma{svedito marditaṅ}
    \Dfootnote{sveditonmarditaṃ  \A.}
}rṇṇama\edtext{nena\edlabel{SS.1.16.20--4}}{
    \lemma{anena}
    \Afootnote{ane \kakapada\ \N.}
} \edtext{mrakṣayedbudhaḥ}{
  \linenum{|\xlineref{SS.1.16.20--4}}\lemma{anena 
  mrakṣayedbudhaḥ}\Dfootnote{snehenaitena yojayet  \A.}
}/ \\ 
\edtext{tato'nupadravaḥ}{\Afootnote{tato nupadravaḥ \H; tato nupadravam \N.}
} samyagbalavāṃśca vivarddhate//15//\Note{\N\ has a \textsanskrit{kākapāda} after 
\textsanskrit{ane}, but the missing letter (one would expect `\textsanskrit{na}') has not 
been supplied in a margin or elsewhere.}\vulgate{20}
 \end{verse}
\pend\bigskip
%

\pstart
\begin{verse}
\edtext{}{
  \lemma{\insertedpassage}\Dfootnote{yavāśvagandhāyaṣṭyāhvaistilaiścodvartanaṃ 
  hitam/  śatāvaryaśvagandhābhyāṃ payasyairaṇḍajīvanaiḥ// tailaṃ vipakvaṃ 
  sakṣīramabhyaṅgātpālivardhanam/ \A.}
}%
%\pend\bigskip
%
%
%\pstart
ye\edlabel{SS.1.16.22--1} tu karṇṇā na varddhante 
snehasvedopapāditāḥ\edlabel{SS.1.16.22--6}\edtext{/}{
  \linenum{|\xlineref{SS.1.16.22--6}}
  \lemma{snehasvedopa°}
  \Dfootnote{svedasnehopa° \A.}
}\\
 teṣā\edlabel{SS.1.16.23--1}mapāṅge
 tva\edlabel{SS.1.16.23--3}\edtext{bahiḥ}{
    \linenum{|\xlineref{SS.1.16.23--3}}
    \lemma{apāṅge tvabahiḥ}
    \Dfootnote{apāṅgadeśe tu  \A.}
    \lemma{abahiḥ}  
    \Afootnote{avarhi \N.}
} \edtext{kuryātpra}{
    \lemma{kuryāt}
  \Afootnote{kuyāt \N.}
}\edtext{chāna}{
    \lemma{prachānam}
    \Afootnote{prachannam \H.}
}meva ca\edlabel{SS.1.16.23--8} \edtext{//16//\Note{Ḍalhaṇa, on \Su{1.16.23}{80}, noted 
that some 
read 
    \textsanskrit{teṣāmapāṅgacchedyaṃ hi kāryamābhyantaraṃ bhavet}.}}{
    \linenum{|\xlineref{SS.1.16.23--8}}
    \lemma{ca}
    \Dfootnote{tu \A.}
}\vulgate{\vtop{22cd\\-23ab}}
\end{verse}
\pend\bigskip

\newpage

\pstart\begin{verse}
\edtext{}{
  \lemma{\insertedpassage}\Dfootnote{bāhyacchedaṃ na kurvīta 
  vyāpadaḥ syustato dhruvāḥ// baddhamātraṃ tu yaḥ karṇaṃ 
  sahasaivābhivardhayet/  āmakośī samādhmātaḥ kṣiprameva vimucyate// \A.}
}%
%\pend\bigskip
%
%
%\pstart
amitāḥ\edlabel{SS.1.16.26.0--1} \edtext{karṇṇabandhā}{
    \lemma{°bandhās}
    \Afootnote{°bandho \H.}
}\edtext{stu}{
    \lemma{tu}
    \Afootnote{stu \H.}
} vijñeyāḥ kuśalairiha |\\  
yo yathā \edtext{suniviṣṭaḥ}{
  \Dfootnote{suviśiṣṭaḥ  \A.}
} \edtext{syā}{
    \lemma{syāt}
  \Dfootnote{taṃ  \A.}
}ttattathā \edtext{yojaye}{
    \lemma{yojayed}
    \Afootnote{yojaye \N.}
    \lemma{yojayedbhiṣak}\Dfootnote{viniyojayet \A.}
}dbhiṣak//17//\Note{Ḍalhaṇa, on \Su{1.16.26}{80}, stated that some read 
\textsanskrit{suniviṣṭaḥ} 
(i.e., the reading of  the Nepalese version) instead of \textsanskrit{suviśiṣṭaḥ}.}
\edtext{}{
  \lemma{\insertedpassage}\Dfootnote{(karṇapālyāmayānnṝṇāṃ 
      punarvakṣyāmi suśruta !//  
  karṇapālyāṃ prakupitā vātapittakaphāstrayaḥ// 1// 
  dvidhā vā'pyatha saṃsṛṣṭāḥ kurvanti vividhā rujaḥ/  
  visphoṭaḥ stabdhatā śophaḥ pālyāṃ doṣe tu vātike 
  dāhavisphoṭajananaṃ śophaḥ pākaśca paittike/  
  kaṇḍūḥ saśvayathuḥ stambho gurutvaṃ ca kaphātmake// 3// 
  yathādoṣaṃ ca saṃśodhya kuryātteṣāṃ cikitsitam/  
  svedābhyaṅgaparīṣekaiḥ pralepāsṛgvimokṣaṇaiḥ// 4//
  mṛdvīṃ kriyāṃ bṛṃhaṇīyairyathāsvaṃ bhojanaistathā/  
  ya evaṃ vetti doṣāṇāṃ cikitsāṃ kartumarhati// 5//
  ata ūrdhvaṃ nāmaliṅgairvakṣye pālyāmupadravān//  
  atpāṭakaścotpuṭakaḥ śyāvaḥ kaṇḍūyuto bhṛśam// 6//
  avamanthaḥ sakaṇḍūko granthiko jambulastathā//  
  srāvī ca dāhavāṃścaiva śṛṇveṣāṃ kramaśaḥ kriyām// 7//
  apāmārgaḥ sarjarasaḥ pāṭalālakucatvacau//  
  utpāṭake pralepaḥ syāttailamebhiśca pācayet// 8//
  śampākaśigrupūtīkāngodāmedo'tha tadvasām//  
  vārāhaṃ gavyamaiṇeyaṃ pittaṃ sarpiśca saṃsṛjet// 9//
  lepamutpuṭake dadyāttailamebhiśca sādhitam//  
  gaurīṃ sugandhāṃ saśyāmāmanantāṃ taṇḍulīyakam// 10// 
  śyāve pralepanaṃ dadyāttailam ebhiśca sādhitam//  
  pāṭhāṃ rasāñjanaṃ kṣaudraṃ tathā syāduṣṇakāñjikam// 11// 
  dadyāllepaṃ sakaṇḍūke tailamebhiśca sādhitam//  
  vraṇībhūtasya deyaṃ syādidaṃ tailaṃ vijānatā// 12// 
  madhukakṣīrakākolījīvakādyairvipācitam//  
  godhāvarāhasarpāṇāṃ vasāḥ syuḥ kṛtabṛṃhaṇe// 13// 
  pralepanamidaṃ dadyādavasicyāvamanthake//  
  prapauṇḍarīkaṃ madhukaṃ samaṅgāṃ dhavameva ca// 14//
  tailamebhiśca saṃpakvaṃ śṛṇu kaṇḍūmataḥ kriyām//  
  sahadevā viśvadevā ajākṣīraṃ sasaindhavametairālepanaṃ 
  dadyāttailam ebhiśca sādhitam// 15//
  granthike guṭikāṃ pūrvaṃ srāvayedavapāṭya tu//  
  tataḥ saindhavacūrṇaṃ tu ghṛṣṭvā lepaṃ pradāpayet// 16//
  likhitvā tatsrutaṃ ghṛṣṭvā cūrṇairlodhrasya jambule//  
  kṣīreṇa pratisāryainaṃ śuddhaṃ saṃropayettataḥ// 17//
  madhuparṇī madhūkaṃ ca ma madhukaṃ madhunā saha//  
  lepaḥ srāviṇi dātavyastailamebhiśca sādhitam//18// 
  pañcavalkaiḥ samadhukaiḥ piṣṭaistaiśca ghṛtānvitaiḥ//  
  jīvakādyaiḥ sasarpiṣkairdahyamānaṃ pralepayet//19//) \A.}
}\vulgate{26.1} 
 \end{verse}

%\pend\bigskip

\pstart
\begin{verse}
jātaromā\emph{\edlabel{SS.1.16.25--0}} \edtext{suvartmā}{
  \Afootnote{suparmā N; suvarmmā \H.}
} ca \edtext{śliṣṭasandhiḥ}{\lemma{°sandhiḥ}
  \Afootnote{°sandhim \N.}
} samaḥ sthiraḥ |\\
surūḍho'vedano \edtext{yastu}{
  \Dfootnote{ca  \A.}
}\edtext{}{
    \lemma{tu}
    \Afootnote{\emended; tat \N\ \H.}
} taṃ karṇṇaṃ varddhayecchanaiḥ//18//\vulgate{25}
\end{verse}
\pend\bigskip

\newpage

\pstart
\begin{verse}
\edtext{viśleṣitāyā\emph{\edlabel{SS.1.16.27--0}}}{
    \lemma{viśleṣitāyām}
  \Dfootnote{viśleṣitāyāstv  \A.}
}matha \edtext{nāsikāyāṃ}{
  \Afootnote{nāsikāyā  \N.}\Dfootnote{nāsikāyā  \A.}
}\\
vakṣyāmi sandhānavidhiṃ yathāvat |\\ 
\edtext{nāsāpramāṇaṃ}{
    \lemma{°pramāṇaṃ} 
    \Afootnote{°pramāṇa° \N.}
}  \edtext{pṛthivīruhāṇāṃ}{ \lemma{°vīruhāṇāṃ} \Afootnote{°vīruhāṇam \N.}
} \\  
\edtext{patraṃ}{\Afootnote{patra \N.}} gṛhītvā tvavalambi tasya 
//19//\Note{Cakrapāṇidatta, on \Cakra{1.16.26}{133}, said that others read 
\textsanskrit{nāsāsandhānavidhim} here. 
Ḍalhaṇa, on \Su{1.16.27–31}{81}, stated that some read, \textsanskrit{chinnāṃ tu nāsikāṃ 
dṛṣṭvā 
vayaḥsthasya śarīriṇaḥ/ nāsānurūpaṃ saṃcchidya patraṃ gaṇḍe 
niveśayet//}}\vulgate{27}
 \end{verse}
\pend\bigskip


\pstart
\begin{verse}
 tena\edlabel{SS.1.16.28--1} pramāṇena hi gaṇḍapārśvād\\  
 utkṛtya \edtext{vadhraṃ}{  \Afootnote{vandhra \H.}  \Dfootnote{baddhaṃ \A.}
 } tvatha nāsikāgram |\\  
vilikhya cāśu pratisandadhīta\\ 
\edtext{taṃ}{\Dfootnote{tat  \A.}
} \edtext{sādhubaddha}{\lemma{°baddham}\Afootnote{°vaddha° \N.}  
\Dfootnote{°bandhair \A.}
}mbhiṣagapramattaḥ//20//\vulgate{28}
\end{verse}
\pend\bigskip


\pstart
\begin{verse}
\edtext{susīvitaṃ\emph{\edlabel{SS.1.16.29--0}}}{
 \lemma{susīvitaṃ} \Afootnote{\emended; susīvita N; suśīvitaṃ 
 \H.}\Dfootnote{susaṃhitaṃ \A.}
} samyagato yathāvan\\  
nāḍīdvayenābhisamīkṣya\edlabel{SS.1.16.29--5} 
\edtext{nahyet}{
  \linenum{|\xlineref{SS.1.16.29--5}}\lemma{nahyet}\Dfootnote{baddhvā \A.}
} |\\ 
\edtext{unnāmayitvā}{
  \Dfootnote{pronnamya cainām \A.}
} tva\edtext{vacūrṇṇayīta}{
    \lemma{avacūrṇṇayīta}
    \Dfootnote{avacūrṇayettu}
}\\ 
\edtext{pattāṅgayaṣṭīmadhukāñjanai}{
    \lemma{pattāṅga°} 
    \Afootnote{\emended; pattrāṅga° \H; pattaṅga° \N.}
    \Dfootnote{pataṅga° \A.}
}śca//21//\vulgate{29}
\end{verse}
\pend\bigskip


\pstart
\begin{verse}
saṃchādya\emph{\edlabel{SS.1.16.30--0}} samyakpicunā 
\edtext{vraṇantu}{
  \Afootnote{vraṇa tun \N.}\Dfootnote{sitena \A.}
}\\  
tailena siñcedasakṛttilānām |\\  
ghṛtañca pāyyaḥ sa naraḥ sujīrṇṇe\\ 
snigdho \edtext{virecyaḥ}{\Afootnote{\A; virecya \N\ \H.}
} \edtext{svayathopadeśam}{ \lemma{°deśam}
\Afootnote{°deśaḥ \N.}\lemma{svayatho°}\Dfootnote{sa yatho° \A.}
}//22//\edlabel{SS.1.16.30--18}\vulgate{30}
\end{verse}
\pend\bigskip

\newpage

\pstart
\begin{verse}
rūḍha\emph{\edlabel{SS.1.16.31--0}}ñca \edtext{sandhāna}{
    \lemma{sandhānam}
  \Afootnote{sandhām \N.}
}\edtext{mupāgataṃ}{
    \lemma{upāgataṃ}
  \Afootnote{upāgataś \H.}
} \edtext{vai}{
  \Afootnote{cai  \H.} \Dfootnote{syāt  \A.}
}\\ \edtext{tadvadhraśeṣaṃ}{\lemma{°śeṣaṃ}
  \Afootnote{°seṣan \N.}\lemma{tadvadhra°}\Dfootnote{tadardha° \A}
} tu punarnikṛntet/ 
\\
\edtext{hīna}{
    \lemma{hīnam}
    \Dfootnote{hīnāṃ  \A.}
}mpunarvarddhayituṃ \edtext{yateta}{
  \Afootnote{yatetaḥ  \N.}
}\\  
\edtext{sama\edlabel{SS.1.16.31--17}}{
    \lemma{samañ}
 \Dfootnote{samāṃ  \A.}
}ñca kuryā\edlabel{SS.1.16.31--19}dativṛddha\edtext{māṃsam}{\lemma{°māṃsam}
 \Afootnote{°mānsam \N.} \Dfootnote{°māṃsām  \A.}
}
//23//\vulgate{31}\\ iti \edtext{om}{
\Afootnote{\omit\ \N.}\Dfootnote{\omit\ \A.}
  }// 
\end{verse}
\pend

\pstart
\edtext{}{
  \lemma{\insertedpassage}
  \Dfootnote{nāḍīyogaṃ 
  vinauṣṭhasya 
  nāsāsandhānavadvidhim/  ya evameva jānīyātsa rājñaḥ kartumarhati// \A.}
}
\pend
\endnumbering
\endgroup
\end{sanskrit}
\end{document}
